\documentclass{patmorin}
\usepackage[utf8]{inputenc}
\usepackage{amsthm,amsmath,graphicx}
\usepackage{array}
\usepackage{pat}
\usepackage{hyperref}
\usepackage[dvipsnames]{xcolor}
\definecolor{linkblue}{named}{Blue}
\hypersetup{colorlinks=true, linkcolor=linkblue,  anchorcolor=linkblue,
citecolor=linkblue, filecolor=linkblue, menucolor=linkblue,
urlcolor=linkblue, pdfcreator=Me, pdfproducer=Me} \setlength{\parskip}{1ex}
\usepackage{tikz}

\usepackage{paralist}

\DeclareMathOperator{\sgn}{sgn}


\listfiles
\newcommand{\lstlabel}[1]{\label{lst:#1}}
\newcommand{\lstref}[1]{Listing~\ref{lst:#1}}
\newcommand{\Lstref}[1]{\lstref{#1}}

\DeclareMathOperator{\block}{block}
\newcommand{\naive}{na\"{\i}ve}


\newcommand{\reals}{\mathbb{R}}
\newcommand{\integers}{\mathbb{Z}}
\newcommand{\naturals}{\mathbb{N}}
\newcommand{\dist}{{d}}

\title{\MakeUppercase{Every Collinear Set is Free}}

\author{Bellairs Workshop on Geometry and Graphs 2017--18}

\begin{document}
\maketitle


\begin{abstract}
  We show that if a planar graph $G$ has a planar straight-line drawing
  in which a subset $S$ of its vertices are collinear, then there is a
  planar straight-line drawing of $G$ in which all vertices in $S$ are
  on the $y$-axis and in which they have prescribed $y$-coordinates.
  This solves an open problem posed by Ravsky and Verbitsky in 2008.
  In their terminology, we show that every collinear set is free.
  This result has applications in graph drawing, untangling, universal
  point subsets, and related areas.
\end{abstract}


\section{Introduction}

In a planar graph, $G=(V,E)$, a \emph{collinear set} is a set of vertices
$S\subset V$ such that $G$ has a planar straight-line drawing in which
all vertices in $S$ are drawn on a single line.  A collinear set $S$
is a \emph{free collinear set} if, for any collinear set of points
$X\subset\R^2$, $|X|=|S|$, $G$ has a planar straight-line drawing in
which the vertices of $S$ are drawn on the points in $X$.  Ravsky and
Verbitsky \cite{ravsky.verbitsky:on,ravsky.verbitsky:on-arxiv} ask the
following question:

\begin{quote}
   How far or close are parameters $\tilde{v}(G)$ and $\bar{v}(G)$? It
   seems that \emph{a priori} we even cannot exclude equality. To clarify
   this question, it would be helpful to (dis)prove that every collinear
   set in any straight line drawing is free.
\end{quote}

In the context of this quote, $\tilde{v}(G)$ and $\bar{v}(G)$ are the
respective sizes of the largest collinear set and largest free collinear
set in $G$.  In this note, we prove that $\tilde{v}(G)=\bar{v}(G)$ by
showing that every collinear set is a free collinear set.  

Da Lozzo \etal\ \cite{dalozzo.dujmovic.ea:drawing}
gave the following characterization of collinear sets:
\begin{thm}\thmlabel{collinear-set}
   A set $S$ of the vertices of a graph $G$ is a collinear set if and
   only if there exists a (topological) drawing of $G$ and a simple curve
   $C$ having both endpoints in a common face of $G$ and such that each
   vertex of $S$ is drawn on $C$ and the intersection of each edge with
   $C$ has at most one connected component.
\end{thm}
The surprising aspect of this characterization is that one can
simultaneously straighten the drawing of the graph so that it becomes
a straight line drawing and straighten $C$ so that it becomes (say)
the y-axis while preserving the combinatorial relationship between $C$
and $G$.


\subsection{Proof Outline}

Without loss of generality we may assume that the line we are interested
in is the y-axis.  Let $C^-$ and $C^+$ denote the finite and infinite
regions, respectively, of $\R^2\setminus C$, and let $\bar{C}^-=C^-\cup C$
and $\bar{C}^+=C^+\cup C$ denote the closure of these two sets.  We say
that an edge of $G$ \emph{crosses} $C$ if it contains one endpoint in
$C^-$ and one in $C^+$.

Tutte's convex embedding theorem \cite{tutte:how} allows one to draw an
internally 3-connected graph with the vertices of the outer face drawn
on any prescribed convex polygon.  If the vertices in $S$ form a path in
$G$, then no edge $G$ crosses $C$. In this case, it is straightforward to
prove that $S$ is a free collinear set using two applications of Tutte's
Convex Embedding Theorem \cite{tutte:how}, one on the graph induced by
$V(G)\cap\bar{C}^-$ and one on the graph induced by $V(G)\cap \bar{C}^+$.

Thus, the main difficulty comes from edges of $G$ that cross $C$.
These edges must be drawn so that they cross the y-axis in prescribed
(and arbitrarily small) intervals between the locations of vertices
in $S$.  An extreme version of this (sub)problem occurs when $Q$ is an
embedded graph in which every edge of $Q$ crosses $C$ and we are given a
prescribed location at which each edge of $Q$ should cross the y-axis.
The most difficult instances occur when $Q$ is edge-maximal, meaning
that $Q$ is a quadrangulation.

In \secref{quadrangulations} we show that, given a quadrangulation $Q$
and a Jordan curve $C$ that intersects every edge of $Q$ in exactly one
point, it is possible to find a plane straight-line drawing of $Q$ whose
edges intersect the y-axis in a prescribed set of points.  This done by
showing that a certain system of linear equations has a solution. This
proof involves some linear algebra and some arguments that use continuity.

In \secref{triangulations} we prove that every collinear set is free.
It turns out that the intuition that quadrangulations with prescribed
edge crossings is the hardest case can be made formal. In particular,
given a curve $C$, triangulation $G$ and a set $S\subset V(G)$ as in
\thmref{collinear-set}, a series of combinatorial reductions can be
performed on $G$ that convert it to a quadrangulation $Q$ on which we can
apply the results in \secref{quadrangulations} to obtain a drawing of $Q$
in which the vertices of $S$ are drawn at the appropriate location on the
y-axis. These reductions can then be undone to obtain a drawing of $G$
with the vertices of $S$ at the appropriate locations on the y-axis.







%Da
%Lozzo \etal\ \cite{dalozzo.dujmovic.ea:drawing} deal with crossing
%edges by subdividing them (placing a vertex in the interior of each
%edge) and adding each subdivision vertex to the collinear set.  Then,
%again, two applications of the convex embedding theorem give a drawing
%of $G$ with the vertices of $S$ at the correct locations on y-axis. The
%problem is that, in this drawing the subdivided edges are not drawn as
%straight-line segments, but rather as x-monotone paths consisting of two
%line segments.  They then rely on a result of Eades \etal \cite{XX} and
%Pach and T\'oth \cite{ptXX} that shows such a drawing can be straightened
%while preserving the x-coordinates of the vertices.  Thus, the vertices
%of $S$ remain on the y-axis but, unfortunately this straightening changes
%their y-coordinates.  This is enough to establish that $S$ is a collinear
%set, but not a free collinear set.
%

%
%If the vertices $S$ do not form a path, it is still possible to 
%
%
%of the interior of $C$ and one for the exterior
%
%
%It follows from Tutte's convex embedding theorem that, if the vertices of $S$ form a path in $G$ that, if the vertices of $S$ form a path in $G$
%
%
%Without loss of generality, we may assume that the the graph we are
%interested in is a (topological) triangulation $T$ and the line we are
%interested in is the y-axis.
%
%Our proof has three steps:
%\begin{enumerate}
%   \item Using combinatorial operations---edge subdivision, edge
%   contraction, and the removal of separating triangles---and induction
%   on the number of vertices, we reduce to a problem in which $T$
%   has no separating triangles and every edge of $T$ is incident to a
%   triangle that has two edges intersecting the y-axis and $S$ is an
%   independent set that is on the y-axis.
%
%   \item Using local operations---removing edges and vertices and
%   introducting edges---on $T$ we obtain a (topological) quadrangulation
%   $Q$ for which every edge intersects the y-axis and $S$ is on the
%   y-axis.  Furthermore, $Q$ is structured so that, if we take any
%   non-crossing straigh-line drawing of $Q$ we can rei
%
%   \item Thus we have a quadrangulation $Q$ whose edges intersect
%   the y-axis in the order $e_1,\ldots,e_m$ and the vertex set of $Q$
%   contains $S$, still on the y-axis.
%
%   Let $y_1\le\cdots\le y_m$ be any sequence of numbers such that
%   $y_i=y_{i+1}$ if and only if $e_i$ and $e_{i+1}$ have a common endpoint
%   in $S$.  We show that, for any $\epsilon >0$, $Q$ has a planar
%   straight-line drawing such that the intersection of $e_i$ with the
%   y-axis is at $y_i\pm\epsilon$.  Furthermore, if $y_i=y_{i+1}$, then
%   the intersection of $e_i$ and $e_{i+1}$ with the y-axis is exactly
%   at $y_i$. In this way every vertex of $S$ is drawn on the y-axis
%   at precisely the desired location.
%\end{enumerate}




\section{Definitions}

Jordan curve

embedding of a graph = each vertex is a distinct point and each edge is a closed curve whose endpoints are its two vertices.  

we identify vertices of an embedded graph with their points and edges with their curves. By default, an edge curve includes its endpoints, otherwise we specify that it is an \emph{open} edge.

clean embedding - an embedding where the intersection between two edge curves is a finite set.  

rotation system = a graph with an ordering of the edges around each vertex. Every clean embedding defines a rotation system.

straight-line embedding = an embedding in which each edge curve is a line segment

plane embedding = an embedding in which no two edges intersect except possibly at their common endpoint.

face = one of the maximal connected subsets of $\R^2$ that remains after removing the union of all edges

for plane embeddings, we use the convention of listing the vertices of a face in counterclockwise order.

Triangulation = plane embedded graph in which each face is bounded by a 3-cycle.

Quadrangulation = plane embedded graph in which each face is bounded by a 4-cycle. Has $n\ge 4$ vertices and $2n-4$ edges.

outerplanar graph = plane embedded graph in which there is a single face that is incident to every vertex

Drawing = plane embedding 

Straight-line drawing = straight-line plane embedding

cutset = set of vertices whose removal disconnects the graph

separating triangle = cut set of size 3 in a triangulation

edge contraction = contracting $xy$ identifies $x$ and $y$ into a single vertex $v$ and eliminates any resulting parallel edges.  Preserves triangulations provided that $x$ and $y$ is not part of any separating triangle.


We say that a Jordan curve $C:[0,1]\to\R^2$ is \emph{nice} for an
embedding $G$ if $C$ intersects each edge of $G$ in at most one connected
component and the endpoint $C(0)=C(1)$ of $C$ is in the interior of the
outer face of $G$.  We say that $C$ is \emph{clean} (for $G$) if its
intersection with each edge of $G$ is either empty or a single point.
We say that $C$ is \emph{tidy} (for $G$) if it does not contain any
vertex of $G$.

We will make use of the following restatement of
\thmref{collinear-set} which follows from the proof
in \cite{dalozzo.dujmovic.ea:drawing}:
\begin{thm}\thmlabel{dujmovic-frati}
   Let $G$ be a plane embedding and let $C:[0,1]\to\R^2$ be a nice
   Jordan curve for $G$.  Then $G$ has a straight-line drawing $\Gamma$
   in which the vertices of $G$ in $C$ are on the y-axis, the vertices
   of $G$ in $C^-$ are to the left of the y-axis and the vertices of $G$
   in $C^+$ are to the right of y-axis. Furthermore, the sequence of
   vertices and edges encountered in $\Gamma$ while traversing the y-axis
   in the positive direction is identical to the sequence of vertices
   and edges encountered in $G$ while traversing $C$.
\end{thm}




\section{Quadrangulations}
\seclabel{quad}
\seclabel{quadrangulations}

%\begin{lem}
%   Let $Q$ be a drawing of a quadrangulation with $n\ge 5$ vertices
%   and let $Q'$ be a straight-line embedding of $Q$ that has the same
%   rotation system as $Q$.  Then $Q'$ is a straight-line drawing of $Q$.
%\end{lem}
%
%\begin{proof}
%   Some argument about how a crossing in $Q'$ would require some
%   quadrilateral face to have a vertex whose edge ordering is different
%   in $Q$ than in $Q'$.  (Maybe triangulate $Q$.)
%\end{proof}


\subsection{Nice Clean Tidy Curves}

This section is devoted to proving the following result:

\begin{thm}\thmlabel{quad}
    Let
    \begin{compactitem}
    \item $Q$ be a quadrangulation with outer face $f$; 
    \item $C:[0,1]\to\R^2$ be a nice clean tidy Jordan curve for $Q$
     that intersects every edge of $Q$;
    \item $e_1,\ldots,e_m$ be the edges of $Q$ in the
    order they are intersected by $C$; 
    \item $y_1<\cdots<y_m$
    be any increasing sequence of numbers; and
    \item $\Delta$ be a triangle whose intersection with the y-axis
     is the segment $p_1p_m$ with endpoints $p_1=(0,y_1)$ and $p_m=(0,y_m)$.
    \end{compactitem}
    Then $Q$ has a straight-line
    drawing in which, for each $i\in\{1,\ldots,m\}$, the intersection
    of $e_i$ with the y-axis is a single point $(0,y_i)$ and the edges
    $e_1$ and $e_m$ are mapped to the two edges of $\Delta$ that
    intersect the y-axis.
\end{thm}

Refer to \figref{delta}.
Note that the requirement that the edges $e_1$ and $e_m$ map to $\Delta$
fixes the drawing of the outer face $f$ of $Q$. Indeed, the vertex of
$f$ common to $e_1$ and $e_m$ will be mapped to the vertex $\gamma$ of
$\Delta$ that is separated, by $C$, from the other two vertices, $\alpha$
and $\beta$.  The other two endpoints of $e_1$ and $e_m$ will be mapped
to $\beta$ and $\alpha$, respectively.  The two remaining edges of $f$
are then fixed by the requirement that they have endpoints at $\alpha$
and $\beta$ and intersect the y-axis in prescribed locations.
\begin{figure}
   \centering{\includegraphics{figs/delta}}
   \caption{The triangle $\Delta$ fixes the embedding of the outer face of $Q$.}
   \figlabel{delta}
\end{figure}
A


In the remainder of the proof, we will show that the unique embedding of
the outer face $f$ extends to the rest of the $Q$ so that there is exactly
one drawing of $Q$ that satisifies the requirements of \thmref{quad}.

\subsubsection{The Linear System $A\cdot s=b$}

We model this problem by a system of equations that has $m$ variables
$s_1,\ldots,s_m$ in which $s_i$ is the slope of the edge $e_i$, so $e_i$
is supported by the line $\{(x,y):y=s_ix + y_i\}$.  Note that, since
each vertex of $Q$ has degree at least 2, the values of $s_1,\ldots,s_m$
completely determine a straight-line embedding of $Q$.  Of course,
not all choices of $s_1,\ldots,s_m$ determine an embedding of $Q$ and
among those choices that do, not all of them determine a straight-line
drawing that satisfies the conditions of the theorem.

Without loss of generality (by reflection through the y-axis and uniform
scaling of all quantities), assume that $\Delta=\alpha\beta\gamma$
has two vertices $\alpha$ and $\beta$ to the left of the y-axis and
the third vertex $\gamma$ to the right of the y-axis and is contained
in $[-1,1]^2$.  The outer face, $f$, of $Q$ has four edges $e_1$, $e_a$,
$e_b$, and $e_m$, where $1 < a < b < m$.  As discussed above, the slopes
$s_1$, $s_a$, $s_b$, and $s_m$ are completely determined by $y_1$, $y_a$,
$y_b$, $y_c$, and $\Delta$.

For every three edges $e_i$, $e_j$, and $e_k$ incident on the same
vertex $v$, the three supporting lines of $e_i$, $e_j$, and $e_k$
must meet at a common point (the location of $v$) and therefore,
any solution $s=(s_1,\ldots,s_m)$ must satisfy the constraint:
\begin{equation}\eqlabel{slope} 
    s_i = \left(\frac{y_k-y_i}{y_k-y_j}\right) s_j 
          + \left(\frac{y_j-y_i}{y_j-y_k}\right)s_k \enspace .
\end{equation}
Note that, since $y_1,\ldots,y_m$ are given, this is a linear equation
in $s_1,\ldots,s_m$.

A result of Felsner \etal\ \cite[Lemma~2.7]{felsner.huemer.ea:binary}
implies that $Q$ has an orientation of its edges so that each vertex
except those on the outer face has in-degree 2, and each vertex of the
outer face has in-degree 1.  Let $\vec{Q}$ be the digraph obtained
from this orientation. We build a $m\times m$ system of equations
in the following way: For each directed edge $\vec{xy}$ of $\vec{Q}$
corresponding to the edge $e_i=xy$ in $Q$, we add \eqref{slope}, where
$e_j$ and $e_k$ denote the two incoming edges of $x$ in $\vec{Q}$ or
(if $x$ is on $f$) the two edges incident on $x$ that are edges of $f$.

This yields a system of equations $A\cdot s = b$, where $A$ is an $m\times
m$ matrix whose entries come from \eqref{slope}, $s=(s_1,\ldots,s_m)$
are the variables we wish to solve for, and $b$ is a column $m$-vector
whose entries also come from \eqref{slope}.  Some entries of $b$ are
non-zero because the four slopes $s_1$, $s_a$, $s_b$, and $s_m$ are fixed.
We will show that $A\cdot s=b$ has a unique solution and that this solution
gives a plane-straight line drawing of $Q$.

It is clear that any solution $s$ to $A\cdot s=b$ determines a straight-line
embedding of $Q$ that satisfies the conditions of the the theorem,
but it is not clear that it determines a straight-line drawing of $Q$.
In particular, it could give an embedding in which the edges of $Q$
cross each other.  As a first step, we consider solutions to $A\cdot s=b$
that satisfy some ordering constraints.

\subsubsection{Ordering constraints}

Define a relation $\prec$ on $\{1,\ldots,m\}$ where $i \prec j$
if and only if
\begin{enumerate}
  \item $i \prec j$ and $e_i$ and $e_j$ are incident to a common vertex
  $v\in C^-$; or
  \item $i > j$ and $e_i$ and $e_j$ are incident to a common vertex $v\in C^+$.
\end{enumerate}
We say that a vector $s=(s_1,\ldots,s_m)$ satisfies $\prec$ if $s_i <
s_j$ for every pair $i,j\in\{1,\ldots,m\}$ where $i\prec j$.  This notion
captures the condition that vertices inside of $C$ should be drawn to
the left of the y-axis and those outside of $C$ should be drawn to
the right of the y-axis.  It is straightforward to verify that $\prec$ 
is actually a partial order. Indeed, $i_1\prec \cdots
\prec i_r$ implies that, for each $j\in\{3,\ldots,r\}$, $y_{i_j}\in
(\min\{y_{i_{j-1}},y_{i_{j-2}}\}, \min\{y_{i_{j-1}},y_{i_{j-2}}\})$. Thus,
a chain $\prec$ corresponds to a sequence of strictly nested intervals.

\begin{lem}\lemlabel{order-gives-embedding}
   Any solution $s$ to $A\cdot s=b$ that satisfies $\prec$ yields a
   straight-line drawing of $Q$.
\end{lem}

\begin{proof}
It is straightforward to check that, if $s$ satisifies the ordering
constraints then, for any face $g$ of $Q$, the edges of $Q$ in the
embedding defined by $s$ are the edges of a simple quadrilateral whose
orientation matches that of $Q$......
\end{proof}


\subsubsection{Strong Ordering Constraints}

For any $\epsilon \ge 0$, we say that $s=(s_1,\ldots,s_m)$ satisfies the
\emph{$\epsilon$-strong ordering constraints} if, for each $1\le i<j\le
m$ such that $e_i\prec e_j$, $s_i < e_j - \epsilon$.  For small values
of $\epsilon$, the $\epsilon$-strong ordering constraints captures the
condition that vertices should not be drawn too far from the y-axis.

Observe that any solution $s$ that satisfies the $\epsilon$-strong
ordering constraints imply that the $s$ satisifies $\prec$. The following
lemma shows that the converse is also true:

\begin{lem}\lemlabel{weak-to-strong}
   Any solution $s$ to $A\cdot s=b$ that satisifies $\prec$ satisfies the
   $\epsilon$-strong ordering constraints for
   for all $\epsilon<\min\{|y_i-y_j| : 1\le i< j\le m\}$.
\end{lem}

\begin{proof}
   Otherwise, the solution $s$ would yields an embedding of $Q$ in
   which the vertex $v$ incident to $e_i$ and $e_j$ has x-coordinate
   $(y_i-y_j)(s_j-s_i)$ outside the interval $[-1,1]$.
\end{proof}

\subsubsection{Uniqueness of solutions satisfying $\prec$}

\begin{lem}\lemlabel{unique}
   If $s$ is a solution to $A\cdot s=b$ that satisfies $\prec$, then $s$ is 
   the unique solution to $A\cdot s=b$.
\end{lem}

\begin{proof}
   Suppose that there is a solution $s$ to $A\cdot s=b$ that satisifies $\prec$,
   but it is not unique.  Since $A\cdot s=b$ is a linear system, this implies
   that there is an entire (at least) 1-parameter family of solutions,
   i.e., there is a non-trivial $m$-vector $r$ such that, for every
   $\lambda\in(-\infty,\infty)$, $A(s+\lambda r)=b$.

   The vector $r=(r_1,\ldots,r_m)$ has at least four zero entries
   $r_1=r_a=r_b=r_m$ since the slopes $s_1$, $s_a$, $s_b$, and $s_m$
   are fixed by $y$ and $\Delta$.  Since $Q$ is connected, this implies
   that there is at least one vertex $v$ with two incident edges $e_i$
   and $e_j$ such that $r_i=0$ and $r_j\neq 0$.  Set $\lambda^* =
   (s_i'-s_j')/r_j$ and observe that $s^*=s+\lambda^* r$ is a solution
   to $As^*=b$ in which $s_i^*=s_j^*$.  Therefore, this is a solution that
   does not satisify $\prec$ (the edges $e_i$ and $e_j$
   are parallel).  

   Without loss of generality, assume $\lambda^* >0$. Then, by continuity,
   this implies that there is some value of $\lambda\in (0,\lambda^*)$
   such that the solution $s+\lambda r$ satisifies the ordering constraint
   but does not satisfy the $\epsilon^*$-strong ordering constraint.
   This is a contradiction, so we conclude that $s$ must be the unique
   solution to $A\cdot s=b$.
\end{proof}

\subsubsection{A Family of Linear Systems}

   Note that $A$ and $b$ are functions of $y=(y_1,\ldots,y_m)$
   and the 6-coordinates $\Delta=(\delta_1,\ldots,\delta_6)$, of
   the three vertices of $\Delta$.  We make this explict by writing
   $A_1=A(y,\Delta)$ and $b_1=b(y,\Delta)$.  Now \thmref{dujmovic-frati}
   implies that there is some straight-line drawing of $Q$ and some
   $y_1'<\cdots< y_m'$ such that, for each $i\in\{1,\ldots,m\}$, $e_i$
   intersects the y-axis in exactly one point $(0,y_i')$.  Again, without loss
   of generality, we may assume that $\Delta'\subset [-1,1]^2$ and that
   $\Delta'$ has two vertices on the left of the y-axis and one vertex
   on the right.

   Thus far, we have established that there exists $y'=(y_1',\ldots,y_m')$
   and $\Delta'=(\delta_1',\ldots,\delta_6')$ such that the
   system $A(y',\Delta')\cdot s' = b(y',\Delta')$ has at least one
   solution $s'=(s_1',\ldots,s_m')$.  We now define a continuous
   family of linear systems that interpolates between the systems
   $A(y',\Delta')\cdot s=b(y',\Delta')$ and $A(y,\Delta)\cdot s=b(y,\Delta)$.

   For $t\in[0,1]$, define $y(t) = (1-t)y' + ty$.
   Observe that, for any $1\le i< j\le m$ and any $0\le t\le 1$,
   \[
       y(t)_j - y(t)_i = (1-t)(y'_j-y'_i) + t(y_j-y_i) > 0 \enspace .
   \]
   Let $\epsilon^*=\min\{|y(t)_i-y(t)_j| : 1\le i< j\le m,\, 0\le t\le 1\}$
   and observe that $\epsilon^* >0$.

   For each $0< t<1$, let $\Delta(t)=(1-t)\Delta' + t\Delta$ and define
   $A_t=A(y(t),\Delta(t))$ and $b_t=b(y(t),\Delta(t))$.  The entries in
   $A_t$ and $b_t$ are derived from \eqref{slope} and the denominators in
   \eqref{slope} have absolute values bounded from below by $\epsilon^*$.
   Thus, each entry in $A_t$ and $b_t$ is finite and is a uniformly
   continuous function of $t$.  Furthermore, for any $0\le t\le 1$,
   \lemref{weak-to-strong} applies to $A_ts =b_t$, so any solution
   $s$ that satisfies $\prec$ also satisifies the $\epsilon^*$-strong
   ordering constraint.

\subsubsection{Existence of solutions to $A_t\cdot s=b_t$}


\begin{lem}
   For every $0\le t\le 1$, the system $A_t\cdot s=b_t$ has a unique solution
   and this solution satisfies $\prec$.
\end{lem}

\begin{proof}
   Recall that, since $A_t$ is an $m\times m$ matrix, the system $A_t\cdot s=b_t$ 
   has a unique solution $s$ if and
   only if $\det(A)\neq 0$.  Cramer's rules states that, in this case
   the solution $s$ is given by $s(t)=(s_1(t),\ldots,s_m(t))$ where,
   for each $i\in\{1,\ldots,m\}$,
   \[ 
       s_i(t) = \frac{\det(A_t^i)}{\det(A_t)} \enspace ,
   \]
   and $A_t^i$ denotes the matrix $A_t$ with its $i$th column replaced
   by $b_t$. The function $f(t)=\det(A_t)$ is continuous because
   it is a polynomial over the entries in $A_t$ and each of these
   entries are continuous in $t$. By the same reasoning, the function
   $f_i(t)=\det(A_t^i)$ is also continuous for each $i\in\{1,\ldots,m\}$.

   We have already established that $A_0s=b_0$ has a solution $s=s'$
   that satisfies $\prec$ and therefore, by \lemref{unique}, this
   solution is unique, so $\det(A_0)\neq 0$.  To show that $A_t\cdot s=tb_t$
   has a unique solution, there are three possible events we must rule out:
   \begin{enumerate}
     \item for some $0<t\le 1$, $\det(A_t)\neq 0$ but
           the solution to $A_t\cdot s=b_t$ does not satisfy $\prec$;
     \item for some $0<t\le 1$, $\det(A_t)=0$ and $A_t\cdot s=b_t$ has multiple 
           solutions; or
     \item for some $0<t\le 1$, $\det(A_t)=0$ and $A_t\cdot s=b_t$ has no solution.
   \end{enumerate}
   If none of these three events occurs then, forall $0\le t\le
   1$, $\det(A_t)\neq 0$ and the $A_t\cdot s=b_t$ has a unique solution that
   satisfies $\prec$, so the proof is complete.

   Suppose therefore, for the sake of contradiction, that $t^*\le 1$
   is the minimum value $t<1$ for which one of the three events occurs.
   We can rule out the first event by an argument similar to the one which
   shows the uniqueness of $s'$.  That is, by arguing that there exists
   a $t<t^*$ such that the solution to $A_{t}\cdot s=b_{t}$ satisifies $\prec$
   but does not satisfy the $\epsilon^*$-strong ordering constraint.

   The second event is ruled out by arguing (through continuity)
   that the sequence of solutions $s(t)$ (each of which satisify the
   $\epsilon^*$-strong ordering constraint) obtained as $t$ approaches
   $t^*$ converges to a solution $\lim_{t\uparrow t^*} s(t)$ that
   satisfies $\prec$.  (Otherwise we would have a discontinuity where
   $s_i(t)-s_j(t)\ge \epsilon^*$ for $t$ in the neighbourhood of $t^*$,
   but $s_i(t)-s_j(t) \le 0$ at $t=t^*$.   In this case, \lemref{unique}
   implies that $s=s(t)$ is the unique solution that satisifies
   $A_{t^*}\cdot s=b_{t^*}$.  Therefore, the second event cannot occur.

   Thus, it remains only to rule out the possibility that $\det(A_{t*})=0$
   and $A_{t^*}\cdot s=b_{t^*}$ has no solutions.  In this case, consider the
   vector $\lim_{t\uparrow t^*} s(t)$, and define three sets:
   \begin{enumerate}
     \item $P=\{i\in \{1,\ldots,m\}:\lim_{t\uparrow t^*} s_i(t)=\infty\}$;
     \item $N=\{i\in \{1,\ldots,m\}:\lim_{t\uparrow t^*} s_i(t)=-\infty\}$; and
     \item $F=\{1,\ldots,m\}\setminus (P\cup N)$.
   \end{enumerate}

   Thus, the partition $(P,N,F)$ of $\{1,\ldots,m\}$ has he following
   properties:
   \begin{enumerate}
    \item $F$ contains the four edges $e_1$, $e_m$, $e_a$ and $e_b$
      on the outer face.
    \item If $i,j\in F$ and $e_i$ and $e_j$ are incident to a common
      vertex $v$ then $k\in F$ for all edges $e_k$ incident to $v$.
    \item If $s_i \prec s_j \prec s_k$ and $i,k\in F$, 
      then $j\in F$.
   \end{enumerate}
   \lemref{partition}, below, shows that in any such partition, the sets
   $P$ and $N$ are empty.

   By continuity of $f_i(t)$, $F$ consists of the indices $i$ such that
   $\lim_{t\uparrow t^*} s_i(t)$ exists.  Since $N$ and $P$ are empty,
   $s^*=\lim_{t\uparrow t^*} s(t)$ exists and for every $t<t^*$, and
   every $e_i$, $e_j$, and $e_k$ incident to a common vertex:
   \begin{equation}
       d_i(t) := s_i(t) - \left(\frac{y_k(t)-y_i(t)}{y_k(t)-y_j(t)}\right) s_j(t) - \left(\frac{y_j(t)-y_i(t)}{y_j(t)-y_k(t)}\right)s_k(t) = 0 \eqlabel{slope-limit}
   \end{equation}
   Each of the denominators in \eqref{slope-limit} is at least
   $\epsilon^*$ and each of the quantities $y_i(t)$, $y_j(t)$, $y_k(t)$
   is defined and continuous for all $0\le t\le 1$.  This implies
   that $\lim_{t\uparrow t^*} d_i(t)$ exists and is equal to 0, which
   implies that $s^*$ is indeed a solution to $A_{t^*}\cdot s^*=b_{t^*}$.
   This concludes the proof.
\end{proof}

\begin{lem}\lemlabel{partition}
   Let $(P,N,F)$ be a partition of $\{1,\ldots,m\}$ such that 
   \begin{enumerate}
    \item $F$ contains the four edges $e_1$, $e_m$, $e_a$ and $e_b$
      on the outer face;
    \item if $i,j\in F$ and $e_i$ and $e_j$ are incident to a common
      vertex $v$ then $k\in F$ for all edges $e_k$ incident to $v$; and
    \item if $s_i \prec s_j \prec s_k$ and $i,k\in F$, 
      then $j\in F$.
   \end{enumerate}
   Then $C=\{1,\ldots,m\}$ and $P=N=\emptyset$.
\end{lem}

\begin{proof}
    TODO
\end{proof}

This completes the proof of \thmref{quad}.  We remark that this proof
actually shows something stronger; that for any $y_1'<\cdots<y_m'$
and any $y_1<\cdots<y_m$, there is a continuous \emph{morph} \cite{x,y,z,w}
between a drawing $Q'$ in which each edge $e_i$ crosses the y-axis at
$y_i'$ and a drawing $Q$ in which each edge $e_i$ crosses the y-axis
at $y_i$.  At any stage in this morph, all edges cross the y-axis and, for each edge $e_i$, the crossing point between $e_i$ and the y-axis moves linearly from $y_i'$ to $y_i$.

\subsection{Generalized Quadrangulations}

In studying collinear sets, we will obtain graphs that are not quite
quadrangulations but that can be made into quadrangulations that
satisify the requirements of \thmref{quad}. However, we require a slight
strengthening of \thmref{quad} that allows us to contract certain edges
so that the resulting vertex can be placed on the y-axis.

Let $Q$ be a quadrangulation satisfying the preconditions of
\thmref{quad}. Then we say an edge $xy$ is a \emph{split edge} of $Q$
(with respect to $C$) if the minimal subcurve of $C$ that intersects all
edges of $Q$ incident to $x$ or $y$ does not intersect any other edges
of $Q$.
\begin{figure}
   \centering{\includegraphics{figs/split-edge}}
   \caption{A split edge $xy$ in a quadrangulation.}
   \figlabel{split-edge}
\end{figure}
A set $S$ of split edges in $Q$ is \emph{independent} there is no
edge in $Q$ that joins the endpoints of two distinct edges in $S$.
Given an independent set $S$ of split edges, we define the graph $Q_S$
by contracting each edge $xy$ in $S$ and placing the resulting vertex
at the intersection of $C$ and $xy$.

There is a small technicality that occurs when $C$ contains a vertex
of $Q_S$ (because $S$ contains an edge of the outer face of $Q$).
To deal with this, we need some restrictions on the triangle $\Delta$.
Let $r_1,\ldots,r_m$ be a sequence of vertices and edges in a planar graph
and let $y_1<\cdots<y_m$ be a sequence of numbers.  We say that a triangle
$\Delta=\alpha\beta\gamma$ is \emph{compatible} with $r_1,\ldots,r_m$
and $y_1,\ldots,y_m$ if
\begin{compactenum}
  \item $\beta=p_1$ if $r_1$ is a vertex, otherwise $p_1$ is in the interior
  of the edge $\alpha\beta$; and
  \item $\gamma=p_m$ if $r_m$ is a vertex, otherwise $p_m$ is in the interior
  of the edge $\alpha\gamma$.
\end{compactenum}

The following result says that, given $Q$ and $S$, it is possible to
prescribe the locations of the three convex vertices on the outer face
of $Q$ and the intersection of each edge of $Q_S$ with the y-axis.  

\begin{cor}\lemlabel{quad2}
    Let
    \begin{compactitem}
    \item $Q$ be a quadrangulation with outer face $f$; 
    \item $C:[0,1]\to\R^2$ be a Jordan curve whose endpoint $C(0)=C(1)$ is
     in the interior of $f$, whose intersection with each open edge of
     $Q$ consists of exactly one point, and that does not contain any
     vertex of $Q$;
    \item $S$ be an independent set of split edges of $Q$ (with respect to $C$);
    \item $r_1,\ldots,r_m$ be the edges and vertices of $Q_S$ in the
    order they are intersected by $C$; 
    \item $a_1<\cdots< a_m$ be any increasing sequence of numbers; and
    \item $\Delta$ be a triangle compatible with $r_1,\ldots,r_m$ and $a_1,\ldots,a_m$.
    \end{compactitem}
    Then $Q$ has a straight-line drawing in which, for each
    $i\in\{1,\ldots,m\}$, the intersection of $r_i$ with the y-axis is
    a single point $(0,a_i)$ and three vertices of the outer face $f$
    are mapped to the three vertices of $\Delta$.
\end{cor}


\begin{proof}
  The proof is an adaption of the ideas used to prove \thmref{quad}
  and we only describe the modifications here.  Let $m'=|E(Q)|-|S|$
  be the number of edges in $Q_S$ and let $e_1,\ldots,e_{m'}$ be the
  edges of $Q_S$.  Each edge $e_i$ of $Q_S$ has a coordinate $y_i$
  at which it should intersect the y-axis:  Either $e_i=r_j$ for some
  $j$ or $e_i$ is incident on a vertex $r_j\in C$. In either case,
  we define $y_i=a_j$.  Note that this implies that $y_1\le\cdots\le
  y_{m'}$, so $y_1,\ldots,y_{m'}$ is non-decreasing, but not necessarily
  stricly increasing.

  We define the system $A\cdot s=b$ as before except that, if $e_i$
  is incident on some vertex $x$ and $x$ is incident on an edge $xy\in
  S$, then \eqref{slope} is invalid. This is because, this equation
  applies when $e_j$ and $e_k$ are also incident on $x$ so, in this case
  $y_k=y_j$, and the denominators in \eqref{slope} are 0.
  Therefore, for each split edge $xy\in S$ and each outgoing edge
  $e_i\neq xy$ of $x$ or $y$, the system $A\cdot s =b$ is missing an
  equation that determines $s_i$.

  Suppose, without loss of generality, that $xy\in S$ has $x\in C^-$ and $y\in
  C^+$ that, in the oriented graph $\vec{Q}$ is oriented as $\vec{xy}$.
  In $\vec{Q}$, $x$ has one two incoming edges coming from neighbours in
  $C^+$ and $y$ has two incoming edges coming from neighbours in $C^-$,
  and one of these is $x$.  When we contract edges in $S$ to obtain
  $Q_S$ we also obtain an oriented graph $\vec{Q}_S$ whose edge
  orientations are inherited form $\vec{Q}$.  The edge $xy$ becomes
  a vertex $v$ in $\vec{Q}_S$ that has two incoming edges $e_j$ and $e_k$
  from $C^+$ and one incoming edge $e_\ell$ from from $C^-$.

  The vertex $v$ has $\deg(x)+\deg(y)-5$ outgoing edges and we would like
  to introduce an equation in our linear system for each of these edges.
  By \thmref{dujmovic-frati}, we know that there is some straight-line
  drawing $Q'$ of $Q$ and some $y_1'<\cdots< y_m'$ such that, for each
  $i\in\{1,\ldots,m'\}$, $e_i$ intersects the y-axis in exactly one
  point $(0,y_i)$.  In this drawing, each edge $e_i$ has some slope which
  we denote by $s_i'$.  Without loss of generality, assume that $s_j'
  > s_k'$, so that $e_j,e_k,e_\ell$ is the counterclockwise order of
  these three edges around $v$.

  Each outgoing edge $e_i=\vec{vw}$ of $v$ is of one of five types,
  which we presented in counterclockwise order around $v$ (see
  \figref{five-types}):
   \begin{figure}[htbp]
      \centering{\includegraphics{figs/five-types}}
      \caption{The five types of outgoing edges from $v$.}
      \figlabel{five-types}
   \end{figure}
  \begin{enumerate}
    \item $w\in C^-$ and $s'_i < s'_\ell$.  In this case, we add the
       equation $s_i = s_\ell - (s'_\ell-s'_i)$.

    \item $w\in C^-$ and $s'_i > s'_\ell$.  In this case, we also add
       the equation $s_i = s_\ell + (s'_i-s'_\ell)$.

    \item $w\in C^+$ and $s'_i < s'_j$. In this case, we add the equation
       $s_i = s_j - (s'_j - s'_i)$.

    \item $w\in C^+$ and $s'_j < s'_i < s'_k$. In this case, we add the equation
       $s_i = \alpha_{ijk} s_j + (1-\alpha_{ijk})s_k$, where $\alpha_{ijk}=(s_i-s_k)/(s_j-s_k)$.

    \item $w\in C^+$ and $s'_i > s'_k$. In this case, we add the equation
       $s_i = s_k + (s'_i - s'_k)$.
  \end{enumerate}
  In this way, we obtain a system $A\cdot s =b$ with $m'$ variables
  $s_1,\ldots,s_{m'}$ and $m'$ equations---one equation for each directed
  edge of $\vec{Q}_S$.

  For all $0\le t\le 1$, we define $A_t$ and $b_t$ as before and notice
  that $s'=(s_1',\ldots,s_{m'})$ satisfies $A_0 s' = b_0$.  The partial
  order $\prec$ is defined in the same way as before, with respect to $Q$
  and $C$, and this partial order behaves as we would expect for edges
  of $Q_S$.  In particular, for a vertex $v\in C$, $prec$ correctly
  orders the edges form $v$ to vertices in $C^+$ by increasing slope
  and edges from $v$ to vertices in $C^-$ by increasing slope.

  We claim that, as before there exists an $\epsilon^*>0$ such that, for
  any $0\le t\le 1$, any solution $s=(s_1,\ldots,s_{m'}$ to $A_t\cdot
  s=b_t$ that satisfies $\prec$ also satisifes the $\epsilon^*$-strong
  ordering constraints.  This is straightforward to verify, except in
  the case of Type~4 edges discussed above.  In order to establish
  that two Type~4 edges incident on a common vertex $v$ satsify the
  $\epsilon^*$-strong ordering constraint, it is sufficient to show
  (using the notation above) that $s_k-s_j\ge \delta$ for some $\delta>0$
  that does not depend on $t$.


  As before, we can define the partial order $\prec$ in the obvious way
  and 


\end{proof}

%\footnote{We abuse the term \emph{drawing} here slightly, since we sometimes require that the two endpoints of a clean edge are drawn at the same location.} in which, for each $i\in\{1,\ldots,m\}$, the intersection
%    of $e_i$ with the y-axis is a single point $(0,y_i)$ and the edges
%    $e_1$ and $e_m$ are mapped to the two edges of $\Delta$ that
%    intersect the y-axis.
%\end{lem}
%
%
%\begin{proof}
%   Add artificial constraints on the slopes of edges incident to split edges (and to split edges themselves.
%\end{proof}
%
%

%
%
%
%
%We say that a 
%Next, we present a strengthening of \lemref{quad} that allows vertices
%of $Q$ to be on the curve $C$.  Let $r_1,\ldots,r_m$ be a sequence
%of vertices and edges in a planar graph and let $y_1<\cdots<y_m$ be a
%sequence of numbers.  We say that a triangle $\Delta=\alpha\beta\gamma$
%is \emph{compatible} with $r_1,\ldots,r_m$ and $y_1,\ldots,y_m$ if
%\begin{compactenum}
%  \item $\beta=p_1$ if $r_1$ is a vertex, otherwise $p_1$ in the interior
%  of the edge $\alpha\beta$; and
%  \item $\gamma=p_m$ if $r_m$ is a vertex, otherwise $p_m$ in the interior
%  of the edge $\alpha\gamma$.
%\end{compactenum}
%
%\begin{lem}\lemlabel{quad2}
%    Let
%    \begin{compactitem}
%    \item $Q$ be a quadrangulation with outer face $f$; 
%    \item $C:[0,1]\to\R^2$ be a Jordan curve
%     whose endpoint $C(0)=C(1)$ is in the interior of $f$,
%     whose intersection with each edge of $Q$
%     consists of exactly one point, and for which no vertex of $Q$ on $C$
%     has neighbours both inside and outside of $C$;
%    \item $r_1,\ldots,r_m$ be the edges and vertices of $Q$ in the
%    order they are intersected by $C$; 
%    \item $y_1<\cdots<y_m$
%    be any increasing sequence of numbers; and
%    \item $\Delta$ be a triangle that is compatible with $r_1,\ldots,r_m$ and $y_1,\ldots,y_m$.
%    \end{compactitem}
%    Then $Q$ has a straight-line
%    drawing in which, for each $i\in\{1,\ldots,m\}$, the intersection
%    of $r_i$ with the y-axis is a single point $(0,y_i)$ and three vertices
%    of $f$ are mapped to the vertices of $\Delta$
%\end{lem}
%
%\begin{proof}
%  Add artifical constraints for vertices on $C$.  If, for some reason, that approach fails, then use the argument we worked out for shifting points off of $C$ and then moving them back to the y-axis.
%%   To fix this, first note that any vertex $r_i$ on $C$ is only incident
%   only to crossing edges and therefore the neighbours of $r_i$ are all
%   contained in $R$. Thus, we can deform $C$ in the neighbourhood of $r_i$
%   so that $r_i$ moves into the interior of $L$ and $C$ intersects each
%   of $r_i$'s incident edges $e_1,\ldots,e_d$ in exactly one point.  Now,
%   to apply \lemref{quad} we must specify numbers $y_1',\ldots,y_d'$ where
%   $(0,y_i')$ is the intersection point between $e_j$ and the y-axis.
%   To do this we choose any $y_1'<\cdots<y_d'$ satisfying
%   \[
%       y_{i-1} < y_1' < y_i < y_d' < y_{i+1} \enspace .
%   \]
%   By doing this for each vertex $r_i$ in $C$ we obtain a curve $C'$ and
%   a sequence $y_1''<\cdots<y_m''$ on which we can apply \lemref{quad} to
%   find a drawing of $Q$.  Now this drawing of $Q$ does not yet satisfy
%   the requirements of the theorem because there are vertices $r_i\in
%   C$ that are not in $C'$. However, the choice of $y_1'<\cdots<y_d'$
%   described above ensures that moving the vertex $r_i$ to $(0,y_i)$
%   does not introduce any crossings and gives a drawing of $Q$ that
%   satisfies all the relevant requirements of the theorem.  Finally,
%   in the process of building $Q$ described above, we showed how the
%   drawing of $Q$ can be extended to a drawing of $T$ that satisifies
%   all the requirements of the theorem.
%\end{proof}



%\begin{lem}
%   Let $Q$ be a straight-line drawing of a quadrangulation each of whose edges intersect the
%   y-axis in exactly one point.  Let $v$ be a vertex of $Q$ that is not on
%   the y-axis, whose incident edges $vx_1,\ldots,vx_d$ intersect
%   the y-axis at y-coordinates $y_1'<\cdots<y_d'$, respectively, 
%   and suppose that no
%   other edges of $Q$ intersect the y-axis with y-coordinates in the interval 
%   $[y_1',y_d']$.
%   Then there is a unique index $i\in\{1,\ldots,d\}$ such that
%   \begin{enumerate}
%     \item for every $j\in\{1,\ldots,i-1\}$, all edges incident to $x_j$, aside from $vx_j$ intersect the y-axis below $(0,y_1')$;
%     \item for every $j\in\{i+1,\ldots,d\}$, all edges incident to $x_j$, aside from $vx_j$ intersect the y-axis above $(0,y_d')$; and
%     \item the embedding of $Q$ obtained by moving $v$ to $(0,y_i')$ is a straight-line drawing.
%   \end{enumerate}
%\end{lem}
%
%\begin{proof}
%   For the first two points, suppose on the contrary that there is some
%   pair of indices $k < \ell$ such that $x_k$ is incident to an edge
%   $wx_k$ that crosses the y-axis at some point $p$ above $(0,y_d')$
%   and $x_\ell$ is incident to an edge $ux_\ell$ that crosses the y-axis
%   at some point $q$ below $(0,y_1')$. Consider the triangle $abc$ with
%   $a=(0,y_k')$, $b=x_k$, and $c=p$.  Then the path $v,x_\ell,u$ enters
%   the interior of $abc$ through the segment $ac$ and exits through one of
%   the other two segments.  But this is a contradiction to the assumption
%   that $Q$ is a straight-line drawing, since it implies that one of the
%   edges in this path crosses at least one of the edges $vx_k$ or $wx_k$.
%
%   For the third point, observe 
%\end{proof}

\section{Triangulations}
\seclabel{triangulations}

We will sometimes make use of this simple fact:
\begin{obs}\obslabel{quad}
  If $q=abcd$ is a simple quadrilateral, then neither of the segments $ac$
  or $bd$ cross any of the edges of $q$.
\end{obs}

\begin{thm}\thmlabel{main}
   Let
   \begin{compactenum}
     \item  $T$ be a triangulation with outer face $f$;
     \item  $C:[0,1]\to\R^2$ be a Jordan curve whose endpoint $C(0)=C(1)$
            is in the interior of $f$ and whose intersection with each
            edge of $Q$ consists of at most one point;
     \item $r_1,\ldots,r_k$ be the sequence of vertices and open edges
           of $T$ that are intersected by $C$, and ordered in the order
           that they are intersected by $C$;
     \item $y_1<\cdots<y_k$ be any sequence of numbers; and
     \item $\Delta$ be a triangle that is compatible with 
           $r_1,\ldots,r_m$ and $y_1,\ldots,y_m$.
  \end{compactenum}
   Then, for any $\epsilon>0$, $T$ has a
   straight-line drawing in which the outer face $f$ is $\Delta$
   and, for each $i\in\{1,\ldots,k\}$, 
   \begin{compactenum}
       \item $r_i$ is drawn on the y-axis, with y-coordinate $y_i$
         if $r_i$ is a vertex; or
       \item (if $r_i$ is an edge) the intersection of $r_i$ with the
         y-axis has a y-coordinate in the interval
         $[y_i-\epsilon,y_i+\epsilon]$.
   \end{compactenum}
\end{thm}

\begin{proof}
   We call $y_i$ the (desired) \emph{crossing coordinate} for $r_i$. If
   a straight-line drawing contains an edge whose intersion with the
   y-axis is $\{(0,y)\}$ or a vertex at $(0,y)$, we say that the edge
   or vertex \emph{crosses} (the y-axis) at $y$.

   Let $L$ and $R$ be the bounded and unbounded components, respectively,
   of $\R^2\setminus C$. Let $\bar{L}=C\cup L$ and $\bar{R}=C\cup R$
   denote the closures of $L$ and $R$.  We say that an edge of $T$ is
   a \emph{crossing edge} if its intersection with $C$ is non-empty.
   A crossing edge is a \emph{proper crossing edge} if its intersection
   with each of $L$ and $R$ is non-empty.  We say that points in $L$
   are \emph{to the left of $C$} and points in $R$ are \emph{to the
   right of $C$}.

%   We prove an extension of the theorem to the case where $T$ is an
%   non-crossing embedded graph whose faces consist of triangles (3-cycles)
%   and quadrilateral (4-cycles) with the resriction that, for every
%   quadrilateral face $q$, all four edges of $q$ are crossing edges.
%   The proof is by induction on the number of non-crossing edges plus the
%   number of vertices of $T$.

%   \paragraph{Base Cases:}
%   There are three base cases tht we handle explicitly.  If $T$ contains
%   2 or fewer crossing edges, If $T$ is the complete graph, $K_4$ on 4
%   vertices, but only has only three crossing edges, then the theorem is
%   also easy to prove directly.  The last base case occurs when all edges
%   of $T$ are crossing edges.  In this case $T$ is bipartite and therefore
%   all its faces are quadrilaterals, so $T$ is a quadrilateralization.
%   This case is handled directly by \lemref{quad2}.
%
%   Thus we may assume that $T$ has at least one non-crossing edge and
%   at least 2 crossing edges.  

   The proof is by induction on $n+m$, where $n$ is the number of
   vertices of $T$ and $m$ is the number of crossing edges.  We begin
   by describing reductions that allow use to apply the inductive
   hypothesis. When none of these reductions are possible, we arrive
   at our base case. To handle this base case we argue that $T$ has a
   sufficiently simple structure that it can be handled by the algorithm
   for obtaining drawings of quadrangulations.

   \paragraph{Separating Triangles.}
   (See \figref{separating}.)
   A \emph{separating triangle} $xyz$ in $T$ is a cycle of length three
   whose removal disconnected $T$.  If $T$ contains a separating triangle
   $xyz$ then we remove all vertices from the interior of $xyz$ to obtain
   a graph $T^+$ in which $xyz$ is a face.  Since the intersection of $C$
   with each of $xy$, $yz$ and $zx$ consists of at most a single point,
   the vertices and edges of $T$ intersected by $C$ that are not in $T^+$
   appear as a contiguous subsequence $r_i,\ldots,r_j$.

   \begin{figure}
      \centering{\includegraphics{figs/separating}}
      \caption{Recursing on separating triangles in the proof of
      \thmref{main}}
      \figlabel{separating}
   \end{figure}

   Observe that each of $r_{i-1}$ and $r_{j+1}$ is either an edge
   or vertex of the triangle $xyz$.  Set $\epsilon'$ to be any
   value less than $\min\{\epsilon,y_{i}-y_{i-1}, y_{j+1}-y_j\}$.
   and apply induction on $T^+$ using the value $\epsilon'$
   and the sequences $r_1,\ldots,r_{i-1},r_{j+1},\ldots,r_k$ and
   $y_1,\ldots,y_{i-1},y_{j+1},\ldots,y_k$ to obtain a drawing of $T^+$.
   In the resulting drawing $xyz$ becomes a triangular face $\Delta'$.

   In the resulting drawing, Let $y_{i-1}'$ and $y_{j+1}'$
   be the respective y-coordinates of the intersections of
   $r_{i-1}$ and $r_{j+1}$ with the y-axis.  By our choice of
   $\epsilon'$, $y_{i-1}'<y_i<\cdots<y_j<y_{j+1}'$.  Observe that
   $\Delta'$ is compatible with $r_{i-1},\ldots,r_{j+1}$ and
   $y_{i-1}',y_i,\ldots,y_j,y_{j+1}'$.

   Let $T^-$ be the graph obtained by removing, from $T$, all
   vertices outside of $xyz$.  Now we apply induction on $T^-$ using
   the triangle $\Delta'$ and the sequences $r_{i-1},\ldots,r_{j+1}$ and
   $y_{i-1}',y_i,\ldots,y_{j},y_{j+1}'$.  Combining the drawings of $T^+$
   and $T^-$ yields a drawing of $T$ that satisfies the requirements of
   the theorem.  Thus, we may assume that $T$ has no separating triangles.

   \paragraph{Contractible Edges:}
   (See \figref{contractible}.)
   We say that a triangular face of $T$ is a \emph{proper crossing
   face} if it is incident to two proper crossing edges.  We say that a
   non-crossing edge of $T$ is \emph{contractible} it is not contained
   in the boundary of any crossing face.  
   \begin{figure}
      \centering{\includegraphics{figs/contractible}}
      \caption{Contracting and uncontracting edges in the proof of
      \thmref{main}}
      \figlabel{contractible}
   \end{figure}

   If $T$ contains a contractible edge $xy$ then we contract $xy$ to
   obtain a new vertex $u$ in a smaller graph $T'$.   We can then apply
   induction on $T'$ with the value $\epsilon'=\epsilon/2$ to obtain a
   drawing of $T'$ that satisfies all the conditions of the theorem under
   the stronger condition that each proper crossing edge $e_i$ crosses
   the y-axis in the interval $[y_i-\epsilon/2,y_i+\epsilon/2]$.

   To obtain a drawing of $T$ we uncontract $v$ by placing $x$ and $y$
   within a ball of radius $\epsilon/2$ centered at $v$. (That such
   a placement is always possible is a standard argument.)  Since the
   distance between $y$ and $v$ and $x$ and $v$ is at most $\epsilon/2$,
   each proper crossing edge $r_i$ incident on $x$ or $y$ will cross
   the y-axis in the interval $[y_i-\epsilon,y_i+\epsilon]$.

   Thus we may assume that $T$ has no separating triangles of contractible
   edges.

%   \paragraph{Eraseable edges}
%   We say that a non-crossing edge of $xy$ of $T$ is \emph{eraseable}
%   if neither of its endpoints is on $C$ and both its incident faces
%   intersect $C$.  If $T$ contains an eraseable edge $xy$, then we remove
%   the edge $xy$ from $T$ to obtain smaller graph $T'$ on which we can
%   apply induction. In the resulting drawing of $T'$, $x$ and $y$ lie on
%   a common face (which may be the outer face of $T'$) and are visible.
%   We can therefore add the edge $xy$ to obtain the desired drawing
%   of $T$.

   \paragraph{Flippable edges.}
   (See \figref{flippable}.)
   We say that a non-crossing edge $xy$ of $T$ is flippable if there
   exists distinct vertices $z$, $a$, $b$, and $c$, such that 
   \begin{compactenum}
      \item $xyb$, $zyc$, $xza$ are crossing faces of $T$;
      \item $xyz$ is a non-crossing face of $T$; and (
      \item $C$ intersects $za$, $xa$, $xb$, $yb$, and $yc$ in this order; or 
      \item or $C$ intersects $xa$, $xb$, $yc$, $zc$, $za$, in this order).  
   \end{compactenum}
   \begin{figure}
      \centering{\includegraphics{figs/flippable}}
      \caption{Flipping edges in the proof of
      \thmref{main}}
      \figlabel{flippable}
   \end{figure}

   If $T$ contains the flippable edge $xy$ then we remove $xy$ and
   replace it with $zc$ to obtain a new graph $T'$.
   Note that, since $T$ has no separating triangles, the edge $zc$
   is not already present in $T$.
   After choosing a crossing coordinate for $zc$ somewhere
   between those of $xc$ and $yc$ we can then inductively draw $T'$.  

   We claim that in the resulting drawing of $T'$, the only open edge
   that intersects the open segment $xy$ is $zc$.  In particular, we must
   ensure that $z$ is not a reflex vertex in the quadrilateral $xcyz$.
   To show this we distinguish between the two possible cases (3 and 4)
   in the definition of flippable edges. In Case~3, The existence of the
   edges $za$ and $zb$ ensure that, in the resulting drawing of $T'$,
   $xcyz$ is convex.  In Case~4, the triangle $zxa$ is convex and $xcyz$
   is contained in this triangle, therefore $z$ is a convex vertex
   of $xcyz$.

   In either case, removing $zc$ from the drawing of $T'$ and replacing
   it with $xy$ yields the desired drawing of $T$.

   \paragraph{The Base Case.}

   Finally, we are left with a situation in which $T$ is a triangulation
   with no separating triangles, no contractible edges, and no flippable
   edges.  
   If $T$ is the complete graph on $K_3$ or $K_4$ on three or four
   vertices, then the theorem is trivial, so we may assume that $T$
   has at least 5 vertices.

   We claim that every non-crossing edge $xy$ of $T$ is contained in the
   boundary of two crossing faces $xya$ and $yxb$.  To see why this is so,
   observe that if some non-crossing edge $xy$ is not contractible then
   one of $xy$'s incident faces, $yxc$ is proper crossing.  Suppose, for
   the sake of contradiction, that the other face $xyz$, incident on $xy$
   is not crossing.  Since neither $zx$ nor $yz$ is contractible, they
   must be incident on crossing triangles $xza$ and $zyb$, respectively.
   Since $T$ contains no separating triangles, we know that $b\neq a$,
   otherwise $xya$ would separate $z$ from $c$.

   This leaves us in the situation in which we have distinct vertices $x$,
   $y$, $z$, $a$, $b$, $c$, such that $xyb$, $zyc$, $xza$ are crossing
   faces of $T$ and $xyz$ is a non-crossing face of $T$.  Checking the
   definition of flippable edge then ensures that at least one of $xy$,
   $yz$, or $za$ is a flippable edge.

   Thus, ever non-crossing edge of $T$ is incident to two crossing
   triangles.  The union of these two triangles is a quadrilateral
   consisting of four crossing edges.  Let $Q^*$ denote the graph obtained
   by removing all non-crossing edges from $T$.  The faces of $Q^*$
   are quadrilaterals, each having four crossing edges or triangles each
   having three crossing edges.

   The only triangles with three crossing edges are those triangles
   having one vertex in each of $C$, $L$ and $R$.  Now, consider any
   vertex $v=r_i$ on $C$.  The graph $Q^*$ has two triangular faces $vab$
   and $vcd$ incident on $v$ such that $ab=r_{i-1}$ and $cd=r_{i+1}$.
   Split $v$ into two vertices $x\in L$ and $y\in R$ joined by the edge
   $xy$, make $x$ adjacent to all neighbours of $v$ in $R$, and make
   $y$ adjacent to all neighbours of $v$ in $L$. See \figref{split}.
   This splitting operation eliminates the triangular faces $vab$ and
   $vcd$ and introduces the quadrangular faces $xyab$ and $yxcd$.

   \begin{figure}
      \centering{
        \begin{tabular}{c|c}
            \includegraphics{figs/split} & \includegraphics{figs/split-outer}
        \end{tabular}}
      \caption{Splitting vertices on $C$ in the proof of
      \thmref{main}.}
      \figlabel{split}
   \end{figure}

   The fact that $C$ intersects each edge of $Q$ in at most one point
   implies that no two vertices on $C$ are adjacent.
   Therefore, this splitting
   operation can be done on every vertex that is on $C$ to obtain
   a quadrangulation $Q$ in which every edge properly crosses $C$,
   as well as an independent set $S$ of split edges in $Q$.  \lemref{quad2}
   then provides a drawing of $Q$ that satisfies all the conditions of
   the theorem and reinserting the non-crossing edges in the resulting
   drawing provides the desired drawing of $T$.
%
%
%   Either these triangles form a separating triangle that separates $z$
%   from $c$ or the existence of the edges $za$ and $zb$ establishes that
%   $xy$ is flippable.
%
%
%   We claim that $T$ has no non-crossing edges except those in $L$
%   that have one endpoint on $C$.  To see why this is so, observe that
%   if some non-crossing edge $xy$ is not contractible then one of $x$
%   or $y$ is on $C$ or one of $xy$'s incident faces, $yxc$ is crossing.
%   Since $xy$ is no erasable, it must be incident on a non-crossing
%   face $xyz$.  Since neither $zx$ nor $yz$ is contractible, they must
%   be incident on crossing triangles $xza$ and $xzb$, respectively.
%   Either these triangles form a separating triangle that separates $z$
%   from $c$ or the existence of the edges $za$ adn $zb$ establishes that
%   $xy$ is flippable.
%
%   Any cycle in $T$ that uses only crossing edges necessarily has even
%   length.  Therefore, the only triangular faces of $T$ include at least
%   one vertex on $C$.  But this implies that every triangular face $T$
%   includes one vertex from each of $C$, $L$, and $R$, otherwise $C$
%   includes a non-crossing edge neither of whose vertices is on $C$.
%
%   TODO: Mention conditions on input having exactly two triangular faces
%   for each vertex on $C$.
%   None of the operations used to arrive at this base case remove
%   crossing edges or edges incident to $C$.  Thus, every vertex on $C$
%   is incident to exactly two triangles in $T$ and these are the only
%   triangles in $T$.  For every vertex $v=r_i$ on $C$, we split $v$
%   into two vertices $x$ and $y$ joined by an edge $xy$ and make $x$
%   is adjacent to $v$'s neighbours in $L$ and $y$ adjacent to $v$'s
%   neighbours in $R$.  We embed $x$ and $y$ in the neighbourhood of $v$
%   in such a way that $C$ crosses every edge incident to $x$ and $y$.
%   We specify crossing coordinates for all of these crossings that have
%   a value very close to that of $y_i$.
%
%   Observe that, after we do this for each every vertex on $C$, the
%   resulting graph is a quadrangulation.  Now apply the theorem above....
%
%
%   \paragraph{Reductions}
%   Next we consider what can be done when $T$ has no separating
%   triangles and no contractible edges.  Note that this implies that
%   every vertex $v$ of $T$ is incident to at least one crossing edge
%   since, otherwise every edge incident to $v$ is contractible.
%  
%   Consider the graph $H$ consisting of only the non-crossing edges of $T$
%   and their endpoints.  Some face $g$ of $H$ contains $C$.  We claim
%   that $H$ is outerplanar because every vertex of $H$ is incident to
%   at least one crossing edge and is therefore on the boundary of $g$.
%
%   Next, we claim that $H$ has only three kinds of 2-connected components:
%   (i)~single edges, (ii)~3-cycles, and (iii)~two three cycles sharing an edge.
%
%   To see why this is so, suppose that some 2-connected component $K$ of $H$ has $k\ge 5$ vertices.  Since $T$ is triangulated, $K$ is a triangulated $k$-gon.  But $ and by repeatedly removing vertices of degree 
%
%
% this component therefore has $  Then
%   To see why this is so, observe that every edge $xy$ of $H$ is not contractible so either:
%   \begin{enumerate}
%       \item $xy$ bounds a non-proper
%
%   \end{enumerate}
%
%
%
%
%   Furthermore, every edge $e$ of $H$ is on the boundary of $g$ since,
%   otherwise, neither of the two triangles incident on $e$ would be proper
%   crossing triangles and $e$ would be contractible.  This means that
%   $H$ is a \emph{cactus graph}---a graph in which each edge is incident
%   to at most one cycle.  Furthermore, each cycle of $H$ is a 3-cycle.
%   To see why, observe that, since $T$ is a triangulation, if $H$
%   contained a $k$-cycle for $k\ge 4$, this cycle would have a chord.
%
%   Because $T$ has at leat one non-crossing edge, $H$ contains at least one
%   edge.  Therefore, $H$ has at least one 2-connected component, 
%   which is either a 3-cycle
%   or a single edge and assume, without loss of generality that this
%   2-connected component is contained in $\bar L$.  
%   There are two cases to consider:
%   \begin{enumerate}
%	\item The 2-connected component is a single edge $xy$.	Let $xya$
%	and $yxb$ be the two faces of $T$ incident to $xy$.  Note that
%	$a$ and $b$ are both in $R$ since, otherwise, the 2-connected
%	component containing $xy$ would be a 3-cycle.
%	Therefore, removing the edge $xy$ from $T$ creates a
%	quadrangular face $xbya$ consisting of only crossing
%	edges. \obsref{quad} ensures that, after applying induction
%	on this smaller graph, the edge $xy$ can be added without
%	introducing crossings.  (It is worth noting that the analysis
%	of this case remains true even if $xy$ is an edge of the
%	outer face.)
%        \begin{figure}
%           \centering{\includegraphics{figs/1b}}
%           \caption{When a 2-connected component of $H$ is an edge. The second figure illustrates the situation when $xy$ is an edge of the outer face.}
%            \figlabel{1b}
%        \end{figure}
%        \item The component is a face $xyz$ of $T$.  Since none of $xy$,
%        $yz$, or $zx$ was contracted, there exists vertices $a,b,c\in R$
%        (possibly not distinct) such that $xya$, $yzb$ and $zxc$ are
%        triangular faces of $T$.\footnote{Note that this is even true if
%        $xy$ was not contracted because $x,y\in C$. In that case there
%        is still an $a\in R$.} 
%
%        We claim that $a$, $b$, and $c$ are distinct vertices. If not,
%        this means that $xyz$ all have a common neighbour, say $a$. Thus
%        $T$ contains the complete graph $K_4$ as a subgraph.  The case in
%        which $T$ is $K_4$ was handled as a base case in our induction,
%        therefore $T$ contains some vertex $v\not\in\{x,y,z,a\}$. But
%        contradicts the assumption that $T$ contains no separating
%        triangles, since $v$ must be in one of the four triangular faces
%        of this $K_4$ subgraph, say $xyz$, but then $xyz$ is a separating
%        triangle that separates $v$ from $a$.
%
%        Thus, $a$, $b$, and $c$ are three distinct vertices of $T$
%        (see \figref{2c}). Note that, in this case, $T$ does not
%        contain the edge $yc$ since, if it did, then $yxc$ would be
%        a separating triangle that separates $a$ from $b$.
%        \begin{figure}
%           \centering{\includegraphics{figs/2b}}
%           \caption{When a 2-connected component of $H$ is a triangle $xyz$ whose three vertices have two common neighbours $a,c\in R$.}
%           \figlabel{2c}
%        \end{figure}
% 
%        There are two cases to consider: 
%
%        If none of $xy$, $yz$, or $zx$ is an edge of the outer face,
%        then (after appropriate relabelling) $C$ intersects the edges
%        $yb$, $zb$, $zc$, $xc$, $xa$, $ya$ in this order.  In this case,
%        we remove the edges $xy$, $yz$ and $zx$ and add the edge $yc$.
%        At the same time we specify a crossing location of the edge
%        $yc$ on the y-axis anywhere in the interior of the segment
%        whose ends points are the crossing locations of $xc$ and $zc$.
%        This creates two quadrangular faces $ycxa$ and $ybzc$.  After
%        inductively drawing this smaller graph, the vertices $x$ and $z$
%        are reflex vertices of these two faces.  Since every non-convex
%        quadrilateral is star-shaped and its kernel contains its reflex
%        vertex, we can remove the edge $yc$ and add the edges $xy$, $yz$
%        adn $zx$ without introducing crossings.
%
%        If $xy$ is on the outer face, then $C$ intersects the edges
%        $ya$, $yb$, $zb$, $zc$, $xc$, $xa$ in this order. In this
%        case we perform exactly the same operation:  we remove the
%        edges $xy$, $yz$ and $zx$ and add the edge $yc$ and specify a
%        crossing location of the edge $yc$ on the y-axis anywhere in
%        the interior of the segment whose ends points are the crossing
%        locations of $xc$ and $zc$.  Again, this creates two quadrangular
%        faces $ycxa$ and $ybzc$ only now $ycxa$ is the outer face.
%        After inductively drawing this smaller graph, the vertices $z$
%        is a reflex vertices in $ybzc$ and the vertex $x$ sees the entire
%        edge $cy$.  This ensure that we can remove the edge $yc$ and
%        add the edges $xy$, $yz$ and $zx$ without introducing crossings.
%   \end{enumerate}
%   This completes the proof.
\end{proof}

\begin{cor}
  Every collinear set is free.
\end{cor}

\begin{proof}
   Subdivide edges contained in $C$ and then apply \thmref{main}.
\end{proof}


\bibliographystyle{plainurl}
\bibliography{freecoll}


\end{document}


