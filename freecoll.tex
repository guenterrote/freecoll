\documentclass{patmorin}
\usepackage[utf8]{inputenc}
\usepackage{amsthm,amsmath,graphicx}
\usepackage{array}
\usepackage{pat}
\usepackage{hyperref}
\usepackage[dvipsnames]{xcolor}
\definecolor{linkblue}{named}{Blue}
\hypersetup{colorlinks=true, linkcolor=linkblue,  anchorcolor=linkblue,
citecolor=linkblue, filecolor=linkblue, menucolor=linkblue,
urlcolor=linkblue, pdfcreator=Me, pdfproducer=Me} \setlength{\parskip}{1ex}
\usepackage{tikz}

\usepackage{paralist}

\DeclareMathOperator{\sgn}{sgn}

\newcommand{\Fary}{Fáry}

\listfiles
\newcommand{\lstlabel}[1]{\label{lst:#1}}
\newcommand{\lstref}[1]{Listing~\ref{lst:#1}}
\newcommand{\Lstref}[1]{\lstref{#1}}

\DeclareMathOperator{\block}{block}
\newcommand{\naive}{na\"{\i}ve}


\newcommand{\reals}{\mathbb{R}}
\newcommand{\integers}{\mathbb{Z}}
\newcommand{\naturals}{\mathbb{N}}
\newcommand{\dist}{{d}}

\title{\MakeUppercase{Every Collinear Set is Free}}

\author{Bellairs Workshop on Geometry and Graphs 2017--18}

\begin{document}
\maketitle


\begin{abstract}
  We show that if a planar graph $G$ has a plane straight-line embedding
  in which a subset $S$ of its vertices are collinear, then there is a
  planar straight-line embedding of $G$ in which all vertices in $S$ are
  on the $y$-axis and in which they have prescribed $y$-coordinates.
  This solves an open problem posed by Ravsky and Verbitsky in 2008.
  In their terminology, we show that every collinear set is free.
  This result has applications in graph drawing, untangling, universal
  point subsets, and related areas.
\end{abstract}


\section{Introduction}

In a planar graph, $G=(V,E)$, a \emph{collinear set} is a set of vertices
$S\subset V$ such that $G$ has a plane straight-line embedding in which
all vertices in $S$ are drawn on a single line.  A collinear set $S$
is a \emph{free collinear set} if, for any collinear set of points
$X\subset\R^2$, $|X|=|S|$, $G$ has a plane straight-line embedding in
which the vertices of $S$ are drawn on the points in $X$.  Ravsky and
Verbitsky \cite{ravsky.verbitsky:on,ravsky.verbitsky:on-arxiv} ask the
following question:

\begin{quote}
   How far or close are parameters $\tilde{v}(G)$ and $\bar{v}(G)$? It
   seems that \emph{a priori} we even cannot exclude equality. To clarify
   this question, it would be helpful to (dis)prove that every collinear
   set in any straight line drawing is free.
\end{quote}

In the context of this quote, $\tilde{v}(G)$ and $\bar{v}(G)$ are the
respective sizes of the largest collinear set and largest free collinear
set in $G$.  In this note, we prove that $\tilde{v}(G)=\bar{v}(G)$ by
showing that every collinear set is a free collinear set.  

Da Lozzo \etal\ \cite{dalozzo.dujmovic.ea:drawing}
gave the following characterization of collinear sets:
\begin{thm}\thmlabel{collinear-set}
   A set $S$ of the vertices of a graph $G$ is a collinear set if and
   only if there exists a plane embedding of $G$ and a simple curve
   $C$ having both endpoints in a common face of $G$ and such that each
   vertex of $S$ is drawn on $C$ and the intersection of each edge with
   $C$ has at most one connected component.
\end{thm}
The surprising aspect of this characterization is that one can
simultaneously straighten the embedding of $G$ so that it becomes
a plane straight line embedding and straighten $C$ so that it becomes (say)
the y-axis while preserving the combinatorial relationship between $C$
and $G$.


\subsection{Proof Outline}

Without loss of generality we may assume that the line we are interested
in is the y-axis.  Let $C^-$ and $C^+$ denote the finite and infinite
connected components, respectively, of $\R^2\setminus C$, which we call
the \emph{interior} and \emph{exterior} of $C$, respectively.  We say
that an edge of $G$ \emph{crosses} $C$ if it contains one endpoint in
$C^-$ and one in $C^+$.

Tutte's convex embedding theorem \cite{tutte:how} allows one to draw an
internally 3-connected graph with the vertices of the outer face drawn
on any prescribed convex polygon.  If the vertices in $S$ form a path in
$G$, then no edge of $G$ crosses $C$. In this case, it is straightforward to
prove that $S$ is a free collinear set using two applications of Tutte's
Convex Embedding Theorem \cite{tutte:how}, one on the graph induced by
$V(G)\cap(C\cup C^-)$ and one on the graph induced by 
$V(G)\cap(C\cup C^+)$.

Thus, the main difficulty comes from edges of $G$ that cross $C$.
These edges must be drawn so that they cross the y-axis in prescribed
(and arbitrarily small) intervals between the locations of vertices
in $S$.  An extreme version of this (sub)problem occurs when $Q$ is an
embedded graph in which every edge of $Q$ crosses $C$ and we are given a
prescribed location at which each edge of $Q$ should cross the y-axis.
The most difficult instances occur when $Q$ is edge-maximal, meaning
that $Q$ is a quadrangulation.

In \secref{quadrangulations} we show that, given a quadrangulation $Q$
and a Jordan curve $C$ that intersects every edge of $Q$ in
exactly one point, it is possible to find a plane straight-line embedding
of $Q$ whose edges intersect the y-axis in a prescribed set of points.
This is done by showing that a certain system of linear equations has
a solution. This proof involves some linear algebra and some arguments
that use continuity.  An extension of this result then shows that some
independent set $S$ of \emph{split edges} in $Q$ can be contracted and
the resulting vertices placed at prescribed points on the y-axis.

In \secref{triangulations} we prove that every collinear set is free.
It turns out that the intuition that quadrangulations with prescribed
edge crossings is the hardest case can be made formal. In particular,
given a curve $C$, triangulation $G$ and a set $S\subset V(G)$ as
in \thmref{collinear-set}, a series of combinatorial reductions can
be performed on $G$ that convert it to a quadrangulation $Q$ with a
special set $S$ of split edges on which we can apply the results in
\secref{quadrangulations} to obtain a plane straight-line embedding of
$Q$ in which the vertices of $S$ are drawn at the appropriate location
on the y-axis. These reductions can then be undone to obtain a plane
straight-line embedding of $G$ with the vertices of $S$ at the appropriate
locations on the y-axis.

\secref{definitions}, next, begins our discussion with a collection of
definitions and results that we use throughout.


\section{Definitions}
\seclabel{definitions}

For a function $f:\R\to\R$, we use $\lim_{x\downarrow t} f(x)$ and
$\lim_{x\uparrow t} f(x)$ to denote the one-sided limits of $f(x)$
as $x$ approaches $t$ from above and below, respectively.  For a
point $x$ in a topological space, any open set that contains $p$ is a
\emph{neighbourhood} of $x$.  For any real number $x$ and any $\delta>0$,
the interval $[x-\delta,x)$ is a \emph{left-neighbourhood} of $x$ and
$(x,x+\delta)$ is a \emph{right-neighbourhood} of $x$.

A \emph{curve} $C$ is a continuous function from $[0,1]$ to $R^2$.
The points $C(0)$ and $C(1)$ are called the \emph{endpoints} of $C$.
A curve $C$ is \emph{simple} if $C(s)\neq C(t)$ for any $0\le s<t< 1$; it
is \emph{closed} if $C(0)=C(1)$.  A \emph{Jordan curve} $C:[0,1]\to\R^2$
is a simple closed curve.  When we talk about the order of the points
on a simple curve $C$ we mean the partial order over $\R^2$, where
$C(a)$ precedes $C(b)$ if and only if $0\le a<b$. For any $0\le a\le
b\le 1$, the \emph{subcurve} of $C$ between $a$ and $b$ is the curve
$C'(t)=C(a+t(b-a))$.  We will often fail to distinguish between a
curve $C$ and it's image $\{C(t):0\le t\le 1\}$.  In such cases we may
qualify the curve as \emph{open} in which case we are referring ot the
set $\{C(t):0< t< 1\}$.

All graphs $G$ considered in this paper are finite, simple, and
undirected.   We use $V(G)$ and $E(G)$ to denote the vertex set and
edge set of $G$, respectively.  

An embedding $\Gamma=(\varphi,\rho)$ of a graph $G$ consists
of a one-to-one mapping $\varphi:V(G)\to\R^2$ and a mapping $\rho$ from
$E(G)$ to curves in $\R^2$ such that, for each $xy\in E(G)$, $\rho(xy)$
has endpoints $\varphi(x)$ and $\varphi(y)$.

Starting immediately,  we will often say that $G$ is an embedded graph
without explicitly referring to the pair $\Gamma=(\varphi,\rho)$.
In these cases, we identify vertices of an embedded graph with their
points and edges with their curves. By default, an edge curve includes
its endpoints, otherwise we specify that it is an \emph{open} edge.

An embedding is \emph{clean} if the intersection between any two edges
of the embedding is a finite set.  

A \emph{straight-line embedding} is an embedding which each edge is a
line segment.  A \emph{plane embedding} is an embedding in which no two
edges intersect except possibly at their common endpoint.  A \emph{Fáry
embedding} is a plane straight-line embedding.

The \emph{faces} of an embedded graph $G$ are the maximal connected
subsets of $\R^2\setminus(\bigcup_{xy\in E(G)} xy$.  Note that one of
these components is unbounded and we call this the \emph{outer face}. The
other faces are called \emph{inner faces} or \emph{bounded faces}.
A \emph{chord} is an edge of $G$ with two endpoints on the outer face but whose
interior is not on the outer face.  When discussing plane embeddings,
we use the convention of listing the vertices of a face as they appear
when traversing the face in counterclockwise order.

A \emph{triangulation} is a plane embedded graph in which each face is
bounded by a 3-cycle.  A \emph{quadrangulation} is plane embedded graph
in which each face is bounded by a 4-cycle. Every quadrangulation has
$n\ge 4$ vertices and Euler's formula implies that it has $2n-4$ edges.
A \emph{near-quadrangulation} is a plane embedded graph in which each
inner face (but not necessarily the outer face) is bounded by a 4-cycle.
An \emph{outerplanar graph} is a plane embedded graph in which there is
a single face that is incident to every vertex.

A \emph{cutset} of a graph $G$ is a set of vertices whose removal
disconnects $G$.  A \emph{separating cycle} is a cutset whose vertices
form a cycle.  A \emph{separating triangle} is a separating cycle of
length 3.  If $G$ is an embedded graph, then any separating cycle $C$
corresponds to a Jordan curve.  In this case, the interior and exterior of
$C$ refer to the interior and exterior of the corresponding Jordan curve.


A \emph{contraction} of the edge $xy$ in a graph $G$ is the
process of identifying $x$ and $y$ to obtain a new graph $G'$ with
$V(G')=V(G)\cup\{v\}\setminus\{x,y\}$ and $E(G')=E(G)\setminus\{ab\in
E(G): \{a,b\}\cap\{x,y\}\neq\emptyset\}\cup\{va: xa\in E(G)\}\cup
\{va:ya\in E(G)\}$.  If $G$ is a triangulation and we contract the edge
$xy\in E(G)$, then the resulting graph $G'$ is also a triangulation
provided that $x$ and $y$ is not part of any separating triangle. (More
specifically, the embedding of $G$ extends to an embedding of $G'$.)

We say that a Jordan curve $C:[0,1]\to\R^2$ is \emph{nice} for an embedded
graph $G$ if $C$ intersects each edge of $G$ in at most one connected
component and the endpoint $C(0)=C(1)$ of $C$ is in the interior of the
outer face of $G$.  We say that $C$ is \emph{clean} (for $G$) if its
intersection with each edge of $G$ is either empty or a single point.
We say that $C$ is \emph{tidy} (for $G$) if it does not contain any
vertex of $G$.

We will make use of the following restatement of \thmref{collinear-set}
which follows from the proof in \cite{dalozzo.dujmovic.ea:drawing}:
\begin{thm}\thmlabel{dujmovic-frati}
   Let $G$ be a plane embedding and let $C:[0,1]\to\R^2$ be a nice
   Jordan curve for $G$.  Then $G$ has a \Fary\ embedding $\Gamma$
   in which the vertices of $G$ in $C$ are on mapped to the y-axis,
   the vertices of $G$ in $C^-$ are to the left of the y-axis and the
   vertices of $G$ in $C^+$ are to the right of y-axis. Furthermore,
   the sequence of vertices and edges encountered in $\Gamma$ while
   traversing the y-axis in the positive direction is identical to the
   sequence of vertices and edges encountered in $G$ while traversing $C$.
\end{thm}




\section{Quadrangulations}
\seclabel{quad}
\seclabel{quadrangulations}

In this section we develop tools for finding \Fary\ embeddings of
a quadrangulation $Q$ whose edges cross the y-axis at prescribed
locations. We begin with the tidy case in which the curve $C$ intersects each
open edge of $Q$ in exactly one point.

\subsection{Nice Clean Tidy Curves}

This section is devoted to proving the following result:

\begin{thm}\thmlabel{quad}
    Let
    \begin{compactitem}
    \item $Q$ be a quadrangulation with outer face $f$; 
    \item $C:[0,1]\to\R^2$ be a nice clean tidy Jordan curve for $Q$
     that intersects every edge of $Q$;
    \item $e_1,\ldots,e_m$ be the edges of $Q$ in the
    order they are intersected by $C$; 
    \item $y_1<\cdots<y_m$
    be any increasing sequence of numbers; and
    \item $\Delta$ be a triangle whose intersection with the y-axis
     is the segment $p_1p_m$ with endpoints $p_1=(0,y_1)$ and $p_m=(0,y_m)$.
    \end{compactitem}
    Then $Q$ has a \Fary\ embedding in which, for each
    $i\in\{1,\ldots,m\}$, the intersection of $e_i$ with the y-axis is
    a single point $(0,y_i)$ and the edges $e_1$ and $e_m$ are mapped
    to the two edges of $\Delta$ that intersect the y-axis.
\end{thm}

Refer to \figref{delta}.
Note that the requirement that the edges $e_1$ and $e_m$ map to $\Delta$
fixes the embedding of the outer face $f$ of $Q$. Indeed, the vertex of
$f$ common to $e_1$ and $e_m$ will be mapped to the vertex $\gamma$ of
$\Delta$ that is separated, by $C$, from the other two vertices, $\alpha$
and $\beta$.  The other two endpoints of $e_1$ and $e_m$ will be mapped
to $\beta$ and $\alpha$, respectively.  The two remaining edges of $f$
are then fixed by the requirement that they have endpoints at $\alpha$
and $\beta$ and intersect the y-axis in prescribed locations.

\begin{figure}
   \centering{\includegraphics{figs/delta}}
   \caption{The triangle $\Delta$ fixes the embedding of the outer face of $Q$.}
   \figlabel{delta}
\end{figure}

In the remainder of the proof, we will show that the unique embedding of
the outer face $f$ extends to the rest of the $Q$ so that there is exactly
one embedding of $Q$ that satisifies the requirements of \thmref{quad}.

\subsubsection{The Linear System $A\cdot s=b$}

We model this problem by a system of equations that has $m$ variables
$s_1,\ldots,s_m$ in which $s_i$ is the slope of the edge $e_i$, so $e_i$
is supported by the line $\{(x,y):y=s_ix + y_i\}$.  Note that, since
each vertex of $Q$ has degree at least 2, the values of $s_1,\ldots,s_m$
completely determine a straight-line embedding of $Q$.

Of course, not all choices of $s_1,\ldots,s_m$ determine an embedding of
$Q$ since all edges incident to a common vertex must meet a common point.
Among the choices of $s_1,\ldots,s_m$ that do define an embedding, not all
of them necesarily determine an embedding that satisfies the conditions
of the theorem. For instance, there is no \emph{apriori} reason to think
that the straight-line embedding obtained this way is a plane embedding.

Without loss of generality (by reflection through the y-axis and uniform
scaling of all quantities), assume that $\Delta=\alpha\beta\gamma$
has two vertices $\alpha$ and $\beta$ to the left of the y-axis and
the third vertex $\gamma$ to the right of the y-axis and is contained
in $[-1,1]^2$.  The outer face, $f$, of $Q$ has four edges $e_1$, $e_a$,
$e_b$, and $e_m$, where $1 < a < b < m$.  As discussed above, the slopes
$s_1$, $s_a$, $s_b$, and $s_m$ are completely determined by $y_1$, $y_a$,
$y_b$, $y_c$, and $\Delta$.

For every three edges $e_i$, $e_j$, and $e_k$ incident on the same
vertex $v$, the three supporting lines of $e_i$, $e_j$, and $e_k$ must
meet at a common point (the location of $v$) and therefore, any solution
$s=(s_1,\ldots,s_m)$ must satisfy the constraint:
\begin{equation}\eqlabel{slope} 
    s_i = \left(\frac{y_k-y_i}{y_k-y_j}\right) s_j 
          + \left(\frac{y_j-y_i}{y_j-y_k}\right)s_k \enspace .
\end{equation}
Note that, since $y_1,\ldots,y_m$ are given, this is a linear equation
in $s_1,\ldots,s_m$.

A result of Felsner \etal\ \cite[Lemma~2.7]{felsner.huemer.ea:binary}
implies that $Q$ has an orientation of its edges so that each vertex
except those on the outer face has in-degree 2, and each vertex of
the outer face has in-degree 1.  Let $\vec{Q}$ be the digraph obtained
from this orientation. We build a $m\times m$ system of equations in
the following way: For each directed edge $\vec{xy}$ of $\vec{Q}$
corresponding to the edge $e_i=xy$ in $Q$, we add the constraint
\eqref{slope} to the system, where $e_j$ and $e_k$ denote the two incoming
edges of $x$ in $\vec{Q}$ or---if $x$ is on $f$---the two edges incident
on $x$ that are edges of $f$.

This yields a system of equations $A\cdot s = b$, where $A$ is an $m\times
m$ matrix whose entries come from \eqref{slope}, $s=(s_1,\ldots,s_m)$
are the variables we wish to solve for, and $b$ is a column $m$-vector
whose entries also come from \eqref{slope}.  Some entries of $b$
are non-zero because the four slopes $s_1$, $s_a$, $s_b$, and $s_m$
are fixed.  We will show that $A\cdot s=b$ has a unique solution and
that this solution gives a \Fary\ embedding of $Q$.

It is clear that any solution $s$ to $A\cdot s=b$ determines a
straight-line embedding of $Q$ that satisfies the conditions of the the
theorem, but it is not clear that it determines a \Fary\ embedding of $Q$.
In particular, it could give an embedding in which edges cross each other.
As a first step, we consider solutions to $A\cdot s=b$ that satisfy some
ordering constraints.

\subsubsection{Ordering constraints}

Define a relation $\prec$ on $\{1,\ldots,m\}$ where $i \prec j$
if and only if
\begin{enumerate}
  \item $i \prec j$ and $e_i$ and $e_j$ are incident to a common vertex
  $v\in C^-$; or
  \item $i > j$ and $e_i$ and $e_j$ are incident to a common vertex $v\in C^+$.
\end{enumerate}
We say that a vector $s=(s_1,\ldots,s_m)$ satisfies $\prec$ if $s_i <
s_j$ for every pair $i,j\in\{1,\ldots,m\}$ where $i\prec j$.  

This definition captures the condition that vertices inside of $C$
should be drawn to the left of the y-axis and those outside of $C$
should be drawn to the right of the y-axis.  It is straightforward
to verify that $\prec$ is actually a partial order. Indeed, $i_1\prec
\cdots \prec i_r$ implies that, for each $j\in\{3,\ldots,r\}$, $y_{i_j}\in
(\min\{y_{i_{j-1}},y_{i_{j-2}}\}, \min\{y_{i_{j-1}},y_{i_{j-2}}\})$. Thus,
a chain $\prec$ corresponds to a sequence of strictly nested intervals.

\begin{lem}\lemlabel{order-gives-embedding}
   Any solution $s$ to $A\cdot s=b$ that satisfies $\prec$ yields a
   \Fary\ embedding of $Q$.
\end{lem}

\begin{proof}
   Devillers \etal\ \cite[Lemma~16]{devillers.liotta.ea:checking} show
   that, if $G$ is a plane embedding of a 2-connected graph and $G'$
   is a straight-line embedding of $G$ in which the cyclic order of the
   edges around every vertex in $G'$ is the same as the cyclic order
   of the edges around every vertex in $G$ and every face of $G$
   has a non-crossing embedding in $G'$, then $G'$ is a \Fary\ embedding.

   In our case, $G=Q$ and $G'=Q'$ is a straight-line embedding $Q'$ of
   $Q$ given by a solution to $A\cdot s = b$ that satisifes $\prec$.
   Since every edge of $Q$ intersects $C$ and the order of
   $y_1,\ldots,y_m$ is the same as the order in which $e_1,\ldots,e_m$
   intersect $C$, the ordering of the edges around each vertex in $Q'$
   is the same as in the embedding of $Q$. Therefore, $Q'$ is a \Fary\
   embedding of $Q$.
\end{proof}


\subsubsection{Strong Ordering Constraints}

For any $\epsilon \ge 0$, we say that $s=(s_1,\ldots,s_m)$ satisfies
the \emph{$\epsilon$-strong ordering constraints} if, for each $1\le
i<j\le m$ such that $i\prec j$, $s_i < e_j - \epsilon$.  Observe that any
solution $s$ that satisfies the $\epsilon$-strong ordering constraints
also satisifies $\prec$. The following lemma shows that the converse is
also true:

\begin{lem}\lemlabel{weak-to-strong}
   Any solution $s$ to $A\cdot s=b$ that satisifies $\prec$ satisfies the
   $\epsilon$-strong ordering constraints for
   for all $\epsilon<\min\{|y_i-y_j| : 1\le i< j\le m\}$.
\end{lem}

\begin{proof}
   Otherwise, the solution $s$ would yields an embedding of $Q$ in
   which the vertex $v$ incident to $e_i$ and $e_j$ has x-coordinate
   $(y_i-y_j)(s_j-s_i)$ outside the interval $[-1,1]$. However,
   \lemref{order-gives-embedding} implies that the outer face of the
   embedding is contained $\Delta\subset[-1,1]^2$, a contradiction.
\end{proof}

\subsubsection{Uniqueness of solutions satisfying $\prec$}

\begin{lem}\lemlabel{unique}
   If $s$ is a solution to $A\cdot s=b$ that satisfies $\prec$, then $s$ is 
   the unique solution to $A\cdot s=b$.
\end{lem}

\begin{proof}
   Suppose that there is a solution $s$ to $A\cdot s=b$ that satisifies $\prec$,
   but it is not unique.  Since $A\cdot s=b$ is a linear system, this implies
   that there is an entire (at least) 1-parameter family of solutions,
   i.e., there is a non-trivial $m$-vector $r$ such that, for every
   $\lambda\in(-\infty,\infty)$, $A(s+\lambda r)=b$.

   The vector $r=(r_1,\ldots,r_m)$ has at least four zero entries
   $r_1=r_a=r_b=r_m$ since the slopes $s_1$, $s_a$, $s_b$, and $s_m$
   are fixed by $y$ and $\Delta$.  Since $Q$ is connected, this implies
   that there is at least one vertex $v$ with two incident edges $e_i$
   and $e_j$ such that $r_i=0$ and $r_j\neq 0$.  Set $\lambda^* =
   (s_i'-s_j')/r_j$ and observe that $s^*=s+\lambda^* r$ is a solution
   to $As^*=b$ in which $s_i^*=s_j^*$.  Therefore, $s^*$ is a solution that
   does not satisify $\prec$ (the edges $e_i$ and $e_j$
   are parallel).  

   Without loss of generality, assume $\lambda^* >0$. Then, by continuity,
   this implies that there is some value of $\lambda\in (0,\lambda^*)$
   such that the solution $s+\lambda r$ satisifies $\prec$ but does
   not satisfy the $\epsilon^*$-strong ordering constraint.  This is a
   contradiction, so we conclude that $s$ must be the unique solution
   to $A\cdot s=b$.
\end{proof}

\subsubsection{A Family of Linear Systems}

   Note that $A$ and $b$ are functions of $y=(y_1,\ldots,y_m)$
   and the 6-coordinates $\Delta=(\delta_1,\ldots,\delta_6)$, of
   the three vertices of $\Delta$.  We make this explict by writing
   $A_1=A(y,\Delta)$ and $b_1=b(y,\Delta)$.  \thmref{dujmovic-frati}
   implies that there is some straight-line drawing of $Q$ and some
   $y_1'<\cdots< y_m'$ such that, for each $i\in\{1,\ldots,m\}$, $e_i$
   intersects the y-axis in exactly one point $(0,y_i')$.  Again, without loss
   of generality, we may assume that $\Delta'\subset [-1,1]^2$ and that
   $\Delta'$ has two vertices on the left of the y-axis and one vertex
   on the right.

   Thus far, we have established that there exists $y'=(y_1',\ldots,y_m')$
   and $\Delta'=(\delta_1',\ldots,\delta_6')$ such that the
   system $A(y',\Delta')\cdot s' = b(y',\Delta')$ has at least one
   solution $s'=(s_1',\ldots,s_m')$.  We now define a continuous
   family of linear systems that interpolates between the systems
   $A(y',\Delta')\cdot s=b(y',\Delta')$ and $A(y,\Delta)\cdot s=b(y,\Delta)$.

   For $0\le t\le 1$, define $y(t) = (1-t)y' + ty$, for each
   $i\in\{1,\ldots,m\}$, let $y_i(t)=(y(t))_i$.  Observe that, for any
   $1\le i< j\le m$ and any $0\le t\le 1$,
   \[
       y_j(t) - y_i(t) = (1-t)(y'_j-y'_i) + t(y_j-y_i) > 0 \enspace .
   \]
   Let $\epsilon^*=\min\{|y(t)_i-y(t)_j| : 1\le i< j\le m,\, 0\le t\le 1\}$
   and observe that $\epsilon^* >0$.

   For each $0\le t\le 1$, let $\Delta(t)=(1-t)\Delta' + t\Delta$ and define
   $A_t=A(y(t),\Delta(t))$ and $b_t=b(y(t),\Delta(t))$.  The entries in
   $A_t$ and $b_t$ are derived from \eqref{slope} and the denominators in
   \eqref{slope} have absolute values bounded from below by $\epsilon^*$.
   Thus, each entry in $A_t$ and $b_t$ is finite and is a uniformly
   continuous function of $t$.  Furthermore, for any $0\le t\le 1$,
   \lemref{weak-to-strong} applies to $A_t\cdot s =b_t$, so any solution
   $s$ that satisfies $\prec$ also satisifies the $\epsilon^*$-strong
   ordering constraints.

\subsubsection{Existence of solutions to $A_t\cdot s=b_t$}


\begin{lem}\lemlabel{uniqueness}
   For every $0\le t\le 1$, the system $A_t\cdot s=b_t$ has a unique solution
   and this solution satisfies $\prec$.
\end{lem}

\begin{proof}
   Recall that, since $A_t$ is an $m\times m$ matrix, the system $A_t\cdot
   s=b_t$ has a unique solution $s$ if and only if $\det(A_t)\neq 0$.
   Cramer's rules states that, in this case the solution $s$ is given
   by $s(t)=(s_1(t),\ldots,s_m(t))$ where, for each $i\in\{1,\ldots,m\}$,
   \[ 
       s_i(t) = \frac{\det(A_t^i)}{\det(A_t)} \enspace ,
   \]
   and $A_t^i$ denotes the matrix $A_t$ with its $i$th column replaced
   by $b_t$. The function $f(t)=\det(A_t)$ is uniformly continuous
   because it is a polynomial over the entries in $A_t$ and each of
   these entries are uniformly continuous in $t$. By the same reasoning,
   the function $f_i(t)=\det(A_t^i)$ is also uniformly continuous for
   each $i\in\{1,\ldots,m\}$.

   We have already established that $A_0s=b_0$ has a solution $s=s'$
   that satisfies $\prec$ and therefore, by \lemref{unique}, this
   solution is unique, so $\det(A_0)\neq 0$.  To show that $A_t\cdot s=b_t$
   has a unique solution, there are three possible events we must rule out:
   \begin{enumerate}
     \item for some $0<t\le 1$, $\det(A_t)\neq 0$ but
           the solution to $A_t\cdot s=b_t$ does not satisfy $\prec$;
     \item for some $0<t\le 1$, $\det(A_t)=0$ and $A_t\cdot s=b_t$ has multiple 
           solutions; or
     \item for some $0<t\le 1$, $\det(A_t)=0$ and $A_t\cdot s=b_t$ has no solution.
   \end{enumerate}
   If none of these three events occurs then, for all $0\le t\le
   1$, $\det(A_t)\neq 0$ and the $A_t\cdot s=b_t$ has a unique solution that
   satisfies $\prec$, so the proof is complete.

   Suppose therefore, for the sake of contradiction, that $t^*\le 1$
   is the minimum value of $t$ for which one of the three events occurs.
   We can rule out the first event by an argument similar to the one which
   shows the uniqueness of $s'$.  That is, by arguing that there exists
   a $t<t^*$ such that the solution to $A_{t}\cdot s=b_{t}$ satisifies $\prec$
   but does not satisfy the $\epsilon^*$-strong ordering constraint.

   The second and third events are distinguished by whether or not
   $\lim_{t\uparrow t^*} s(t)$ exists.  In the case where it does
   exist, the second event has occured and $s^*=\lim_{t\uparrow t^*}
   s(t)$ is a a solution to $A_{t^*}\cdot s^*=b_{t^*}$.  We claim that,
   in this case $s^*$ satisifies $\prec$ and, by \lemref{uniqueness},
   is therefore unique.  To see why $s^*$ satisifes $\prec$, recall
   that $s(t)$ satisfies the $\epsilon^*$-strong ordering constraint
   for all $0\le t<t^*$.  If $s^*$ did not satisfy $\prec$, then there
   would be a a discontinuity where, for some $i,j\in\{1,\ldots,m\}$,
   $s_i(t)-s_j(t)\ge \epsilon^*$ for all $t$ in a left-neighbourhood of
   $t^*$, but $s_i(t)-s_j(t) \le 0$ at $t=t^*$.

   Thus, it remains only to rule out the possibility that $\det(A_{t*})=0$
   and $A_{t^*}\cdot s=b_{t^*}$ has no solutions because $\lim_{t\uparrow
   t^*} s(t)$ does not exist.  Define the sets
   \begin{enumerate}
     \item $H=\{i\in \{1,\ldots,m\}:\text{$\lim_{t\uparrow t^*} s_i(t)$ exists}\}$; and
     \item $V=\{1,\ldots,m\}\setminus H$.
   \end{enumerate}
   An event of Type~3 occurs if and only if $V$ is non-empty.
   We now show that this is not possible.  The partition
   $(H,V)$ of $\{1,\ldots,m\}$ has he following
   properties:
   \begin{enumerate}
    \item $H$ contains the four edges $e_1$, $e_m$, $e_a$ and $e_b$
      on the outer face.
    \item If $i,j\in F$ and $e_i$ and $e_j$ are incident to a common
      vertex $v$ then $k\in F$ for all edges $e_k$ incident to $v$.
    \item If $s_i \prec s_j \prec s_k$ and $i,k\in F$, 
      then $j\in F$.
   \end{enumerate}
   \lemref{partition}, below, shows that in any such partition, the set
   $V$ is empty, therefore no event of Type~3 occurs.  This 
   completes the proof.
\end{proof}

\begin{lem}\lemlabel{partition}
   Let $Q'$ be a subgraph of $Q$ in which each inner face is a
   quadrilateral and let $E=\{i:e_i\in E(Q')\}$.  Let $(V,H)$ be a
   partition of $E$ such that
   \begin{compactenum}
    \item $H$ contains all $i\in E$ such that $e_i$ is on the outer face 
      of $Q'$;
    \item if $i,j\in F$ and $e_i$ and $e_j$ are incident to a common
      vertex $v$ then $k\in F$ for all edges $e_k$ incident to $v$; and
    \item if $i \prec j \prec k$ and $i,k\in F$, 
      then $j\in F$.
   \end{compactenum}
   Then $H=\{1,\ldots,m\}$ and $V=\emptyset$.
\end{lem}

\begin{proof}
   We call an edge $e_i$ an \emph{h-edge} (horizontal-edge) if $i
   F$ and a \emph{v-edge} (vertical-edge) otherwise.  The proof is
   by induction on the number of bounded faces of $Q'$.  The base case
   occurs when $Q'$ has all its edges on the outer face, in which case 
   the conclusion of the lemma is implied by Condition~1.

   If $Q'$ is not connected then we can apply induction to each component
   of $Q'$ separately. If $Q'$ has a cut vertex, $v$, whose removal
   separates $Q'$ into components $A_1,\ldots,A_r$ then, for each
   $i\in\{1,\ldots,r\}$, we can apply induction on the subgraph of $Q'$
   induced by $V(A_i)\cup\{v\}$.  
   Otherwise $Q'$ is a 2-connected \emph{near-quadrangulation} whose outer face 
   is a simple cycle.

   Let $q=abcd$ be a face of $Q'$ that $C$ intersects first at $ab$
   and then at $bc$.  Then we call $c$ the \emph{reflex vertex} of $q$.
   We claim that, if the reflex vertex $c$ is incident on two h-edges,
   then all edges of $q$ are h-edges. Indeed, with a slight abuse of
   notation, we have $bc \prec ab\prec da\prec cd$.  Then the reflex
   vertex is $c$, so $bc$ and $cd$ are h-edges and, by Condition~3, $ab$
   and $da$ are also h-edges.

   Refer to \figref{lemma-y}.
   Now, since the outer face of $Q'$ is a simple cycle, contains at
   least four vertices, and $C$ intersects every edge of this face, the
   outer face of $Q'$ must contain a path $u,v,w$ such that $C$ exits
   an innner face through $uv$ and enters an inner face through $vw$.
   This implies that $v$ is a reflex-vertex of some bounded face $vabc$
   of $Q'$.  Indeed, $vc$ is the first edge incident to $v$ crossed by
   $C$ and $va$ is the last edge incident to $v$ crossed by $C$.

  \begin{figure}
     \centering{\includegraphics{figs/lemma-y}}
     \caption{The proof of \lemref{partition}.}
     \figlabel{lemma-y}
  \end{figure}

   All edges incident on $v$ are h-edges and $v$ is a reflex vertex of
   $q$, therefore all edges of $q$ are h-edges.
%   Furthermore,
%   neither $a$ nor $c$ are on the outer face of $Q'$ since, otherwise,
%   $v,a,b,c$ or $a,b,c,v$ would be an h-separating path.  
   We obtain a new graph $Q''$ by splitting $u$ into two vertices $x$
   and $y$.  We make the vertex $x$ adjacent to $u$ and every neighbour
   $z$ of $v$ such $C$ intersects $vz$ before it intersects $vu$.  We make
   $y$ adjacent to all of $v$'s neighbours that are not adjacent to $x$.
   In $Q''$, the $q$ is part of the outer face.  $Q''$ has one less inner
   face that $Q'$.  All edges on the outer face of $Q''$ are h-edges.
   In $Q''$ some of the $\prec$ relations involving edges incident
   to $v$ are missing, but no new ones are introduced, so $Q''$ still
   satisifies Condition~3.  Thus, we can apply induction to conclude
   that all edges of $Q''$.  Every edge of $Q'$ is present in $Q''$ or
   has a representative (incident to $x$ or $y$) in $Q''$.  Therefore,
   all edges of $Q'$ are h-edges.
\end{proof}
%   Let $Q^-\subset\R^2$ denote the union of all the edges, vertices, and
%   faces of $Q'$.  A \emph{h-separating} path $P=v_1,\ldots,v_k$ in $Q'$
%   is a path consisting only of $h$ edges and such that $Q^-\setminus
%   P$ has more than one connected-component.  If $Q'$ contains an
%   h-separating path $P$, then we consider each component $A$ of
%   $Q^-\setminus P$ and let $Q'_A$ denote the subgraph of $Q'$ containing
%   only the edges and vertices of faces of $Q'$ contained in $A$.  Now,
%   $Q'_A$ satisifies the conditions of the lemma and has few bounded
%   faces than $Q'$ so, by induction, every edge in $Q'_A$ is an h-edge.
%
%   Note that the assumption that $Q'$ contains no h-separating path has
%   several implications since (i)~a cut vertex is an h-separating path;
%   and (ii)~a chord is an h-separating path.  Thus, we may assume that
%   $Q'$ is 2-connected and has no chords.  
%
%   The removal of h-separating paths also ensures that $Q'$ has no
%   degree-2 vertex on its outer face.  Indeed, if $v$ is such a vertex
%   with neighbours $x$ and $y$, then it forms a quadrilateral face $vxay$
%   with some vertex $a$.  By Condition~2, $xa$ and $ay$ are h-edges
%   and therefore $xay$ is an h-separating path.
%
%   The removal of separating paths also ensures that $Q'$ has no edge $vw$
%   on its outer face where $vw$ is on the boundary of a quadrilaterial
%   $vwab$ such that $ab$ is an h-edge.  If it did then, by Conditions~1
%   and 2, $va$ and $bw$ are both h-edges, so $vabw$ is an h-separating
%   path.


%
%   Let $u=v_1,\ldots,v_r=w$ be the neighbours of $v$ in $Q'$ as
%   they appear in counterclockwise order around $v$ and, for each
%   $i\in\{1,\ldots,r-1\}$, let $q_i$ be the face having $vv_i$ and
%   $vv_{i+1}$ on its boundary, and let $a_i\not\in\{v,v_{i},v_{i+1}\}$
%   be the fourth vertex of $q_i$.
%
%   We know that $a_1v_2$ is a v-edge, otherwise $v_1,a_1,v_2,v$ would
%   be an h-separating path.  Similarly, $v_{r-1}a_{r-1}$ is a v-edge
%   otherwise $v,v_{r-1},a_{r-1},v_r$ would be an h-separating path.
%   We also know that $v_j$ is not on the outer face of $Q'$ for any
%   $j\in\{2,\ldots,r-1\}$ otherwise $vv_i$ would be a chord.
%
%   Let $q_i$, $i\in\{2,\ldots,r-2\}$ be the face of $Q'$ in which $v$
%   is a reflex vertex.  Then we consider a new graph $Q''$ obtained by
%   splitting $v$ into two vertices $x$ and $y$ in which $x$ is adjacent
%   to $v_1,\ldots,v_i$ and $y$ is adjacent to $v_{i+1},\ldots,v_r$.
%   and whose embedding is amost the same as $Q'$ except that $x$ and
%   $y$ are slightly separated so that $q_i$ is part of the outer face
%   of $Q''$.
%
%   If we define the edges of $Q''$ incident on $x$ and $y$ to be h-edges
%   then $Q''$ satisifies the conditions of the lemma and has one fewer
%   face than $Q'$.  Thus, by induction, every edge of $Q''$ is an h-edge.
%   However, with the exception of edges incident to $v$ in $Q'$ and $x$ and $y$ $Q''$, $Q'$ and $Q''$ have the same edges. We conclude that all edges of $Q'$ ar e h-edges.
%
%   
%
%   We claim that, for any bounded face $q$ of $Q'$ that include a vertex $v$
%   on the outer face of $Q'$, the vertex $v$ is  not a reflex vertex
%   of $q$.  Let $v$ be a vertex on the outer face of $Q'$ and let
%   $a_1,\ldots,a_r$ be the edges of $Q'$ incident on $Q$ ordered in 
%   the counterclockwise order they appear around $q$ and such that $a_1$
%   and $a_r$ are on the outer face of $Q'$.  
%
%   Without loss of generality, we may assume that $v\in C^-$, in which
%   case it is straightforward to check that there is some $\alpha$ such
%   that $C$ crosses  $a_\alpha,a_{\alpha+1}, a_k,a_1,\ldots,a_{\alpha-1}$
%   in this order.
%
%   For each $i\in\{1,\ldots,r-1\}$, let $q_i$ be the bounded
%   face of $Q'$ having $a_i$ and $a_{i+1}$ on its boundary and let
%   $u_i\not\in\{v,a_i,a_{i+1}\}$ be the vertex of $q_i$ not adjacent
%   to $v$.  Both edges $a_1$ and $a_r$ are h-edges, so, by Condition~2
%   $a_1,\ldots,a_r$ are all h-edges.  We have also determined that $q_1$
%   and $q_{r-1}$ each have exactly one edge, $u_1a_2$ and $u_{k-1}a_{k-1}$
%   respectively, that is a v-edge, otherwise we would have removed $a_1$
%   or $a_r$ and applied induction.  Therefore, $v$ is not a reflex vertex
%   in $q_1$ or in $q_{r-1}$.  By similar reasoning, we can deduce that,
%   if $v$ is a reflex vertex in $q_i$ for some $i\in\{2,\ldots,r-2\}$,
%   then all edges of $q_i$ are h-edges.
%
%   This is the picture so far: $v, v_1, v_2$ is a path of h-edges and
%   $u_1,v_2,u_2$ is a path of v-edges, and $u_{i-1},v_{i},u_i$ is a path
%   of h-edges.  Therefore, there is a minimum $j$ $j\in\{3,\ldots,i\}$
%   such that $u_{j-1},v_j,u_j$ is a path of h-edges.  We claim that this
%   situation is not possible. In fact we make a slightly stronger claim:
%
%   \begin{clm}\clmlabel{yuck}
%      $Q'$ does not contain any cycle $A=v,v_1,u_1,\ldots,v_{j-1},u_{j-1},v_j$ such that
%   \begin{compactenum}
%      \item $vv_i$ is an h-edge for each $i\in\{1,\ldots,j\}$; 
%      \item $v,v_1,u_1$ and $v_{j-1},u_{j-1},v_j$ are paths of h-edges;
%      \item $u_1,v_2,u_2$ and $u_{j-2},v_{j-1},u_{j-1}$; are paths of v-edges; 
%      \item $v_2,u_2,\ldots,v_{j-1},u_{j-1}$ does not contain two consecutive h-edges.
%      \item No edge of $v_1,u_1,\ldots,v_j,u_j$ is on the outer face of $Q'$.
%   \end{clm}
%
%   \begin{proof}[Proof of \clmref{yuck}]
%      Consider some bounded face $q$ of $Q'$ that is outside of $A$ and shares one, two, or three edges of $u_2,v_3,\ldots,v_{j-1},u_{j-1}$.
%
%
%
%$v_quadrilateral face $q'$
%
%   is some 
%
%   Therefore, if there is some $i\in\{2,\ldots,r-1\}$ for which $v$ is a reflex vertex, then there exists some 
%
%   
%
%
%
%Suppose, therefore, that there is some minimum $i$ such
%   that 
%
%   Suppose now that there is some minimum $i$ such that $v$
%   is a reflex vertex in $q_i$.  This implies 
%
%
%   Since $a_i$ and $a_{i+1}$ are h-edges,
%   this implies that $q_i$ consists entirely of h-edges.  However, this implies
%   that $a_i$ has an endpoint $x\neq v$ with two incident h-edges.  But this
%   implies that 
%
%
%  But this means that $q_i$
%
%   We claim that, for any bounded
%   face of $Q'$ that includes a vertex $v$ on the outer face of $Q'$, the
%   vertex $v$ is  not a reflex vertex of $Q'$. 
%   
%
%
% 
%   In our case, the vertex $v$ is on the outer face of $Q'$ so both edges of $q$ incident to $v$ are h-edges.  This means
%
%vertex $v$ o the outer face of $Q'$ and any bounded face $q$ of $Q'$ that includes $v$
%
%   In a quadrilateral face of $Q'$, we call the vertex 
%   ....
%%
%   Now, consider the order in which the Jordan curve $C$ intersects
%   the edges of the outer face of $Q'$.  Since $C$ intersects each edge
%   exactly once and is a Jordan curve, there exists two edges $xv$ and
%   $vy$ on the outer face of $Q'$ such that the minimal subcurve of $C$
%   that intersects $xv$ and $xy$ does not intersect any other edges of
%   the outer face of $Q'$.  Without loss of generality, assume that $v\in
%   C^-$ and $x,y\in C^+$.   There are two quadrilateral faces $vxab$ and
%   $vcdy$ of $Q'$ incident on these two edges.  Since $Q'$ contains no
%   cut-path of length 2, $a\neq d$, though it may be the case that $b=c$.
%
%   
%
%   Let $J$ be the set of edges of $Q'$ that contain exactly one vertex
%   on the outer face of $Q'$ and let $B$ the set of vertices on outer
%   face of $Q''$.     
%   Observe that any vertex $v$ in $B$ is incident to at most one edge
%   of $J$. Otherwise, two of $v$'s consecutive edges $va$ and $vc$
%   would belong to $J$, forming a quadrilateral face $vabc$ in which $b$
%   is a degree-2 vertex on the outer face of $Q'$.
%
%   Therefore, every vertex $v\in B$ is incident to at most one edge of $J$.
%
%
%If $v$ were incident to two or more edges of $J$
%   We claim that every vertex in $B$ is incident to at most one edge
%   in $J$.  To see this, let $v$ be some vertex in $B$ and note that $Q'$
%   has a quadrilateral face $abcv$ that contains $v$ and at least one
%   vertex not in $Q''$.   If $ 
%
%Consider the outer face of the graph obtained by removing all
%   vertices on the outer face of $Q'$.
%
%   Any two edge $xa$ and $xb$ of $Q$ incident to the common vertex $x$
%   define two \emph{corners} $\measuredangle axb$ and $\measuredangle
%   bxa$.  If $x\in C^-$ and $C$ intersects $xa$ before intersecting $xb$
%   or if $x\in C^+$ and $C$ intersects $xb$ before intersecting $xa$,
%   then we say that $\measuredangle axb$ is a 0-corner.  Otherwise, we
%   say that $\measuredangle axb$ is a 1-corner.  Note that $\measuredangle
%   axb$ is a 1-corner if and only if $\measuredangle xba$ is a 0-corner.
%
%%   If $v_1,\ldots,v_\tau$ is a walk around a face of $Q$, in clockwise
%%   order, then there are exactly $\tau-3$ indices $i$ such that
%%   $\measuredangle v_{i-1}v_iv_{i+1}$ is a 1-corner. Geometrically,
%%   this can be interpreted as the statement that any simple $\tau$-gon,
%%   each of whose edges properly crosses the y-axis, has $\tau-3$ reflex
%%   vertices.\footnote{A facial walk in $Q$ may not define a simple
%%   polygon, but it does define a weakly-simple polygon and this is
%%   sufficient.} This implies that, for any face of $Q$ whose vertices
%%   in clockwise order are $abcd$, exactly one of $\measuredangle abc$,
%%   $\measuredangle bcd$, $\measuredangle cda$, and $\measuredangle dab$
%%   is a 1-corner.
%
%   Let $H$ be a connected component of the subgraph of $Q$ induced by
%   the endpoints of all edges $e_i$ with $i\in P\cup N$.  Note that $H$
%   does not contain any vertex of the outer face because Conditions~1 and
%   2 of the lemma imply that every edge incident an outer face vertex is
%   in $F$.  An edge of $Q$ is a \emph{cut edge} if it has one vertex of 
%   the outer face of $H$ and one vertex not in $H$.  
%
%   Let $R$ be the graph that contains every edge of every face of
%   $Q$ that contains a cut edge, and let $v_1,\ldots,v_\tau$ be the
%   sequence of vertices encountered while traversing the outer face of
%   $R$ in clockwise order.  Define $v_0=v_\tau$ and $v_{\tau+1}=v-1$.
%   there are exactly $\tau-3$ indices $i$ such that $\measuredangle
%   v_{i-1}v_iv_{i+1}$ is a 1-corner. Geometrically, this can be
%   interpreted as the statement that any simple $\tau$-gon, each
%   of whose edges properly crosses the y-axis, has $\tau-3$ reflex
%   vertices.\footnote{A facial walk in $Q$ may not define a simple
%   polygon, but it does define a weakly-simple polygon and this is
%   sufficient.}
%
%   We aim to show that this situation is impossible; for each
%   $i\in\{1,\ldots,\tau\}$, $\measuredangle v_{i-1}v_iv_{i+1}$ is a
%   0-corner, yielding a contradiction for $\tau \ge 4$.  We call an
%   edge $e_i$ an \emph{h-edge} (horizontal-edge) if $i\in F$ and a
%   \emph{v-edge} (vertical-edge) otherwise.
%
%   For each $i\in\{1,\ldots,\tau\}$, the graph $R$ contains some
%   number of edges $\alpha_1,\ldots,\alpha_k$ incident to $v_i$, where
%   $\alpha_1=v_{i-1}v_i$ and $\alpha_k=v_iv_{i+1}$.  Each consecutive pair
%   of edges is on the boundary of some quadrilateral $q=abcv_i$ in $Q$.
%   We claim that $\measuredangle cv_ia$ is a 0-corner.  This claim is
%   easily established using Condition~3 and case analysis of the number
%   of h-edges in $q$ (TODO: Do the analysis and make a figure).  
%   But this implies that
%   $\measuredangle v_{i-1}v_iv_{i+1}$ is 0-corner.
%\end{proof}

This completes the proof of \thmref{quad}.  We remark that this proof
actually shows something stronger; that for any $y_1'<\cdots<y_m'$ and
any $y_1<\cdots<y_m$, there is a continuous morph
between a drawing $Q'$ in which each edge $e_i$ crosses the y-axis at
$y_i'$ and a drawing $Q$ in which each edge $e_i$ crosses the y-axis
at $y_i$.  At any stage in this morph, all edges cross the y-axis and,
for each edge $e_i$, the crossing point between $e_i$ and the y-axis
moves linearly from $y_i'$ to $y_i$.

\subsection{From Quadrangulations to Collinear Sets}

To show that a collinear set in a triangulation $T$ is free, we will
reduce $T$ to a graph $T'$ that is not quite a quadrangulation.  However,
$T'$ can be made into a quadrangulation $Q$ satisfying the requirements
of \thmref{quad} by splitting each vertex on $C$ into a short edge. The
purpose of this section is to show that, given the drawing of $Q$ from
\thmref{quad}, this splitting can be undone by contracting these split
edges so that each split edge again becomes a vertex that is placed at
the appropriate place on the y-axis.  We begin with a geometric lemma:

\begin{lem}\lemlabel{convergence}
  Let $abcd\subset[-1,1]^2$ be a quadrilateral whose edges
  $e_1=ab$, $e_2=bc$, $e_3=cd$ and $e_4=da$ intersect the y-axis at
  $y_1<y_2<y_3<y_4$, and define $\epsilon=y_3-y_2$ and $g=y_4-y_3$.  Then
  the the x-coordinate of $c$ has absolute value at most $\epsilon/g$ and
  the distance between $c$ and $(0,y_2)$ is at most $\sqrt{5}\epsilon/g$.
\end{lem}

\begin{proof}
  Refer to \figref{convergence}.  Without loss of generality assume
  $a$ and $c$ are the to left of the y-axis and $b$ and $d$ are to
  the right of the y-axis.  Define $r$ and $q$ so that $c=(-r,y_2-q)$
  so that we want to prove $r\le\epsilon/g$ and $\sqrt{r^2+q^2}\le
  \sqrt{5}\epsilon/g$.

  \begin{figure}
     \centering{\includegraphics{figs/convergence}}
     \caption{The proof of \lemref{convergence}.}
     \figlabel{convergence}
  \end{figure}

  For each $i\in\{1,\ldots,4\}$, let $s_i$ denote the slope of $e_i$.
  Then $s_2=q/r$, $s_3=(q+\epsilon)/r=s_2+\epsilon/r$.
  The $x$-coordinate of $d$ is
  \begin{align*}
      d_0 & = \frac{g}{s_3-s_4} \\ 
          & > \frac{g}{s_3-s_1} & \text{(since $s_4 > s_1$)} \\
          & > \frac{g}{s_3-s_2} & \text{(since $s_1 > s_2$)}\\
          & = \frac{rg}{\epsilon} \enspace .
  \end{align*}
  But, since $abcd\subset[-1,1]^2$,  $1\ge d_0> rg/\epsilon$.
  Rewriting this gives $r < \epsilon/g$.

  The $y$-coordinate of $d$ is
  \begin{align*}
     d_1 & = y_4 + d_0 s_4 \\
     & > y_4 + d_0 s_2 & \text{(since $s_4>s_1>s_2$)} \\
     & = y_4 + d_0 \cdot\frac{q}{r} & \text{(since $s_2=q/r$)} \\
     & > y_4 + \frac{rg}{\epsilon}\cdot\frac{q}{r} & \text{(since $d_0>rg/\epsilon$)} \\
     & = y_4 + \frac{gq}{\epsilon} \\
     & > -1 + \frac{gq}{\epsilon} \enspace. \\
  \end{align*}
  Again $1\ge d_1 > -1 + \frac{gq}{\epsilon}$ and rewriting this gives
  $q < 2\epsilon/g$.  Therefore $\sqrt{r^2+q^2} \le \sqrt{5}\epsilon/g$.
\end{proof}

Let $Q$ be a quadrangulation satisfying the preconditions of
\thmref{quad}. Then we say an edge $xy$ is a \emph{split edge} of $Q$
(with respect to $C$) if the minimal subcurve of $C$ that intersects all
edges of $Q$ incident to $x$ or $y$ does not intersect any other edges
of $Q$. (See \figref{split-edge}.)

\begin{figure}
   \centering{\includegraphics{figs/split-edge}}
   \caption{A split edge $xy$ in a quadrangulation.  All split edges are shown in bold and the portion of $C$ that proves $xy$ is a split edge is highlighted}
   \figlabel{split-edge}
\end{figure}

\begin{cor}\corlabel{split-edges}
  Let $Q$, $C$, $e_1,\ldots,e_m$, $y_1,\ldots,y_m$, and $\Delta$
  be as in \thmref{quad}.  Let $xy$ be a split edge of $Q$ with
  respect to $C$, let $I=\{i:\text{$e_i$ is incident on $x$ or
  $y$}\}$, let $\epsilon=\max\{|y_i-y_j|:i,j\in I,\, i\neq j\}$, and
  let $g=\min\{|y_i-y_j|: i\in I,\, j\in\{1,\ldots,m\}\setminus I\}$.
  Then, in the drawing of $Q$ produced by \thmref{quad}, the absolute
  value of $x$ and $y$'s x-coordinates is at most $\epsilon/g$ and the
  distance between $x$ and $y$ is at most $2\sqrt{5}\epsilon/g$.
\end{cor}

\begin{proof}
   The edge $xy$ is incident to two quadrilaterals. Applying
   \lemref{convergence} to one of these establishes the distance bound
   for $x$ and applying \lemref{convergence} to the other establishes
   the distance bound for $y$.
\end{proof}

A set $S$ of split edges in $Q$ is \emph{independent} if there is no
edge in $Q$ that joins the endpoints of two distinct edges in $S$.
Given an independent set $S$ of split edges, we define the graph $Q_S$
by contracting each edge $xy$ in $S$ and placing the resulting vertex at
the intersection of $C$ and $xy$.  Note that, since $S$ is independent,
each edge of $Q_S$ intersects $C$ in exactly one point (though it may
be an endpoint).

We need a generalization of \thmref{quad} that allows us to prescribe
the y-coordinates of edges and vertices of $Q_S$ intersected by $C$.
This results in an annoying case distinction that occurs when $C$ contains
a vertex on the outer face of $Q_S$ (because $S$ contains an edge of the
outer face of $Q$).  To deal with this, we need some restrictions on the
triangle $\Delta$.  We say that a triangle $\Delta=\alpha\beta\gamma$
is \emph{compatible} with $Q$, $C$, $S$ and $y_1,\ldots,y_m$ if
\begin{compactenum}
  \item $\beta=p_1$ if $e_1\in S$, otherwise $p_1$ is in the interior
  of the edge $\alpha\beta$; and
  \item $\gamma=p_m$ if $e_m\in S$, otherwise $p_m$ is in the interior
  of the edge $\alpha\gamma$.
\end{compactenum}

\begin{thm}\thmlabel{quad2}
   Let $Q$, $C$, $e_1,\ldots,e_m$, $y_1,\ldots,y_m$ be as in
  \thmref{quad}, let $S$ be an independent set of split edges in $Q$,
  and let $\Delta$ be a triangle compatible with $Q$, $C$, $S$, and
  $y_1,\ldots,y_m$.  Then $Q$ has a straight-line drawing in which, for
  each $i\in\{1,\ldots,m\}$, the intersection of $e_i$ with the y-axis is
  \begin{compactenum}
     \item $(0,y_i)$ if $e_i$ does not share a vertex with any edge in $S$; or
     \item $(0,y_j)$ if $e_i$ shares an endpoint with $e_j\in S$.
  \end{compactenum}
\end{thm}

\begin{proof}
  This proof is another continuity argument. For any $\epsilon \ge 0$, we define
  $y(\epsilon)=(y_1(\epsilon),\ldots,y_m(\epsilon))$ as follows:
  \begin{enumerate}
    \item $y_i(\epsilon)= y_i$ if $e_i$ is not incident to a split edge.
    \item For each split edge $e_s=xy$ where $x$'s incident edges
  $e_{i_1},\ldots,e_{i_d}$ cross $C$ after $e_j$, we
  set $y_{i_\ell}(\epsilon)=y_s+\epsilon\ell/d$, for each
  $\ell\in\{1,\ldots,d\}$.  Similarly, if $y$ has neighbours
  $e_{k_1},\ldots,e_{k_r}$, we set each $y_{k_\ell}(\epsilon)=y_i -
  \epsilon(k-\ell+1)/k$. 
  \end{enumerate}
  In this way, the edges incident to $x$ have $y(\epsilon)$ values evenly
  spaced in the interval $[y_s,y_s+\epsilon]$ and edges incident $y$
  have $y(\epsilon)$ values evently spaced in $[y_s-\epsilon,y_s]$.

  For all sufficiently small $\epsilon >0$, \thmref{quad} ensures that
  $Q$ has a straight-line drawing $Q_\epsilon$ in which $e_i$ crosses
  the y-axis at $y_i(\epsilon)$. We use $s_i(\epsilon)$ to denote the
  slope of $e_i$ in $Q_\epsilon$.  The drawing $Q_\epsilon$ changes
  continuously with $\epsilon$ so we can ask if $\lim_{\epsilon\downarrow
  0}Q_\epsilon$ exists and, if it does, does it define the straight-line 
  embedding of $Q_S$ that
  we want?  This answer to both questions is yes.  To establish this,
  we first show that there is a $\delta>0$ such $Q_\epsilon$ satisfies
  the $\delta$-strong ordering constraint for every sufficiently small
  $\epsilon >0$.

  Let $g=\min\{|y_i-y_j|: i,j\in\{1,\ldots,m\},\, i\neq j\}$ and observe
  that $g>0$ and does not depend on $\epsilon$.  There are two cases
  to consider:

  \begin{enumerate}
     \item If two edges $e_i$ and $e_j$ are incident to a common
     vertex $x$ that is not the endpoint of an edge in $S$, then
     in $Q_\epsilon$, $e_i$ crosses the y-axis at $y_i(\epsilon) =
     y_i\pm\epsilon$ and $e_j$ crosses at $y_j(\epsilon)=y_j\pm\epsilon$,
     so $|s_i(\epsilon)-s_j(\epsilon)| \ge |y_i-y_j|-2\epsilon \ge
     g-2\epsilon > g/2$ for all $\epsilon < g/4$.

    \item On the other hand, if $xy=e_s\in S$ and $e_i$ and $e_j$
    are both incident to $x$, then $|y_i(\epsilon)-y_j(\epsilon)|
    \ge \epsilon/\deg(x) > \epsilon/n$.  However, in this case,
    \corref{split-edges} ensures that the x-coordinate of $x$ is at
    most $\epsilon/g$.  But this means that
    \[
       |s_i(\epsilon)-s_j(\epsilon)|(\epsilon/g) 
            \ge |y_i(\epsilon)-y_j(\epsilon)| 
            \ge \epsilon/n \enspace .
    \]
    Rewriting this gives $|s_i(\epsilon)-s_j(\epsilon)| > g/n$.  
  \end{enumerate}
  Therefore, for all $0<\epsilon\le g/4$, $Q_\epsilon$ satisifies the
  $g/n$-strong ordering constraint.

  At this point, we are done. The same argument used to exclude events
  of Type~3 in the proof of \lemref{uniqueness} shows that, for each
  $i\in\{1,\ldots,m\}$, $\lim_{\epsilon\downarrow 0} s_i(\epsilon)$
  exists, so $s(0)=\lim_{\epsilon\downarrow 0} s(\epsilon)$ exists.
  Furthermore, for all $\epsilon$ in a right-neighbourhood of $0$,
  $Q_\epsilon$ satisfies the $\delta$-strong ordering constraints and
  therefore $s(0)$ satisfies the ordering constraints and determines
  a \Fary\ embedding $Q_0$ of $Q_S$ that fulfills the conditions of
  the theorem.
\end{proof}


%\footnote{We abuse the term \emph{drawing} here slightly, since we sometimes require that the two endpoints of a clean edge are drawn at the same location.} in which, for each $i\in\{1,\ldots,m\}$, the intersection
%    of $e_i$ with the y-axis is a single point $(0,y_i)$ and the edges
%    $e_1$ and $e_m$ are mapped to the two edges of $\Delta$ that
%    intersect the y-axis.
%\end{lem}
%
%
%\begin{proof}
%   Add artificial constraints on the slopes of edges incident to split edges (and to split edges themselves.
%\end{proof}
%
%

%
%
%
%
%We say that a 
%Next, we present a strengthening of \lemref{quad} that allows vertices
%of $Q$ to be on the curve $C$.  Let $r_1,\ldots,r_m$ be a sequence
%of vertices and edges in a planar graph and let $y_1<\cdots<y_m$ be a
%sequence of numbers.  We say that a triangle $\Delta=\alpha\beta\gamma$
%is \emph{compatible} with $r_1,\ldots,r_m$ and $y_1,\ldots,y_m$ if
%\begin{compactenum}
%  \item $\beta=p_1$ if $r_1$ is a vertex, otherwise $p_1$ in the interior
%  of the edge $\alpha\beta$; and
%  \item $\gamma=p_m$ if $r_m$ is a vertex, otherwise $p_m$ in the interior
%  of the edge $\alpha\gamma$.
%\end{compactenum}
%
%\begin{lem}\lemlabel{quad2}
%    Let
%    \begin{compactitem}
%    \item $Q$ be a quadrangulation with outer face $f$; 
%    \item $C:[0,1]\to\R^2$ be a Jordan curve
%     whose endpoint $C(0)=C(1)$ is in the interior of $f$,
%     whose intersection with each edge of $Q$
%     consists of exactly one point, and for which no vertex of $Q$ on $C$
%     has neighbours both inside and outside of $C$;
%    \item $r_1,\ldots,r_m$ be the edges and vertices of $Q$ in the
%    order they are intersected by $C$; 
%    \item $y_1<\cdots<y_m$
%    be any increasing sequence of numbers; and
%    \item $\Delta$ be a triangle that is compatible with $r_1,\ldots,r_m$ and $y_1,\ldots,y_m$.
%    \end{compactitem}
%    Then $Q$ has a straight-line
%    drawing in which, for each $i\in\{1,\ldots,m\}$, the intersection
%    of $r_i$ with the y-axis is a single point $(0,y_i)$ and three vertices
%    of $f$ are mapped to the vertices of $\Delta$
%\end{lem}
%
%\begin{proof}
%  Add artifical constraints for vertices on $C$.  If, for some reason, that approach fails, then use the argument we worked out for shifting points off of $C$ and then moving them back to the y-axis.
%%   To fix this, first note that any vertex $r_i$ on $C$ is only incident
%   only to crossing edges and therefore the neighbours of $r_i$ are all
%   contained in $R$. Thus, we can deform $C$ in the neighbourhood of $r_i$
%   so that $r_i$ moves into the interior of $L$ and $C$ intersects each
%   of $r_i$'s incident edges $e_1,\ldots,e_d$ in exactly one point.  Now,
%   to apply \lemref{quad} we must specify numbers $y_1',\ldots,y_d'$ where
%   $(0,y_i')$ is the intersection point between $e_j$ and the y-axis.
%   To do this we choose any $y_1'<\cdots<y_d'$ satisfying
%   \[
%       y_{i-1} < y_1' < y_i < y_d' < y_{i+1} \enspace .
%   \]
%   By doing this for each vertex $r_i$ in $C$ we obtain a curve $C'$ and
%   a sequence $y_1''<\cdots<y_m''$ on which we can apply \lemref{quad} to
%   find a drawing of $Q$.  Now this drawing of $Q$ does not yet satisfy
%   the requirements of the theorem because there are vertices $r_i\in
%   C$ that are not in $C'$. However, the choice of $y_1'<\cdots<y_d'$
%   described above ensures that moving the vertex $r_i$ to $(0,y_i)$
%   does not introduce any crossings and gives a drawing of $Q$ that
%   satisfies all the relevant requirements of the theorem.  Finally,
%   in the process of building $Q$ described above, we showed how the
%   drawing of $Q$ can be extended to a drawing of $T$ that satisifies
%   all the requirements of the theorem.
%\end{proof}



%\begin{lem}
%   Let $Q$ be a straight-line drawing of a quadrangulation each of whose edges intersect the
%   y-axis in exactly one point.  Let $v$ be a vertex of $Q$ that is not on
%   the y-axis, whose incident edges $vx_1,\ldots,vx_d$ intersect
%   the y-axis at y-coordinates $y_1'<\cdots<y_d'$, respectively, 
%   and suppose that no
%   other edges of $Q$ intersect the y-axis with y-coordinates in the interval 
%   $[y_1',y_d']$.
%   Then there is a unique index $i\in\{1,\ldots,d\}$ such that
%   \begin{enumerate}
%     \item for every $j\in\{1,\ldots,i-1\}$, all edges incident to $x_j$, aside from $vx_j$ intersect the y-axis below $(0,y_1')$;
%     \item for every $j\in\{i+1,\ldots,d\}$, all edges incident to $x_j$, aside from $vx_j$ intersect the y-axis above $(0,y_d')$; and
%     \item the embedding of $Q$ obtained by moving $v$ to $(0,y_i')$ is a straight-line drawing.
%   \end{enumerate}
%\end{lem}
%
%\begin{proof}
%   For the first two points, suppose on the contrary that there is some
%   pair of indices $k < \ell$ such that $x_k$ is incident to an edge
%   $wx_k$ that crosses the y-axis at some point $p$ above $(0,y_d')$
%   and $x_\ell$ is incident to an edge $ux_\ell$ that crosses the y-axis
%   at some point $q$ below $(0,y_1')$. Consider the triangle $abc$ with
%   $a=(0,y_k')$, $b=x_k$, and $c=p$.  Then the path $v,x_\ell,u$ enters
%   the interior of $abc$ through the segment $ac$ and exits through one of
%   the other two segments.  But this is a contradiction to the assumption
%   that $Q$ is a straight-line drawing, since it implies that one of the
%   edges in this path crosses at least one of the edges $vx_k$ or $wx_k$.
%
%   For the third point, observe 
%\end{proof}

\section{Triangulations}
\seclabel{triangulations}

In this section we prove that every collinear set is free.
We will sometimes make use of this simple fact:
\begin{obs}\obslabel{quad}
  If $q=abcd$ is a simple quadrilateral, then neither of the segments $ac$
  or $bd$ cross any of the edges of $q$.
\end{obs}

\begin{thm}\thmlabel{main}
   Let
   \begin{compactenum}
     \item  $T$ be a triangulation with outer face $f$;
     \item  $C:[0,1]\to\R^2$ be a nice clean Jordan curve for $T$;
     \item $r_1,\ldots,r_k$ be the sequence of vertices and open edges
           of $T$ that are intersected by $C$, and ordered in the order
           that they are intersected by $C$;
     \item $y_1<\cdots<y_k$ be any sequence of numbers; and
     \item $\Delta$ be a triangle that is compatible with 
           $r_1,\ldots,r_m$ and $y_1,\ldots,y_m$.
  \end{compactenum}
   Then, for any $\epsilon>0$, $T$ has a
   \Fary\ embedding in which the outer face $f$ is $\Delta$
   and, for each $i\in\{1,\ldots,k\}$, 
   \begin{compactenum}
       \item $r_i$ is drawn on the y-axis, with y-coordinate $y_i$
         if $r_i$ is a vertex; or
       \item (if $r_i$ is an edge) the intersection of $r_i$ with the
         y-axis has a y-coordinate in the interval
         $[y_i-\epsilon,y_i+\epsilon]$.
   \end{compactenum}
\end{thm}

\begin{proof}
   We call $y_i$ the (desired) \emph{crossing coordinate} for $r_i$. If
   a \Fary\ embedding contains an edge whose intersection with the
   y-axis is $\{(0,y)\}$ or a vertex at $(0,y)$, we say that the edge
   or vertex \emph{crosses the y-axis at $y$}.

   Let $L=C^-$, $R=C^+$.  We say that an edge of $T$ is a \emph{crossing
   edge} if its intersection with $C$ is non-empty.  A crossing edge is
   a \emph{proper crossing edge} if its intersection with each of $L$
   and $R$ is non-empty.  We say that points in $L$ are \emph{to the
   left of $C$} and points in $R$ are \emph{to the right of $C$}.

%   We prove an extension of the theorem to the case where $T$ is an
%   non-crossing embedded graph whose faces consist of triangles (3-cycles)
%   and quadrilateral (4-cycles) with the resriction that, for every
%   quadrilateral face $q$, all four edges of $q$ are crossing edges.
%   The proof is by induction on the number of non-crossing edges plus the
%   number of vertices of $T$.

%   \paragraph{Base Cases:}
%   There are three base cases tht we handle explicitly.  If $T$ contains
%   2 or fewer crossing edges, If $T$ is the complete graph, $K_4$ on 4
%   vertices, but only has only three crossing edges, then the theorem is
%   also easy to prove directly.  The last base case occurs when all edges
%   of $T$ are crossing edges.  In this case $T$ is bipartite and therefore
%   all its faces are quadrilaterals, so $T$ is a quadrilateralization.
%   This case is handled directly by \lemref{quad2}.
%
%   Thus we may assume that $T$ has at least one non-crossing edge and
%   at least 2 crossing edges.  

   The proof is by induction on $n+m$, where $n$ is the number of
   vertices of $T$ and $m$ is the number of crossing edges.  We begin
   by describing reductions that allow use to apply the inductive
   hypothesis. When none of these reductions are possible, we arrive
   at our base case. To handle this base case we argue that $T$ has a
   sufficiently simple structure that it can be handled by \thmref{quad2}.

   \paragraph{Separating Triangles.}
   (See \figref{separating}.)
   If $T$ contains a separating triangle
   $xyz$ then we remove all vertices from the interior of $xyz$ to obtain
   a graph $T^+$ in which $xyz$ is a face.  Since the intersection of $C$
   with each of $xy$, $yz$ and $zx$ consists of at most a single point,
   the vertices and edges of $T$ intersected by $C$ that are not in $T^+$
   appear as a contiguous subsequence $r_i,\ldots,r_j$.

   \begin{figure}
      \centering{\includegraphics{figs/separating}}
      \caption{Recursing on separating triangles in the proof of
      \thmref{main}}
      \figlabel{separating}
   \end{figure}

   Observe that each of $r_{i-1}$ and $r_{j+1}$ is either an edge
   or vertex of the triangle $xyz$.  Set $\epsilon'$ to be any
   value less than $\min\{\epsilon,y_{i}-y_{i-1}, y_{j+1}-y_j\}$.
   and apply induction on $T^+$ using the value $\epsilon'$
   and the sequences $r_1,\ldots,r_{i-1},r_{j+1},\ldots,r_k$ and
   $y_1,\ldots,y_{i-1},y_{j+1},\ldots,y_k$ to obtain an embedding
   of $T^+$.  In the resulting embedding $xyz$ becomes a triangular face
   $\Delta'$.

   In the resulting embedding, Let $y_{i-1}'$ and $y_{j+1}'$
   be the respective y-coordinates of the intersections of
   $r_{i-1}$ and $r_{j+1}$ with the y-axis.  By our choice of
   $\epsilon'$, $y_{i-1}'<y_i<\cdots<y_j<y_{j+1}'$.  Observe that
   $\Delta'$ is compatible with $r_{i-1},\ldots,r_{j+1}$ and
   $y_{i-1}',y_i,\ldots,y_j,y_{j+1}'$.

   Let $T^-$ be the graph obtained by removing, from $T$, all
   vertices outside of $xyz$.  Now we apply induction on $T^-$ using
   the triangle $\Delta'$ and the sequences $r_{i-1},\ldots,r_{j+1}$ and
   $y_{i-1}',y_i,\ldots,y_{j},y_{j+1}'$.  Combining the embeddings of $T^+$
   and $T^-$ yields an embedding of $T$ that satisfies the requirements of
   the theorem.  Thus, we may assume that $T$ has no separating triangles.

   \paragraph{Contractible Edges:}
   (See \figref{contractible}.)
   We say that a triangular face of $T$ is a \emph{proper crossing
   face} if it is incident to two proper crossing edges.  We say that a
   non-crossing edge of $T$ is \emph{contractible} it is not contained
   in the boundary of any crossing face.  
   \begin{figure}
      \centering{\includegraphics{figs/contractible}}
      \caption{Contracting and uncontracting edges in the proof of
      \thmref{main}}
      \figlabel{contractible}
   \end{figure}

   If $T$ contains a contractible edge $xy$ then we contract $xy$ to
   obtain a new vertex $v$ in a smaller graph $T'$.   We can then apply
   induction on $T'$ with the value $\epsilon'=\epsilon/2$ to obtain an
   embedding of $T'$ that satisfies all the conditions of the theorem under
   the stronger condition that each proper crossing edge $e_i$ crosses
   the y-axis in the interval $[y_i-\epsilon/2,y_i+\epsilon/2]$.

   To obtain an embedding of $T$ we uncontract $v$ by placing $x$ and $y$
   within a ball of radius $\epsilon/2$ centered at $v$. (That such
   a placement is always possible is a standard argument.)  Since the
   distance between $y$ and $v$ and $x$ and $v$ is at most $\epsilon/2$,
   each proper crossing edge $r_i$ incident on $x$ or $y$ will cross
   the y-axis in the interval $[y_i-\epsilon,y_i+\epsilon]$, as required.
   Thus we may assume that $T$ has no separating triangles of contractible
   edges.

%   \paragraph{Eraseable edges}
%   We say that a non-crossing edge of $xy$ of $T$ is \emph{eraseable}
%   if neither of its endpoints is on $C$ and both its incident faces
%   intersect $C$.  If $T$ contains an eraseable edge $xy$, then we remove
%   the edge $xy$ from $T$ to obtain smaller graph $T'$ on which we can
%   apply induction. In the resulting drawing of $T'$, $x$ and $y$ lie on
%   a common face (which may be the outer face of $T'$) and are visible.
%   We can therefore add the edge $xy$ to obtain the desired drawing
%   of $T$.

   \paragraph{Flippable edges.}
   (See \figref{flippable}.)
   We say that a non-crossing edge $xy$ of $T$ is \emph{flippable} if there
   exists distinct vertices $z$, $a$, $b$, and $c$, such that 
   \begin{compactenum}
      \item $xyb$, $zyc$, $xza$ are crossing faces of $T$;
      \item $xyz$ is a non-crossing face of $T$; and (
      \item $C$ intersects $za$, $xa$, $xb$, $yb$, and $yc$ in this order; or 
      \item or $C$ intersects $xa$, $xb$, $yc$, $zc$, $za$, in this order).  
   \end{compactenum}
   \begin{figure}
      \centering{\includegraphics{figs/flippable}}
      \caption{Flipping edges in the proof of
      \thmref{main}}
      \figlabel{flippable}
   \end{figure}

   If $T$ contains the flippable edge $xy$ then we remove $xy$ and replace
   it with $zc$ to obtain a new graph $T'$.  Note that, since $T$ has
   no separating triangles, the edge $zc$ is not already present in $T$.
   After choosing a crossing coordinate for $zc$ somewhere between those
   of $xc$ and $yc$ we can then inductively embed $T'$.

   We claim that in the resulting embedding of $T'$, the only open edge
   that intersects the open segment $xy$ is $zc$.  In particular, we must
   ensure that $z$ is not a reflex vertex in the quadrilateral $xcyz$.
   To show this we distinguish between the two possible cases (3 and 4)
   in the definition of flippable edges. In Case~3, The existence of the
   edges $za$ and $zb$ ensure that, in the resulting embedding of $T'$,
   $xcyz$ is convex.  In Case~4, the triangle $zxa$ is convex and $xcyz$
   is contained in this triangle, therefore $z$ is a convex vertex
   of $xcyz$.

   In either case, removing $zc$ from the embedding of $T'$ and replacing
   it with $xy$ yields the desired embedding of $T$.

   \paragraph{The Base Case.}

   Finally, we are left with a situation in which $T$ is a triangulation
   with no separating triangles, no contractible edges, and no flippable
   edges.  If $T$ is the complete graph on $K_3$ or $K_4$ on three or
   four vertices, then the theorem is trivial, so we may assume that $T$
   has at least 5 vertices.

   We claim that every non-crossing edge $xy$ of $T$ is contained in the
   boundary of two crossing faces $xya$ and $yxb$.  To see why this is so,
   observe that if some non-crossing edge $xy$ is not contractible then
   one of $xy$'s incident faces, $yxc$ is proper crossing.  Suppose, for
   the sake of contradiction, that the other face $xyz$, incident on $xy$
   is not crossing.  Since neither $zx$ nor $yz$ is contractible, they
   must be incident on crossing triangles $xza$ and $zyb$, respectively.
   Since $T$ contains no separating triangles, we know that $b\neq a$,
   otherwise $xya$ would separate $z$ from $c$.

   This leaves us in the situation in which we have distinct vertices $x$,
   $y$, $z$, $a$, $b$, $c$, such that $xyb$, $zyc$, $xza$ are crossing
   faces of $T$ and $xyz$ is a non-crossing face of $T$.  Checking the
   definition of flippable edge then ensures that at least one of $xy$,
   $yz$, or $za$ is a flippable edge.

   Thus, every non-crossing edge of $T$ is incident to two crossing
   triangles.  The union of these two triangles is a quadrilateral
   consisting of four crossing edges.  Let $Q^*$ denote the graph obtained
   by removing all non-crossing edges from $T$.  The faces of $Q^*$
   are quadrilaterals, each having four crossing edges or triangles each
   having three crossing edges.

   The only triangles with three crossing edges are those triangles
   having one vertex in each of $C$, $L$ and $R$.  Now, consider any
   vertex $v=r_i$ on $C$.  The graph $Q^*$ has two triangular faces $vab$
   and $vcd$ incident on $v$ such that $ab=r_{i-1}$ and $cd=r_{i+1}$.
   Split $v$ into two vertices $x\in L$ and $y\in R$ joined by the edge
   $xy$, make $x$ adjacent to all neighbours of $v$ in $R$, and make
   $y$ adjacent to all neighbours of $v$ in $L$. See \figref{split}.
   This splitting operation eliminates the triangular faces $vab$ and
   $vcd$ and introduces the quadrangular faces $xyab$ and $yxcd$.

   \begin{figure}
      \centering{
        \begin{tabular}{c|c}
            \includegraphics{figs/split} & \includegraphics{figs/split-outer}
        \end{tabular}}
      \caption{Splitting vertices on $C$ in the proof of
      \thmref{main}.}
      \figlabel{split}
   \end{figure}

   The fact that $C$ intersects each edge of $Q$ in at most one point
   implies that no two vertices on $C$ are adjacent.
   Therefore, this splitting
   operation can be done on every vertex of $T$ that is on $C$ to obtain
   a quadrangulation $Q$ in which every edge properly crosses $C$,
   as well as an independent set $S$ of split edges in $Q$.  \thmref{quad2}
   then provides an embedding of $Q_S=Q^*$ that satisfies all the conditions of
   the theorem.  Reinserting the crossing edges in the resulting
   embedding provides the desired embedding of $T$.
%
%
%   Either these triangles form a separating triangle that separates $z$
%   from $c$ or the existence of the edges $za$ and $zb$ establishes that
%   $xy$ is flippable.
%
%
%   We claim that $T$ has no non-crossing edges except those in $L$
%   that have one endpoint on $C$.  To see why this is so, observe that
%   if some non-crossing edge $xy$ is not contractible then one of $x$
%   or $y$ is on $C$ or one of $xy$'s incident faces, $yxc$ is crossing.
%   Since $xy$ is no erasable, it must be incident on a non-crossing
%   face $xyz$.  Since neither $zx$ nor $yz$ is contractible, they must
%   be incident on crossing triangles $xza$ and $xzb$, respectively.
%   Either these triangles form a separating triangle that separates $z$
%   from $c$ or the existence of the edges $za$ adn $zb$ establishes that
%   $xy$ is flippable.
%
%   Any cycle in $T$ that uses only crossing edges necessarily has even
%   length.  Therefore, the only triangular faces of $T$ include at least
%   one vertex on $C$.  But this implies that every triangular face $T$
%   includes one vertex from each of $C$, $L$, and $R$, otherwise $C$
%   includes a non-crossing edge neither of whose vertices is on $C$.
%
%   TODO: Mention conditions on input having exactly two triangular faces
%   for each vertex on $C$.
%   None of the operations used to arrive at this base case remove
%   crossing edges or edges incident to $C$.  Thus, every vertex on $C$
%   is incident to exactly two triangles in $T$ and these are the only
%   triangles in $T$.  For every vertex $v=r_i$ on $C$, we split $v$
%   into two vertices $x$ and $y$ joined by an edge $xy$ and make $x$
%   is adjacent to $v$'s neighbours in $L$ and $y$ adjacent to $v$'s
%   neighbours in $R$.  We embed $x$ and $y$ in the neighbourhood of $v$
%   in such a way that $C$ crosses every edge incident to $x$ and $y$.
%   We specify crossing coordinates for all of these crossings that have
%   a value very close to that of $y_i$.
%
%   Observe that, after we do this for each every vertex on $C$, the
%   resulting graph is a quadrangulation.  Now apply the theorem above....
%
%
%   \paragraph{Reductions}
%   Next we consider what can be done when $T$ has no separating
%   triangles and no contractible edges.  Note that this implies that
%   every vertex $v$ of $T$ is incident to at least one crossing edge
%   since, otherwise every edge incident to $v$ is contractible.
%  
%   Consider the graph $H$ consisting of only the non-crossing edges of $T$
%   and their endpoints.  Some face $g$ of $H$ contains $C$.  We claim
%   that $H$ is outerplanar because every vertex of $H$ is incident to
%   at least one crossing edge and is therefore on the boundary of $g$.
%
%   Next, we claim that $H$ has only three kinds of 2-connected components:
%   (i)~single edges, (ii)~3-cycles, and (iii)~two three cycles sharing an edge.
%
%   To see why this is so, suppose that some 2-connected component $K$ of $H$ has $k\ge 5$ vertices.  Since $T$ is triangulated, $K$ is a triangulated $k$-gon.  But $ and by repeatedly removing vertices of degree 
%
%
% this component therefore has $  Then
%   To see why this is so, observe that every edge $xy$ of $H$ is not contractible so either:
%   \begin{enumerate}
%       \item $xy$ bounds a non-proper
%
%   \end{enumerate}
%
%
%
%
%   Furthermore, every edge $e$ of $H$ is on the boundary of $g$ since,
%   otherwise, neither of the two triangles incident on $e$ would be proper
%   crossing triangles and $e$ would be contractible.  This means that
%   $H$ is a \emph{cactus graph}---a graph in which each edge is incident
%   to at most one cycle.  Furthermore, each cycle of $H$ is a 3-cycle.
%   To see why, observe that, since $T$ is a triangulation, if $H$
%   contained a $k$-cycle for $k\ge 4$, this cycle would have a chord.
%
%   Because $T$ has at leat one non-crossing edge, $H$ contains at least one
%   edge.  Therefore, $H$ has at least one 2-connected component, 
%   which is either a 3-cycle
%   or a single edge and assume, without loss of generality that this
%   2-connected component is contained in $\bar L$.  
%   There are two cases to consider:
%   \begin{enumerate}
%	\item The 2-connected component is a single edge $xy$.	Let $xya$
%	and $yxb$ be the two faces of $T$ incident to $xy$.  Note that
%	$a$ and $b$ are both in $R$ since, otherwise, the 2-connected
%	component containing $xy$ would be a 3-cycle.
%	Therefore, removing the edge $xy$ from $T$ creates a
%	quadrangular face $xbya$ consisting of only crossing
%	edges. \obsref{quad} ensures that, after applying induction
%	on this smaller graph, the edge $xy$ can be added without
%	introducing crossings.  (It is worth noting that the analysis
%	of this case remains true even if $xy$ is an edge of the
%	outer face.)
%        \begin{figure}
%           \centering{\includegraphics{figs/1b}}
%           \caption{When a 2-connected component of $H$ is an edge. The second figure illustrates the situation when $xy$ is an edge of the outer face.}
%            \figlabel{1b}
%        \end{figure}
%        \item The component is a face $xyz$ of $T$.  Since none of $xy$,
%        $yz$, or $zx$ was contracted, there exists vertices $a,b,c\in R$
%        (possibly not distinct) such that $xya$, $yzb$ and $zxc$ are
%        triangular faces of $T$.\footnote{Note that this is even true if
%        $xy$ was not contracted because $x,y\in C$. In that case there
%        is still an $a\in R$.} 
%
%        We claim that $a$, $b$, and $c$ are distinct vertices. If not,
%        this means that $xyz$ all have a common neighbour, say $a$. Thus
%        $T$ contains the complete graph $K_4$ as a subgraph.  The case in
%        which $T$ is $K_4$ was handled as a base case in our induction,
%        therefore $T$ contains some vertex $v\not\in\{x,y,z,a\}$. But
%        contradicts the assumption that $T$ contains no separating
%        triangles, since $v$ must be in one of the four triangular faces
%        of this $K_4$ subgraph, say $xyz$, but then $xyz$ is a separating
%        triangle that separates $v$ from $a$.
%
%        Thus, $a$, $b$, and $c$ are three distinct vertices of $T$
%        (see \figref{2c}). Note that, in this case, $T$ does not
%        contain the edge $yc$ since, if it did, then $yxc$ would be
%        a separating triangle that separates $a$ from $b$.
%        \begin{figure}
%           \centering{\includegraphics{figs/2b}}
%           \caption{When a 2-connected component of $H$ is a triangle $xyz$ whose three vertices have two common neighbours $a,c\in R$.}
%           \figlabel{2c}
%        \end{figure}
% 
%        There are two cases to consider: 
%
%        If none of $xy$, $yz$, or $zx$ is an edge of the outer face,
%        then (after appropriate relabelling) $C$ intersects the edges
%        $yb$, $zb$, $zc$, $xc$, $xa$, $ya$ in this order.  In this case,
%        we remove the edges $xy$, $yz$ and $zx$ and add the edge $yc$.
%        At the same time we specify a crossing location of the edge
%        $yc$ on the y-axis anywhere in the interior of the segment
%        whose ends points are the crossing locations of $xc$ and $zc$.
%        This creates two quadrangular faces $ycxa$ and $ybzc$.  After
%        inductively drawing this smaller graph, the vertices $x$ and $z$
%        are reflex vertices of these two faces.  Since every non-convex
%        quadrilateral is star-shaped and its kernel contains its reflex
%        vertex, we can remove the edge $yc$ and add the edges $xy$, $yz$
%        adn $zx$ without introducing crossings.
%
%        If $xy$ is on the outer face, then $C$ intersects the edges
%        $ya$, $yb$, $zb$, $zc$, $xc$, $xa$ in this order. In this
%        case we perform exactly the same operation:  we remove the
%        edges $xy$, $yz$ and $zx$ and add the edge $yc$ and specify a
%        crossing location of the edge $yc$ on the y-axis anywhere in
%        the interior of the segment whose ends points are the crossing
%        locations of $xc$ and $zc$.  Again, this creates two quadrangular
%        faces $ycxa$ and $ybzc$ only now $ycxa$ is the outer face.
%        After inductively drawing this smaller graph, the vertices $z$
%        is a reflex vertices in $ybzc$ and the vertex $x$ sees the entire
%        edge $cy$.  This ensure that we can remove the edge $yc$ and
%        add the edges $xy$, $yz$ and $zx$ without introducing crossings.
%   \end{enumerate}
%   This completes the proof.
\end{proof}

\begin{cor}
  Every collinear set is free.
\end{cor}

\begin{proof}
   Given an embedded graph $G$, a collinear set $S$, $|S|=k$ in $G$ and
   any $y_1<\cdots<y_k$, we need to show that $G$ has a \Fary\ embedding
   in which the vertices of $S$ are placed at $(0,y_1),\ldots,(0,y_k)$.
   \thmref{dujmovic-frati} implies that there exists a curve $C$ that is
   nice for $G$ and that contains all the vertices of $S$ in some order,
   say $v_1,\ldots,v_k$.  That $S$ is free follows almost immediately form
   \thmref{main}.  The only caveat is that \thmref{main} requires that
   $C$ is clean, which would imply that $S$ is an independent set in $G$.

   To get around this, we first augment $G$ by adding edges to obtain a
   triangulation $T^*$. If two vertices $x$ and $y$ of $S$ are adjacent
   in $T^*$, then the edge $xy$ is incident to triangles $xya$ and
   $yxb$.  We \emph{subdivide} the edge $xy$ by removing it and adding
   a \emph{subdivision vertex} $c$ and the edges $ca$, $cx$, $cb$, $cy$.
   We place $c$ so that it is not on $C$ and embed the edges incident to
   $c$ so that they only intersect $C$ at $x$ and $y$.  By doing this
   for each edge of $T^*$ that joins two vertices of $S$, we obtain a
   triangulation $T$ in which $S$ is an independent set.  We then apply
   \thmref{main} to $T$, $C$, the sequence $r_1,\ldots,r_m$ of edges
   and vertices intersected by $C$, any sequence $y_1'<\cdots<y_m'$
   in which $y_j'=y_i$ if $r_j=v_i$, and an appropriate triangle $\Delta$.

   This gives a \Fary\ mebedding of $T$ in which $v_i$ is at $(0,y_i)$
   for each $i\in\{1,\ldots,k\}$.  Removing subdivision vertices and
   reinserting their corresponding edges gives an embedding of $T^*$.
   Removing the edges in $T^*$ that are not in $G$ then gives a
   \Fary\ embedding of $T$ in which $v_i$ is at $(0,y_i)$ for each
   $i\in\{1,\ldots,k\}$.
\end{proof}


\bibliographystyle{plainurl}
\bibliography{freecoll}


\end{document}


