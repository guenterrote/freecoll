\documentclass{patmorin}
\usepackage[utf8]{inputenc}
\usepackage{amsthm,amsmath,graphicx}
\usepackage{array}
\usepackage{pat}
\usepackage{hyperref}
\usepackage[dvipsnames]{xcolor}
\definecolor{linkblue}{named}{Blue}
\hypersetup{colorlinks=true, linkcolor=linkblue,  anchorcolor=linkblue,
citecolor=linkblue, filecolor=linkblue, menucolor=linkblue,
urlcolor=linkblue, pdfcreator=Me, pdfproducer=Me} \setlength{\parskip}{1ex}
\usepackage{tikz}

\listfiles
\newcommand{\lstlabel}[1]{\label{lst:#1}}
\newcommand{\lstref}[1]{Listing~\ref{lst:#1}}
\newcommand{\Lstref}[1]{\lstref{#1}}

\DeclareMathOperator{\block}{block}
\newcommand{\naive}{na\"{\i}ve}


\newcommand{\reals}{\mathbb{R}}
\newcommand{\integers}{\mathbb{Z}}
\newcommand{\naturals}{\mathbb{N}}
\newcommand{\dist}{{d}}

\title{\MakeUppercase{A Lemma on Order Types with Applications to Collinear Sets in Graph Drawing}\thanks{This research is partially funded by NSERC and the Ontario Ministry of Research and Innovation.}}

\author{Vida Dujmovi\'c\thanks{Department of Computer Science and Electrical Engineering, University of Ottawa}\, and Pat Morin\thanks{School of Computer Science, Carleton University}}

\begin{document}
\maketitle


\begin{abstract}
  We define full order types and $k$ full order types and use them to
  show that any collinear set of vertices in a plane drawing of a graph
  $G$ is also a free collinear set (Dujmovi\'c and Frati 2016) of $G$.
\end{abstract}


\section{The Whole Thing}

For two distinct points $p$ and $q$,
$\overline{pq}$ denotes the line segment with endpoints $p$ and
$q$, $\overrightarrow{pq}$ denotes the ray originating at $p$ and
containing $q$, $\overleftarrow{pq}\equiv \overrightarrow{qp}$, 
$p\!\overrightarrow{q} = \overrightarrow{pq}\setminus \overline{pq}$, 
and $\overleftarrow{p}\!q\equiv q\overrightarrow{p}$.

Let $P=p_1,\ldots,p_n$ be a sequence of points in the plane (possibly
with repetitions).  The \emph{full order type} of $P$ is a function
$f_P\colon \{1,\ldots,n\}^3\to \{0,1,\ldots,8\}$ defined as
\[
   f_P(i,j,k) = 
   \begin{cases}
      0 & \text{if $p_i=p_j=p_k$} \\
      1 & \text{if $p_i=p_j\neq p_k$} \\
      2 & \text{if $p_i=p_k\neq p_k$} \\
      3 & \text{if $p_j=p_k\neq p_i$} \\
      4 & \text{if $p_i,p_j,p_k$ distinct and collinear with $p_k\in\overline{p_ip_j}$} \\
      5 & \text{if $p_i,p_j,p_k$ distinct and collinear with $p_k\in p_i\!\overrightarrow{p_j}$} \\
      6 & \text{if $p_i,p_j,p_k$ distinct and collinear with $p_k\in\overleftarrow{p_i}\!p_j$} \\
      7 & \text{if $p_i,p_j,p_k$ are oriented counterclockwise} \\
      8 & \text{if $p_i,p_j,p_k$ are oriented clockwise} \\
   \end{cases}
\]

For a point sequence $P=p_1,\ldots,p_n$, we define sequence
of point sequences $\langle P_k:i\in \N\rangle$ inductively, as
follows.  For the base case, $P_1=P$. In the general case, $P_{k}$
is obtained from $P_{k-1}$ by considering the sequence of at most
$\binom{|P_{k-1}|}{2}$ lines $L=\ell_1,\ldots,\ell_{n_k}$ determined by
pairs of distinct points in $P_{k-1}$, ordered in some canonical way.
Using the point/line duality $\phi$ that takes the line $\ell=\{(x,y):
y=ax+b\}$ onto the point $(a,b)$, these lines determine a sequence of
dual points $\phi(L)=\phi(\ell_1),\ldots,\phi(\ell_{n_i})$.  We define
$P_{k+1}=P_k,\phi(L)$ as the concatentation of the sequences $P_k$
and $\phi(L)$.

Finally, we define the \emph{$k$ full order type}, $f^k_P$ of the
point set $P$ as $f_{P_k}$.  We use the following lemma to establish the
equivalence between collinear sets and free collinear sets.

\begin{lem}\lemlabel{main}
   Let $P=p_1,\ldots,p_n$ and $Q=q_1,\ldots,q_n$ be two point
   sequences such that  $f^k_P=f^{k}_Q$.  Then, for every point
   $p\in \R^2$ that is not collinear with any two points in $P_k$,
   there exists a point $q\in\R^2$ such that $f^{k-1}_{p_1,\ldots,p_n,p} =
   f^{k-1}_{q_1,\ldots,q_n,q}$.
\end{lem}

Before proving \lemref{main}, we first point out that it can not be
strengthened by replacing $f^{k-1}$ with $f^k$. Indeed, the two point
sets in \figref{counterexample} have the same full order type but if
we place a point $p$ in the triangle bounded by the lines in the left
figure, there is no corresponding location to place $q$ in the right 
figure.

\begin{proof}[Proof of \lemref{main}]
   Consider the arrangement, $A_P$, of the set, $L_P$, of lines formed
   by pairs of points in $P_{k-1}$. (These are the lines whose dual
   points define $P_k\setminus P_{k-1}$.)  The combinatorial structure
   of $A_P$---in particular, the set of faces in $A_P$---is completely
   determined by the full order type of $P_k$.

   By assumption, the point $p$ is not contained in any line in
   $L_P$, so $p$ is in the interior of some face, $F_p$, of $A_P$.
   The exact location $p$ in $F_p$ does not matter: For any point
   $p'$, in the interior of $F_p$, $f^{k-1}_{p_1,\ldots,p_n,p} =
   f^{k-1}_{p_1,\ldots,p_n,p'}$.  This last observation is due to the fact
   that the (left-of/right-of) relationship between a (directed) line in
   $L$ and   a point $p'$ is determined by the orientation---clockwise
   or counterclockwise---of $a,b,p'$, where $a$ and $b$ are two points
   in $P_{i-1}$.

   Since the combinatorial structure of $A_P$ is completely determined
   by the full order type, $f^{k}_P$ of $P_k$, and $f^k_{P}=f^k_Q$, it
   follows that the arrangement $A_Q$ of the set of lines $L_Q$ determined
   by pairs of points in $Q_{k-1}$ is combinatorially equivalent to $A_P$.
   In particular, $A_Q$ contains a face $F_Q$ that is equivalent to $F_P$.
   Taking any point $q$ in the interior of $F_Q$ produces a point that
   satisfies the conditions of the lemma.
\end{proof}

\begin{lem}\lemlabel{free}
   Let $G$ be a planar graph.  Then any collinear set in $G$ is a free
   collinear set in $G$.
\end{lem}

\begin{proof}
   Suppose $G$ has a plane drawing with its vertex coordinates at position
   $P=p_1,\ldots,p_n$ so that vertices $p_1,\ldots,p_r$ appear, in order,
   on the $x$-axis.  Therefore, $G$ has a collinear set of size $r$.
   Without loss of generality, we may assume that the $x$-coordinate of
   $p_1$ is greater than 0.

   For a sufficiently small $\epsilon>0$, we can randomly perturb each
   of $p_{r+1},\ldots,p_n$ by a distance of at most $\epsilon$ and the
   drawing will remain a plane drawing. (The supremeum such value of
   $\epsilon$ is called the \emph{tolerance} of $G$ \cite{X}.)  After this
   perturbation, the only points on the $x$-axis are $p_1,\ldots,p_r$,
   no two points in $P$ have the same $x$-coordinate and no three
   vertices in the drawing are collinear, unless all three vertices are
   in $p_1,\ldots,p_r$.

   To prove the lemma, we must show that, for any $0<x_1<x_2<\cdots<x_r$
   there exists a plane drawing of $G$ so that $p_i$ is drawn at
   $q_i=(x_i,0)$ for each $i\in\{1,\ldots,r\}$.  We will prove
   something stronger: There exists $q_{r+1},\ldots,q_n$ such that
   the point sequence $q_1,\ldots,q_n$ has the same full order type
   as $p_1,\ldots,p_n$. Since the full order type of a point set
   is sufficient to test if two segments cross and to identify, and
   determine the ordering of, collinear points, the full order type
   contains enough information to determine if a graph drawing is plane.

   We first observe that, for any $k\in\N$,
   $f^{k}_{p_1,\ldots,p_r}=f^{k}_{q_1,\ldots,q_r}$.  Indeed,
   this is easily verified for the case $k=1$ and, for $k>1$,
   the only additional point that appears is the dual of
   the $x$-axis, which is the origin, $(0,0)$.  In particular,
   $f^{k}_{p_1,\ldots,p_r}=f^{k}_{q_1,\ldots,q_r}$ for $k=n-r+1$.
   Now, by the random perturbation, $p_{r+1}$ satisfies the
   conditions of \lemref{main}, so we can find a point $q_{r+1}$ so
   that $f^{k-1}_{q_1,\ldots,q_{r+1}}=f^{k-1}_{p_1,\ldots,p_{r+1}}$.
   In the same way, for each of $i=2,\ldots,n-r$ we can find $q_i$ so
   that $f^{k-i}_{q_1,\ldots,q_{r+i}} = f^{k-i}_{p_1,\ldots,p_{r+i}}$.
   In particular, the end result is the point sequence $q_1,\ldots,q_n$
   such that $f^{1}(q_1,\ldots,q_n)=f^{1}(p_1,\ldots,p_n)$, i.e.,
   $q_1,\ldots,q_n$ has the same full order type as $p_1,\ldots,p_n$.
\end{proof}


\end{document}


