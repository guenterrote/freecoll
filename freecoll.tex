\documentclass{patmorin}
\usepackage[utf8]{inputenc}
\usepackage{amsthm,amsmath,graphicx}
\usepackage{array}
\usepackage{pat}
\usepackage{hyperref}
\usepackage[dvipsnames]{xcolor}
\definecolor{linkblue}{named}{Blue}
\hypersetup{colorlinks=true, linkcolor=linkblue,  anchorcolor=linkblue,
citecolor=linkblue, filecolor=linkblue, menucolor=linkblue,
urlcolor=linkblue, pdfcreator=Me, pdfproducer=Me} \setlength{\parskip}{1ex}
\usepackage{tikz}

\listfiles
\newcommand{\lstlabel}[1]{\label{lst:#1}}
\newcommand{\lstref}[1]{Listing~\ref{lst:#1}}
\newcommand{\Lstref}[1]{\lstref{#1}}

\DeclareMathOperator{\block}{block}
\newcommand{\naive}{na\"{\i}ve}


\newcommand{\reals}{\mathbb{R}}
\newcommand{\integers}{\mathbb{Z}}
\newcommand{\naturals}{\mathbb{N}}
\newcommand{\dist}{{d}}

\title{\MakeUppercase{A Lemma on Order Types with Applications to Collinear Sets in Graph Drawing}\thanks{This research is partially funded by NSERC and the Ontario Ministry of Research and Innovation.}}

\author{Vida Dujmovi\'c,\thanks{Department of Computer Science and Electrical Engineering, University of Ottawa}\, and Pat Morin\thanks{School of Computer Science, Carleton University}}

\begin{document}
\maketitle


\begin{abstract}
  We define strong order types and $k$-strong order types and use them to
  show that any collinear set of vertices in a plane drawing of a graph
  $G$ is also a free collinear set (Dujmovi\'c and Frati 2016) of $G$.
\end{abstract}


\section{The Whole Thing}

For two distinct points $p$ and $q$,
$\overline{pq}$ denotes the line segment with endpoints $p$ and
$q$, $\overrightarrow{pq}$ denotes the ray originating at $p$ and
containing $q$, $\overleftarrow{pq}\equiv \overrightarrow{qp}$, 
$p\!\overrightarrow{q} = \overrightarrow{pq}\setminus \overline{pq}$, 
and $\overleftarrow{p}\!q\equiv q\overrightarrow{p}$.

Let $P=p_1,\ldots,p_n$ be a sequence of points in the plane (possibly
with repetitions).  The \emph{full order type} of $P$ is a function
$f_P\colon \{1,\ldots,n\}^3\to \{0,1,\ldots,8\}$ defined as
\[
   f_P(i,j,k) = 
   \begin{cases}
      0 & \text{if $p_i=p_j=p_k$} \\
      1 & \text{if $p_i=p_j\neq p_k$} \\
      2 & \text{if $p_i=p_k\neq p_k$} \\
      3 & \text{if $p_j=p_k\neq p_i$} \\
      4 & \text{if $p_i,p_j,p_k$ distinct and collinear with $p_k\in\overline{p_ip_j}$} \\
      5 & \text{if $p_i,p_j,p_k$ distinct and collinear with $p_k\in p_i\!\overrightarrow{p_j}$} \\
      6 & \text{if $p_i,p_j,p_k$ distinct and collinear with $p_k\in\overleftarrow{p_i}\!p_j$} \\
      7 & \text{if $p_i,p_j,p_k$ are oriented counterclockwise} \\
      8 & \text{if $p_i,p_j,p_k$ are oriented clockwise} \\
   \end{cases}
\]
For a sequence $L=\ell_1,\ldots,\ell_n$ of lines, none of which is
vertical, we define the full order type of $f_L$ of $L$ as the full
order type of the point sequence $\phi(\ell_1),\ldots,\phi(\ell_n)$,
we where $\phi$ is the usual point/line duality that takes the line
$\ell=\{(x,y): y=ax+b\}$ onto the point $(a,b)$.

For a point sequence $P=p_1,\ldots,p_n$, we define a sequence
of point sequences $\langle P_i:i\in \N\}$ inductively,
as follows.  For the base case, $P_1=P$. In the general case,
$P_{i+1}$ is obtained from $P_{i}$ by considering the sequence of
$\binom{|P_i|}{2}$ lines $L=\ell_1,\ldots,\ell_{\binom{|P_i|}{2}}$
determined by pairs of points in $P_i$, ordered lexicographically
by index in $P_i$.  These lines determine a set of dual points
$P_{i+1}=\phi(\ell_1),\ldots,\phi(\ell_{\binom{n}{2}})$.

Finally, we define the \emph{$k$ full order type}, $f^k(P)$ of the point set $P$
as the sequence of full order types $f_{P_1},\ldots,f_{P_k}$.
We use the following lemma to establish the equivalence between collinear sets and free collinear sets.

\begin{lem}\lemlabel{main}
   Let $P=p_1,\ldots,p_n$ and $Q=q_1,\ldots,q_n$ be two point sequences
   such that  $f^k(P)=f^{k}(Q)$.  Then, for every point $p\in \R^2$,
   there exists a $q\in\R^2$ such that $f^{k-1}(p_1,\ldots,p_n,p) =
   f^{k-1}(q_1,\ldots,q_n,q)$.
\end{lem}

Before proving 

\begin{proof}
   Trivial.
\end{proof}


\end{document}


