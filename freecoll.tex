\documentclass{patmorin}
\usepackage[utf8]{inputenc}
\usepackage{amsthm,amsmath,graphicx}
\usepackage{array}
\usepackage{pat}
\usepackage{hyperref}
\usepackage[dvipsnames]{xcolor}
\definecolor{linkblue}{named}{Blue}
\hypersetup{colorlinks=true, linkcolor=linkblue,  anchorcolor=linkblue,
citecolor=linkblue, filecolor=linkblue, menucolor=linkblue,
urlcolor=linkblue, pdfcreator=Me, pdfproducer=Me} \setlength{\parskip}{1ex}
\usepackage{tikz}

\usepackage{paralist}

\DeclareMathOperator{\sgn}{sgn}


\listfiles
\newcommand{\lstlabel}[1]{\label{lst:#1}}
\newcommand{\lstref}[1]{Listing~\ref{lst:#1}}
\newcommand{\Lstref}[1]{\lstref{#1}}

\DeclareMathOperator{\block}{block}
\newcommand{\naive}{na\"{\i}ve}


\newcommand{\reals}{\mathbb{R}}
\newcommand{\integers}{\mathbb{Z}}
\newcommand{\naturals}{\mathbb{N}}
\newcommand{\dist}{{d}}

\title{\MakeUppercase{Every Collinear Set is Free}}

\author{Bellairs Workshop on Geometry and Graphs 2017--18}

\begin{document}
\maketitle


\begin{abstract}
  We show that if a planar graph $G$ has a planar straight-line drawing
  in which a subset $S$ of its vertices are collinear, then there is a
  planar straight-line drawing of $G$ in which all vertices in $S$ are
  on the $y$-axis and in which they have prescribed $y$-coordinates.
  This solves an open problem posed by Ravsky and Verbitsky in 2008.
  In their terminology, we show that every collinear set is free.
  This result has applications in graph drawing, untangling, universal
  point subsets, and related areas.
\end{abstract}


\section{Introduction}

In a planar graph, $G=(V,E)$, a \emph{collinear set} is a set of vertices
$S\subset V$ such that $G$ has a planar straight-line drawing in which
all vertices in $S$ are drawn on a single line.  A collinear set $S$
is a \emph{free collinear set} if, for any collinear set of points
$X\subset\R^2$, $|X|=|S|$, $G$ has a planar straight-line drawing in
which the vertices of $S$ are drawn on the points in $X$.  Ravsky and
Verbitsky \cite{ravsky.verbitsky:on} ask the following question:

\begin{quote}
   How far or close are parameters $\tilde{v}(G)$ and $\bar{v}(G)$? It
   seems that \emph{a priori} we even cannot exclude equality. To clarify
   this question, it would be helpful to (dis)prove that every collinear
   set in any straight line drawing is free.
\end{quote}

In the context of this quote, $\tilde{v}(G)$ and $\bar{v}(G)$ are the
respective sizes of the largest collinear set and largest free collinear
set in $G$.  In this note, we prove that $\tilde{v}(G)=\bar{v}(G)$ by
showing that every collinear set is a free collinear set.  

Dujmovi\'c and Frati gave the following characterization of collinear sets:
\begin{thm}
   A set $S$ of the vertices of a graph $G$ is a collinear set if and
   only if there exists a (topological) drawing of $G$ and a simple curve
   $C$ having both endpoints in a common face of $G$ and such that each
   vertex of $S$ is drawn on $C$ and the intersection of each edge with
   $C$ has at most one connected component.
\end{thm}
The surprising aspect of this characterization is that one can
simultaneously straighten the drawing of the graph so that it becomes a
straight line drawing and straighten $C$ so that it becomes
(say) the y-axis while preserving the combinatorial relationship between
$C$ and $G$.

\subsection{Proof Sketch}


Without loss of generality, we may assume that the the graph we are
interested in is a (topological) triangulation $T$ and the line we are
interested in is the y-axis.

%Our proof has three steps:
%\begin{enumerate}
%   \item Using combinatorial operations---edge subdivision, edge
%   contraction, and the removal of separating triangles---and induction
%   on the number of vertices, we reduce to a problem in which $T$
%   has no separating triangles and every edge of $T$ is incident to a
%   triangle that has two edges intersecting the y-axis and $S$ is an
%   independent set that is on the y-axis.
%
%   \item Using local operations---removing edges and vertices and
%   introducting edges---on $T$ we obtain a (topological) quadrangulation
%   $Q$ for which every edge intersects the y-axis and $S$ is on the
%   y-axis.  Furthermore, $Q$ is structured so that, if we take any
%   non-crossing straigh-line drawing of $Q$ we can rei
%
%   \item Thus we have a quadrangulation $Q$ whose edges intersect
%   the y-axis in the order $e_1,\ldots,e_m$ and the vertex set of $Q$
%   contains $S$, still on the y-axis.
%
%   Let $y_1\le\cdots\le y_m$ be any sequence of numbers such that
%   $y_i=y_{i+1}$ if and only if $e_i$ and $e_{i+1}$ have a common endpoint
%   in $S$.  We show that, for any $\epsilon >0$, $Q$ has a planar
%   straight-line drawing such that the intersection of $e_i$ with the
%   y-axis is at $y_i\pm\epsilon$.  Furthermore, if $y_i=y_{i+1}$, then
%   the intersection of $e_i$ and $e_{i+1}$ with the y-axis is exactly
%   at $y_i$. In this way every vertex of $S$ is drawn on the y-axis
%   at precisely the desired location.
%\end{enumerate}




\section{Definitions}

Jordan curve

embedding of a graph = each vertex is a distinct point and each edge is a closed curve whose endpoints are its two vertices.  

we identify vertices of an embedded graph with their points and edges with their curves. By default, an edge curve includes its endpoints, otherwise we specify that it is an \emph{open} edge.

clean embedding - an embedding where the intersection between two edge curves is a finite set.  

rotation system = a graph with an ordering of the edges around each vertex. Every clean embedding defines a rotation system.

straight-line embedding = an embedding in which each edge curve is a line segment

plane embedding = an embedding in which no two edges intersect except possibly at their common endpoint.

face = one of the maximal connected subsets of $\R^2$ that remains after removing the union of all edges

for plane embeddings, we use the convention of listing the vertices of a face in counterclockwise order.

Triangulation = plane embedded graph in which each face is bounded by a 3-cycle.

Quadrangulation = plane embedded graph in which each face is bounded by a 4-cycle. Has $n\ge 4$ vertices and $2n-4$ edges.

outerplanar graph = plane embedded graph in which there is a single face that is incident to every vertex

Drawing = plane embedding 

Straight-line drawing = straight-line plane embedding

cutset = set of vertices whose removal disconnects the graph

separating triangle = cut set of size 3 in a triangulation

edge contraction = contracting $xy$ identifies $x$ and $y$ into a single vertex $v$ and eliminates any resulting parallel edges.  Preserves triangulations provided that $x$ and $y$ is not part of any separating triangle.


\section{The Proof}

\subsection{Drawing Quadrangulations}
\seclabel{quad}

\begin{lem}
   Let $Q$ be a drawing of a quadrangulation with $n\ge 5$ vertices
   and let $Q'$ be a straight-line embedding of $Q$ that has the same
   rotation system as $Q$.  Then $Q'$ is a straight-line drawing of $Q$.
\end{lem}

\begin{proof}
   Some argument about how a crossing in $Q'$ would require some
   quadrilateral face to have a vertex whose edge ordering is different
   in $Q$ than in $Q'$
\end{proof}



\begin{lem}\lemlabel{quad}
    Let
    \begin{compactitem}
    \item $Q$ be a quadrangulation with outer face $f$; 
    \item $C:[0,1]\to\R^2$ be a simple curve
     whose endpoints are both in the interior of $f$, 
     whose intersection with each open edge of $Q$
     consists of exactly one point, and that does not contain
     any vertex of $Q$; 
    \item $e_1,\ldots,e_m$ be the edges of $Q$ in the
    order they are intersected by $C$; 
    \item $y_1<\cdots<y_m$
    be any increasing sequence of numbers; and
    \item $\Delta$ be a triangle whose intersection with the y-axis
     is the segment $p_1p_m$ with endpoints $p_1=(0,y_1)$ and $p_m=(0,y_m)$.
    \end{compactitem}
    Then $Q$ has a straight-line
    drawing in which, for each $i\in\{1,\ldots,m\}$, the intersection
    of $e_i$ with the y-axis is a single point $(0,y_i)$ and the edges
    $e_1$ and $e_m$ are mapped to the the two edges of $\Delta$ that
    intersect the y-axis.
\end{lem}

\begin{proof}
   G\"unter's proof showing that the resulting system of equations has
   a unique solution.
\end{proof}

Next, we present a strengthening of \lemref{quad} that allows vertices
of $Q$ to be on the curve $C$.  Let $r_1,\ldots,r_m$ be a sequence
of vertices and edges in a planar graph and let $y_1<\cdots<y_m$ be a
sequence of numbers.  We say that a triangle $\Delta=\alpha\beta\gamma$
is \emph{compatible} with $r_1,\ldots,r_m$ and $y_1,\ldots,y_m$ if
\begin{compactenum}
  \item $\beta=p_1$ if $r_1$ is a vertex, otherwise $p_1$ in the interior
  of the edge $\alpha\beta$; and
  \item $\gamma=p_m$ if $r_m$ is a vertex, otherwise $p_m$ in the interior
  of the edge $\alpha\gamma$.
\end{compactenum}

\begin{lem}\lemlabel{quad2}
    Let
    \begin{compactitem}
    \item $Q$ be a quadrangulation with outer face $f$; 
    \item $C:[0,1]\to\R^2$ be a Jordan curve
     whose endpoint $C(0)=C(1)$ is in the interior of $f$,
     whose intersection with each edge of $Q$
     consists of exactly one point, and for which no vertex of $Q$ on $C$
     has neighbours both inside and outside of $C$;
    \item $r_1,\ldots,r_m$ be the edges and vertices of $Q$ in the
    order they are intersected by $C$; 
    \item $y_1<\cdots<y_m$
    be any increasing sequence of numbers; and
    \item $\Delta$ be a triangle that is compatible with $r_1,\ldots,r_m$ and $y_1,\ldots,y_m$.
    \end{compactitem}
    Then $Q$ has a straight-line
    drawing in which, for each $i\in\{1,\ldots,m\}$, the intersection
    of $r_i$ with the y-axis is a single point $(0,y_i)$ and three vertices
    of $f$ are mapped to the vertices of $\Delta$
\end{lem}

\begin{proof}
  Add artifical constraints for vertices on $C$.  If, for some reason, that approach fails, then use the argument we worked out for shifting points off of $C$ and then moving them back to the y-axis.
%   To fix this, first note that any vertex $r_i$ on $C$ is only incident
%   only to crossing edges and therefore the neighbours of $r_i$ are all
%   contained in $R$. Thus, we can deform $C$ in the neighbourhood of $r_i$
%   so that $r_i$ moves into the interior of $L$ and $C$ intersects each
%   of $r_i$'s incident edges $e_1,\ldots,e_d$ in exactly one point.  Now,
%   to apply \lemref{quad} we must specify numbers $y_1',\ldots,y_d'$ where
%   $(0,y_i')$ is the intersection point between $e_j$ and the y-axis.
%   To do this we choose any $y_1'<\cdots<y_d'$ satisfying
%   \[
%       y_{i-1} < y_1' < y_i < y_d' < y_{i+1} \enspace .
%   \]
%   By doing this for each vertex $r_i$ in $C$ we obtain a curve $C'$ and
%   a sequence $y_1''<\cdots<y_m''$ on which we can apply \lemref{quad} to
%   find a drawing of $Q$.  Now this drawing of $Q$ does not yet satisfy
%   the requirements of the theorem because there are vertices $r_i\in
%   C$ that are not in $C'$. However, the choice of $y_1'<\cdots<y_d'$
%   described above ensures that moving the vertex $r_i$ to $(0,y_i)$
%   does not introduce any crossings and gives a drawing of $Q$ that
%   satisfies all the relevant requirements of the theorem.  Finally,
%   in the process of building $Q$ described above, we showed how the
%   drawing of $Q$ can be extended to a drawing of $T$ that satisifies
%   all the requirements of the theorem.
\end{proof}



%\begin{lem}
%   Let $Q$ be a straight-line drawing of a quadrangulation each of whose edges intersect the
%   y-axis in exactly one point.  Let $v$ be a vertex of $Q$ that is not on
%   the y-axis, whose incident edges $vx_1,\ldots,vx_d$ intersect
%   the y-axis at y-coordinates $y_1'<\cdots<y_d'$, respectively, 
%   and suppose that no
%   other edges of $Q$ intersect the y-axis with y-coordinates in the interval 
%   $[y_1',y_d']$.
%   Then there is a unique index $i\in\{1,\ldots,d\}$ such that
%   \begin{enumerate}
%     \item for every $j\in\{1,\ldots,i-1\}$, all edges incident to $x_j$, aside from $vx_j$ intersect the y-axis below $(0,y_1')$;
%     \item for every $j\in\{i+1,\ldots,d\}$, all edges incident to $x_j$, aside from $vx_j$ intersect the y-axis above $(0,y_d')$; and
%     \item the embedding of $Q$ obtained by moving $v$ to $(0,y_i')$ is a straight-line drawing.
%   \end{enumerate}
%\end{lem}
%
%\begin{proof}
%   For the first two points, suppose on the contrary that there is some
%   pair of indices $k < \ell$ such that $x_k$ is incident to an edge
%   $wx_k$ that crosses the y-axis at some point $p$ above $(0,y_d')$
%   and $x_\ell$ is incident to an edge $ux_\ell$ that crosses the y-axis
%   at some point $q$ below $(0,y_1')$. Consider the triangle $abc$ with
%   $a=(0,y_k')$, $b=x_k$, and $c=p$.  Then the path $v,x_\ell,u$ enters
%   the interior of $abc$ through the segment $ac$ and exits through one of
%   the other two segments.  But this is a contradiction to the assumption
%   that $Q$ is a straight-line drawing, since it implies that one of the
%   edges in this path crosses at least one of the edges $vx_k$ or $wx_k$.
%
%   For the third point, observe 
%\end{proof}

\subsection{Drawing Collinear Sets}


We will sometimes make use of this simple fact:
\begin{obs}\obslabel{quad}
  If $q=abcd$ is a simple quadrilateral, then neither of the segments $ac$
  or $bd$ cross any of the edges of $q$.
\end{obs}

\begin{thm}\thmlabel{collinear-set}
   Let
   \begin{compactenum}
     \item  $T$ be a triangulation with outer face $f$;
     \item  $C:[0,1]\to\R^2$ be a Jordan curve whose endpoint $C(0)=C(1)$
            is in the interior of $f$ and whose intersection with each
            edge of $Q$ consists of at most one point;
     \item $r_1,\ldots,r_k$ be the sequence of vertices and open edges
           of $T$ that are intersected by $C$, and ordered in the order
           that they are intersected by $C$;
     \item $y_1<\cdots<y_k$ be any sequence of numbers; and
     \item $\Delta$ be a triangle that is compatible with 
           $r_1,\ldots,r_m$ and $y_1,\ldots,y_m$.
  \end{compactenum}
   Then, for any $\epsilon>0$, $T$ has a
   straight-line drawing in which the outer face $f$ is $\Delta$
   and, for each $i\in\{1,\ldots,k\}$, 
   \begin{compactenum}
       \item $r_i$ is drawn on the y-axis, with y-coordinate $y_i$
         if $r_i$ is a vertex; or
       \item (if $r_i$ is an edge) the intersection of $r_i$ with the
         y-axis has a y-coordinate in the interval
         $[y_i-\epsilon,y_i+\epsilon]$.
   \end{compactenum}
\end{thm}

\begin{proof}
   We call $y_i$ the (desired) \emph{crossing coordinate} for $r_i$. If
   a straight-line drawing contains an edge whose intersion with the
   y-axis is $\{(0,y)\}$ or a vertex at $(0,y)$, we say that the edge
   or vertex \emph{crosses} (the y-axis) at $y$.

   Let $L$ and $R$ be the bounded and unbounded components, respectively,
   of $\R^2\setminus C$. Let $\bar{L}=C\cup L$ and $\bar{R}=C\cup R$
   denote the closures of $L$ and $R$.  We say that an edge of $T$ is
   a \emph{crossing edge} if its intersection with $C$ is non-empty.
   A crossing edge is a \emph{proper crossing edge} if its intersection
   with each of $L$ and $R$ is non-empty.  We say that points in $L$
   are \emph{to the left of $C$} and points in $R$ are \emph{to the
   right of $C$}.

%   We prove an extension of the theorem to the case where $T$ is an
%   non-crossing embedded graph whose faces consist of triangles (3-cycles)
%   and quadrilateral (4-cycles) with the resriction that, for every
%   quadrilateral face $q$, all four edges of $q$ are crossing edges.
%   The proof is by induction on the number of non-crossing edges plus the
%   number of vertices of $T$.

%   \paragraph{Base Cases:}
%   There are three base cases tht we handle explicitly.  If $T$ contains
%   2 or fewer crossing edges, If $T$ is the complete graph, $K_4$ on 4
%   vertices, but only has only three crossing edges, then the theorem is
%   also easy to prove directly.  The last base case occurs when all edges
%   of $T$ are crossing edges.  In this case $T$ is bipartite and therefore
%   all its faces are quadrilaterals, so $T$ is a quadrilateralization.
%   This case is handled directly by \lemref{quad2}.
%
%   Thus we may assume that $T$ has at least one non-crossing edge and
%   at least 2 crossing edges.  

   The proof is by induction on $n+m$, where $n$ is the number of
   vertices of $T$ and $m$ is the number of crossing edges.  We begin
   by describing reductions that allow use to apply the inductive
   hypothesis. When none of these reductions are possible, we arrive
   at our base case. To handle this base case we argue that $T$ has a
   sufficiently simple structure that it can be handled by the algorithm
   for obtaining drawings of quadrangulations.

   \paragraph{Separating Triangles.}
   (See \figref{separating}.)
   A \emph{separating triangle} $xyz$ in $T$ is a cycle of length three
   whose removal disconnected $T$.  If $T$ contains a separating triangle
   $xyz$ then we remove all vertices from the interior of $xyz$ to obtain
   a graph $T^+$ in which $xyz$ is a face.  Since the intersection of $C$
   with each of $xy$, $yz$ and $zx$ consists of at most a single point,
   the vertices and edges of $T$ intersected by $C$ that are not in $T^+$
   appear as a contiguous subsequence $r_i,\ldots,r_j$.

   \begin{figure}
      \centering{\includegraphics{figs/separating}}
      \caption{Recursing on separating triangles in the proof of
      \thmref{collinear-set}}
      \figlabel{separating}
   \end{figure}

   Observe that each of $r_{i-1}$ and $r_{j+1}$ is either an edge
   or vertex of the triangle $xyz$.  Set $\epsilon'$ to be any
   value less than $\min\{\epsilon,y_{i}-y_{i-1}, y_{j+1}-y_j\}$.
   and apply induction on $T^+$ using the value $\epsilon'$
   and the sequences $r_1,\ldots,r_{i-1},r_{j+1},\ldots,r_k$ and
   $y_1,\ldots,y_{i-1},y_{j+1},\ldots,y_k$ to obtain a drawing of $T^+$.
   In the resulting drawing $xyz$ becomes a triangular face $\Delta'$.

   In the resulting drawing, Let $y_{i-1}'$ and $y_{j+1}'$
   be the respective y-coordinates of the intersections of
   $r_{i-1}$ and $r_{j+1}$ with the y-axis.  By our choice of
   $\epsilon'$, $y_{i-1}'<y_i<\cdots<y_j<y_{j+1}'$.  Observe that
   $\Delta'$ is compatible with $r_{i-1},\ldots,r_{j+1}$ and
   $y_{i-1}',y_i,\ldots,y_j,y_{j+1}'$.

   Let $T^-$ be the graph obtained by removing, from $T$, all
   vertices outside of $xyz$.  Now we apply induction on $T^-$ using
   the triangle $\Delta'$ and the sequences $r_{i-1},\ldots,r_{j+1}$ and
   $y_{i-1}',y_i,\ldots,y_{j},y_{j+1}'$.  Combining the drawings of $T^+$
   and $T^-$ yields a drawing of $T$ that satisfies the requirements of
   the theorem.  Thus, we may assume that $T$ has no separating triangles.

   \paragraph{Contractible Edges:}
   (See \figref{contractible}.)
   We say that a triangular face of $T$ is a \emph{proper crossing
   face} if it is incident to two proper crossing edges.  We say that a
   non-crossing edge of $T$ is \emph{contractible} it is not contained
   in the boundary of any crossing face.  
   \begin{figure}
      \centering{\includegraphics{figs/contractible}}
      \caption{Contracting and uncontracting edges in the proof of
      \thmref{collinear-set}}
      \figlabel{contractible}
   \end{figure}

   If $T$ contains a contractible edge $xy$ then we contract $xy$ to
   obtain a new vertex $u$ in a smaller graph $T'$.   We can then apply
   induction on $T'$ with the value $\epsilon'=\epsilon/2$ to obtain a
   drawing of $T'$ that satisfies all the conditions of the theorem under
   the stronger condition that each proper crossing edge $e_i$ crosses
   the y-axis in the interval $[y_i-\epsilon/2,y_i+\epsilon/2]$.

   To obtain a drawing of $T$ we uncontract $v$ by placing $x$ and $y$
   within a ball of radius $\epsilon/2$ centered at $v$. (That such
   a placement is always possible is a standard argument.)  Since the
   distance between $y$ and $v$ and $x$ and $v$ is at most $\epsilon/2$,
   each proper crossing edge $r_i$ incident on $x$ or $y$ will cross
   the y-axis in the interval $[y_i-\epsilon,y_i+\epsilon]$.

   Thus we may assume that $T$ has no separating triangles of contractible
   edges.

%   \paragraph{Eraseable edges}
%   We say that a non-crossing edge of $xy$ of $T$ is \emph{eraseable}
%   if neither of its endpoints is on $C$ and both its incident faces
%   intersect $C$.  If $T$ contains an eraseable edge $xy$, then we remove
%   the edge $xy$ from $T$ to obtain smaller graph $T'$ on which we can
%   apply induction. In the resulting drawing of $T'$, $x$ and $y$ lie on
%   a common face (which may be the outer face of $T'$) and are visible.
%   We can therefore add the edge $xy$ to obtain the desired drawing
%   of $T$.

   \paragraph{Flippable edges.}
   (See \figref{flippable}.)
   We say that a non-crossing edge $xy$ of $T$ is flippable if there
   exists distinct vertices $z$, $a$, $b$, and $c$, such that 
   \begin{compactenum}
      \item $xyb$, $zyc$, $xza$ are crossing faces of $T$;
      \item $xyz$ is a non-crossing face of $T$; and (
      \item $C$ intersects $za$, $xa$, $xb$, $yb$, and $yc$ in this order; or 
      \item or $C$ intersects $xa$, $xb$, $yc$, $zc$, $za$, in this order).  
   \end{compactenum}
   \begin{figure}
      \centering{\includegraphics{figs/flippable}}
      \caption{Flipping edges in the proof of
      \thmref{collinear-set}}
      \figlabel{flippable}
   \end{figure}

   If $T$ contains the flippable edge $xy$ then we remove $xy$ and
   replace it with $zc$ to obtain a new graph $T'$.
   Note that, since $T$ has no separating triangles, the edge $zc$
   is not already present in $T$.
   After choosing a crossing coordinate for $zc$ somewhere
   between those of $xc$ and $yc$ we can then inductively draw $T'$.  

   We claim that in the resulting drawing of $T'$, the only open edge
   that intersects the open segment $xy$ is $zc$.  In particular, we must
   ensure that $z$ is not a reflex vertex in the quadrilateral $xcyz$.
   To show this we distinguish between the two possible cases (3 and 4)
   in the definition of flippable edges. In Case~3, The existence of the
   edges $za$ and $zb$ ensure that, in the resulting drawing of $T'$,
   $xcyz$ is convex.  In Case~4, the triangle $zxa$ is convex and $xcyz$
   is contained in this triangle, therefore $z$ is a convex vertex
   of $xcyz$.

   In either case, removing $zc$ from the drawing of $T'$ and replacing
   it with $xy$ yields the desired drawing of $T$.

   \paragraph{The Base Case.}

   Finally, we are left with a situation in which $T$ is a triangulation
   with no separating triangles, no contractible edges, and no flippable
   edges.

   If $T$ is the complete graph on $K_3$ or $K_4$ on three or four
   vertices, then the theorem is trivial, so we may assume that $T$
   has at least 5 vertices.

%   We claim that every non-crossing edge $xy$ of $T$ is contained in the
%   boundary of two crossing faces $xya$ and $yxb$.  To see why this is so,
%   observe that if some non-crossing edge $xy$ is not contractible then
%   one of $xy$'s incident faces, $yxc$ is proper crossing.  Suppose, for
%   the sake of contradiction, that the other face $xyz$, incident on $xy$
%   is not crossing.  Since neither $zx$ nor $yz$ is contractible, they
%   must be incident on crossing triangles $xza$ and $zyb$, respectively.
%   Since $T$ contains no separating triangles, we know that $b\neq a$,
%   otherwise $xya$ would separate $z$ from $c$.
%
%   This leaves us in the situation in which we have distinct vertices $x$,
%   $y$, $z$, $a$, $b$, $c$, such that $xyb$, $zyc$, $xza$ are crossing
%   faces of $T$ and $xyz$ is a non-crossing face of $T$.  Checking the
%   definition of flippable edge then ensures that at least one of $xy$,
%   $yz$, or $za$ is a flippable edge.
%
%
%
%   Either these triangles form a separating triangle that separates $z$
%   from $c$ or the existence of the edges $za$ and $zb$ establishes that
%   $xy$ is flippable.
%
%
%   We claim that $T$ has no non-crossing edges except those in $L$
%   that have one endpoint on $C$.  To see why this is so, observe that
%   if some non-crossing edge $xy$ is not contractible then one of $x$
%   or $y$ is on $C$ or one of $xy$'s incident faces, $yxc$ is crossing.
%   Since $xy$ is no erasable, it must be incident on a non-crossing
%   face $xyz$.  Since neither $zx$ nor $yz$ is contractible, they must
%   be incident on crossing triangles $xza$ and $xzb$, respectively.
%   Either these triangles form a separating triangle that separates $z$
%   from $c$ or the existence of the edges $za$ adn $zb$ establishes that
%   $xy$ is flippable.
%
%   Any cycle in $T$ that uses only crossing edges necessarily has even
%   length.  Therefore, the only triangular faces of $T$ include at least
%   one vertex on $C$.  But this implies that every triangular face $T$
%   includes one vertex from each of $C$, $L$, and $R$, otherwise $C$
%   includes a non-crossing edge neither of whose vertices is on $C$.
%
%   TODO: Mention conditions on input having exactly two triangular faces
%   for each vertex on $C$.
%   None of the operations used to arrive at this base case remove
%   crossing edges or edges incident to $C$.  Thus, every vertex on $C$
%   is incident to exactly two triangles in $T$ and these are the only
%   triangles in $T$.  For every vertex $v=r_i$ on $C$, we split $v$
%   into two vertices $x$ and $y$ joined by an edge $xy$ and make $x$
%   is adjacent to $v$'s neighbours in $L$ and $y$ adjacent to $v$'s
%   neighbours in $R$.  We embed $x$ and $y$ in the neighbourhood of $v$
%   in such a way that $C$ crosses every edge incident to $x$ and $y$.
%   We specify crossing coordinates for all of these crossings that have
%   a value very close to that of $y_i$.
%
%   Observe that, after we do this for each every vertex on $C$, the
%   resulting graph is a quadrangulation.  Now apply the theorem above....
%
%
%   \paragraph{Reductions}
%   Next we consider what can be done when $T$ has no separating
%   triangles and no contractible edges.  Note that this implies that
%   every vertex $v$ of $T$ is incident to at least one crossing edge
%   since, otherwise every edge incident to $v$ is contractible.
%  
%   Consider the graph $H$ consisting of only the non-crossing edges of $T$
%   and their endpoints.  Some face $g$ of $H$ contains $C$.  We claim
%   that $H$ is outerplanar because every vertex of $H$ is incident to
%   at least one crossing edge and is therefore on the boundary of $g$.
%
%   Next, we claim that $H$ has only three kinds of 2-connected components:
%   (i)~single edges, (ii)~3-cycles, and (iii)~two three cycles sharing an edge.
%
%   To see why this is so, suppose that some 2-connected component $K$ of $H$ has $k\ge 5$ vertices.  Since $T$ is triangulated, $K$ is a triangulated $k$-gon.  But $ and by repeatedly removing vertices of degree 
%
%
% this component therefore has $  Then
%   To see why this is so, observe that every edge $xy$ of $H$ is not contractible so either:
%   \begin{enumerate}
%       \item $xy$ bounds a non-proper
%
%   \end{enumerate}
%
%
%
%
%   Furthermore, every edge $e$ of $H$ is on the boundary of $g$ since,
%   otherwise, neither of the two triangles incident on $e$ would be proper
%   crossing triangles and $e$ would be contractible.  This means that
%   $H$ is a \emph{cactus graph}---a graph in which each edge is incident
%   to at most one cycle.  Furthermore, each cycle of $H$ is a 3-cycle.
%   To see why, observe that, since $T$ is a triangulation, if $H$
%   contained a $k$-cycle for $k\ge 4$, this cycle would have a chord.
%
%   Because $T$ has at leat one non-crossing edge, $H$ contains at least one
%   edge.  Therefore, $H$ has at least one 2-connected component, 
%   which is either a 3-cycle
%   or a single edge and assume, without loss of generality that this
%   2-connected component is contained in $\bar L$.  
%   There are two cases to consider:
%   \begin{enumerate}
%	\item The 2-connected component is a single edge $xy$.	Let $xya$
%	and $yxb$ be the two faces of $T$ incident to $xy$.  Note that
%	$a$ and $b$ are both in $R$ since, otherwise, the 2-connected
%	component containing $xy$ would be a 3-cycle.
%	Therefore, removing the edge $xy$ from $T$ creates a
%	quadrangular face $xbya$ consisting of only crossing
%	edges. \obsref{quad} ensures that, after applying induction
%	on this smaller graph, the edge $xy$ can be added without
%	introducing crossings.  (It is worth noting that the analysis
%	of this case remains true even if $xy$ is an edge of the
%	outer face.)
%        \begin{figure}
%           \centering{\includegraphics{figs/1b}}
%           \caption{When a 2-connected component of $H$ is an edge. The second figure illustrates the situation when $xy$ is an edge of the outer face.}
%            \figlabel{1b}
%        \end{figure}
%        \item The component is a face $xyz$ of $T$.  Since none of $xy$,
%        $yz$, or $zx$ was contracted, there exists vertices $a,b,c\in R$
%        (possibly not distinct) such that $xya$, $yzb$ and $zxc$ are
%        triangular faces of $T$.\footnote{Note that this is even true if
%        $xy$ was not contracted because $x,y\in C$. In that case there
%        is still an $a\in R$.} 
%
%        We claim that $a$, $b$, and $c$ are distinct vertices. If not,
%        this means that $xyz$ all have a common neighbour, say $a$. Thus
%        $T$ contains the complete graph $K_4$ as a subgraph.  The case in
%        which $T$ is $K_4$ was handled as a base case in our induction,
%        therefore $T$ contains some vertex $v\not\in\{x,y,z,a\}$. But
%        contradicts the assumption that $T$ contains no separating
%        triangles, since $v$ must be in one of the four triangular faces
%        of this $K_4$ subgraph, say $xyz$, but then $xyz$ is a separating
%        triangle that separates $v$ from $a$.
%
%        Thus, $a$, $b$, and $c$ are three distinct vertices of $T$
%        (see \figref{2c}). Note that, in this case, $T$ does not
%        contain the edge $yc$ since, if it did, then $yxc$ would be
%        a separating triangle that separates $a$ from $b$.
%        \begin{figure}
%           \centering{\includegraphics{figs/2b}}
%           \caption{When a 2-connected component of $H$ is a triangle $xyz$ whose three vertices have two common neighbours $a,c\in R$.}
%           \figlabel{2c}
%        \end{figure}
% 
%        There are two cases to consider: 
%
%        If none of $xy$, $yz$, or $zx$ is an edge of the outer face,
%        then (after appropriate relabelling) $C$ intersects the edges
%        $yb$, $zb$, $zc$, $xc$, $xa$, $ya$ in this order.  In this case,
%        we remove the edges $xy$, $yz$ and $zx$ and add the edge $yc$.
%        At the same time we specify a crossing location of the edge
%        $yc$ on the y-axis anywhere in the interior of the segment
%        whose ends points are the crossing locations of $xc$ and $zc$.
%        This creates two quadrangular faces $ycxa$ and $ybzc$.  After
%        inductively drawing this smaller graph, the vertices $x$ and $z$
%        are reflex vertices of these two faces.  Since every non-convex
%        quadrilateral is star-shaped and its kernel contains its reflex
%        vertex, we can remove the edge $yc$ and add the edges $xy$, $yz$
%        adn $zx$ without introducing crossings.
%
%        If $xy$ is on the outer face, then $C$ intersects the edges
%        $ya$, $yb$, $zb$, $zc$, $xc$, $xa$ in this order. In this
%        case we perform exactly the same operation:  we remove the
%        edges $xy$, $yz$ and $zx$ and add the edge $yc$ and specify a
%        crossing location of the edge $yc$ on the y-axis anywhere in
%        the interior of the segment whose ends points are the crossing
%        locations of $xc$ and $zc$.  Again, this creates two quadrangular
%        faces $ycxa$ and $ybzc$ only now $ycxa$ is the outer face.
%        After inductively drawing this smaller graph, the vertices $z$
%        is a reflex vertices in $ybzc$ and the vertex $x$ sees the entire
%        edge $cy$.  This ensure that we can remove the edge $yc$ and
%        add the edges $xy$, $yz$ and $zx$ without introducing crossings.
%   \end{enumerate}
%   This completes the proof.
\end{proof}

\begin{cor}
  Every collinear set is free.
\end{cor}

\begin{proof}
   Subdivide edges contained in $C$ and then apply \thmref{main}.
\end{proof}


\end{document}


