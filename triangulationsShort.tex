%!TEX root = mainShort.tex
\section{Triangulations}
\seclabel{triangulations}

%We will sometimes make use of this simple fact:
%\begin{obs}\obslabel{quad}
%	If $q=abcd$ is a simple quadrilateral, then neither of the segments $ac$
%	or $bd$ cross any of the edges of $q$.
%\end{obs}

%As is the case with \thmref{a-graph} there is an annoying case distinction
%that occurs when $Y$ contains vertices on the outer face.  

In this section we prove that every collinear set is free. Recall that a {\em triangulation} is a (not necessarily straight-line) plane embedding of an edge-maximal planar graph. Let $T$ be a triangulation, let $r_1,\ldots,r_k$ be a sequence of vertices and edges in $T$, and let $y_1<\cdots<y_k$ be a sequence of numbers. A triangle $\Delta=\alpha\beta\gamma$ is \emph{compatible} with $r_1,\ldots,r_k$ and $y_1,\ldots,y_k$ if the following conditions hold: (1) if $r_1$ is a vertex, then $\beta=(0,y_1)$, otherwise $(0, y_1)$ in the interior of the edge $r_1=\beta\gamma$; and (2) if $r_m$ is a vertex, then $\alpha=(0,y_m)$, otherwise $(0,y_m)$ in the interior of the edge $r_m=\alpha\gamma$. We have the following.


%A triangulation is \emph{admissible} if the intersection of $Y$ with each edge of the triangulation is either empty, a single point,
%or the entire edge.  


\begin{thm}\thmlabel{main}
	Let $T$ be a triangulation and $C$ be an admissible curve for $T$. Further, let $y_1<\cdots<y_k$ be a sequence of numbers and let $r_1,\ldots,r_k$ be the sequence of vertices and open edges of $T$ that are intersected by $C$, in the order in which they are intersected by $C$ (each edge of $T$ that is entirely on $C$ has its end-vertices represented by two consecutive elements $r_i$ and $r_{i+1}$ of $r_1,\ldots,r_k$, while the open edge is not in $r_1,\ldots,r_k$). Finally, let $\Delta$ be a triangle that is compatible with 
		$r_1,\ldots,r_k$ and $y_1,\ldots,y_k$.

	Then, for any $\epsilon>0$, $T$ has a \Fary\ embedding such that the outer face is delimited by $\Delta$ and such that the following hold for each $i=1,\ldots,k$: (1) if $r_i$ is a vertex, then it is drawn at $(0,y_i)$; and (2) if $r_i$ is an edge, then the intersection of $r_i$ with $Y$ has $y$-coordinate in the interval $[y_i-\epsilon,y_i+\epsilon]$.
\end{thm}

\begin{proof}
	%   We call $y_i$ the (desired) \emph{crossing coordinate} for $r_i$. If
	%   a \Fary\ embedding contains an edge whose intersection with the
	%   $y$-axis is $\{(0,y)\}$ or a vertex at $(0,y)$, we say that the edge
	%   or vertex \emph{crosses the $y$-axis at $y$}.
	%
	%   Let $L=C^-$, $R=C^+$.  
We start by classifying the edges of $T$. An edge of $T$ is \emph{marked} if its intersection with $C$ is non-empty, otherwise it is \emph{unmarked}. A marked edge might intersect $C$ in a single point (then it is a \emph{crossing edge}), or might lie on $C$. If an edge lies on $C$ or if it is unmarked, then it is a \emph{non-crossing edge}. 
	
	%   We prove an extension of the theorem to the case where $T$ is an
	%   non-crossing embedded graph whose faces consist of triangles (3-cycles)
	%   and quadrilateral (4-cycles) with the resriction that, for every
	%   quadrilateral face $q$, all four edges of $q$ are crossing edges.
	%   The proof is by induction on the number of non-crossing edges plus the
	%   number of vertices of $T$.
	
	%   \paragraph{Base Cases:}
	%   There are three base cases tht we handle explicitly.  If $T$ contains
	%   2 or fewer crossing edges, If $T$ is the complete graph, $K_4$ on 4
	%   vertices, but only has only three crossing edges, then the theorem is
	%   also easy to prove directly.  The last base case occurs when all edges
	%   of $T$ are crossing edges.  In this case $T$ is bipartite and therefore
	%   all its faces are quadrilaterals, so $T$ is a quadrilateralization.
	%   This case is handled directly by \lemref{quad2}.
	%
	%   Thus we may assume that $T$ has at least one non-crossing edge and
	%   at least 2 crossing edges.  
	
	The proof is by induction on the number of vertices of $T$, primarily, and on the number of non-crossing edges of $T$, secondarily.
	We begin by describing reductions that allow us to apply the
	inductive hypothesis. When none of these reductions applies,
	we arrive at our base case. To handle the base case 
	we remove every unmarked edge of $T$ to obtain an A-graph, on which we
	apply \thmref{a-graph}.

% we argue that
%	$T$ has a sufficiently simple structure that it can be handled by
%	\thmref{a-graph}.  In particular, when no reduction applies, we can
	
%	Before continuing, we dispense with one easy special case.  If $Y$
%	contains an edge $e$ of the outer face, then every vertex
%	of $G$ is contained in $L\cup Y$ or every vertex of $G$ is contained
%	in $Y\cup R$.  In this case, the definition of compatible triangle
%	implies that the edge $\alpha\beta$ of $\Delta$ is contained in the
%	$y$-axis.  In this case, we can simply apply Tutte's Convex Embedding
%	Theorem to obtain a \Fary\ embedding of $G$ in which the outer face is embedded
%	on $\Delta$ with $e$ embedded on $\alpha\beta$, .  This embedding
%	satisifies all the conditions of the theorem.  Therefore, for the
%	remainder of this proof, we assume that $C$ intersects the interior
%	of at least one inner face of $T$.
	
	\subsubsection{Separating Triangles.}
	%(See \figref{separating}.)
	If $T$ contains a separating triangle $xyz$, then denote by $T^+$ (respectively, $T^-$) the triangulation obtained from $T$ by removing all the vertices in the interior (respectively, exterior) of $xyz$. Note that $xyz$ delimits an inner face of $T^+$ and the outer face of $T^-$. 
	
	If the interior of $xyz$ does not intersect $C$, we apply induction on $T^+$ (note that $|V(T^+)|<|V(T)|$) and then use Tutte's Convex Embedding Theorem  \cite{tutte:how} to draw $T^-$ so that its outer face is delimited by the triangle representing the cycle $xyz$ in the constructed \Fary\ embedding of $T^+$.
	
%	\begin{figure}[tb]
%		\centering{\includegraphics{figs/separating}}
%		\caption{Recursing on separating triangles in the proof of
%			\thmref{main}}
%		\figlabel{separating}
%	\end{figure}
	If the interior of $xyz$ intersects $C$, then the vertices and edges of $T$ intersected by $C$ that are not in $T^+$ form a contiguous
	subsequence $r_i,\ldots,r_j$ of $r_1,\ldots,r_k$. Each of $r_{i-1}$ and $r_{j+1}$ is either an edge or a vertex of $xyz$.  
	Let $\epsilon'$ be any
	value less than $\min\{\epsilon,y_{i}-y_{i-1}, y_{j+1}-y_j\}$. Apply induction on $T^+$ with value $\epsilon'$ and sequences $r_1,\ldots,r_{i-1},r_{j+1},\ldots,r_k$ and
	$y_1,\ldots,y_{i-1},y_{j+1},\ldots,y_k$. In the obtained \Fary\ embedding of $T^+$ let $\Delta'$ be the triangle representing  $xyz$ and let $y_{i-1}'$ and $y_{j+1}'$
	be the $y$-coordinates of the intersections of
	$r_{i-1}$ and $r_{j+1}$ with $Y$.  By the choice of
	$\epsilon'$ we have $y_{i-1}'<y_i<\cdots<y_j<y_{j+1}'$.  Observe that
	$\Delta'$ is compatible with $r_{i-1},\ldots,r_{j+1}$ and
	$y_{i-1}',y_i,\ldots,y_j,y_{j+1}'$.
	%
	Apply induction on $T^-$ with value $\epsilon$ using the triangle $\Delta'$ and the sequences $r_{i-1},\ldots,r_{j+1}$ and
	$y_{i-1}',y_i,\ldots,y_{j},y_{j+1}'$.  Combining the \Fary\ embeddings of $T^+$
	and $T^-$ yields the desired \Fary\ embedding of $T$. In the following we assume that $T$ has no separating triangles.
	
	
	\begin{figure}[tb]
		\centering{\includegraphics[scale=0.85]{figs/contractible}}
		\caption{Contracting and uncontracting edges in the proof of
			\thmref{main}}
		\figlabel{contractible}
	\end{figure}

	\subsubsection{Contractible Edges.}
	(See \figref{contractible}.)
	A face of $T$ is a \emph{crossing
		face} if it is incident to two crossing edges.  An
	unmarked edge of $T$ is \emph{contractible} if it is not contained
	in the boundary of any crossing face.  
	
	If $T$ contains a contractible edge $xy$ then we contract $xy$ to
	obtain a new vertex $v$ in a smaller triangulation $T'$.   We then apply
	induction on $T'$ with the value $\epsilon'=\epsilon/2$ to obtain a \Fary\
	embedding of $T'$ such that each crossing edge $e_i$ crosses
	$Y$ in the interval $[y_i-\epsilon/2,y_i+\epsilon/2]$.
	%
	To obtain a \Fary\ embedding of $T$ we uncontract $v$ by placing $x$ and $y$
	within a ball of radius $\epsilon/2$ centered at $v$. (That such
	a placement is always possible is a standard argument, see, e.g.,~\cite{fary,w-sp-05}.)  Since the
	distance between $y$ and $v$ and the distance between $x$ and $v$ are each at most $\epsilon/2$,
	each crossing edge $r_i$ incident to $x$ or $y$ crosses $Y$ in the interval $[y_i-\epsilon,y_i+\epsilon]$, as required.
	From now on we assume that $T$ has no separating triangles or contractible
	edges.
	
	%   \paragraph{Eraseable edges}
	%   We say that a non-crossing edge of $xy$ of $T$ is \emph{eraseable}
	%   if neither of its endpoints is on $C$ and both its incident faces
	%   intersect $C$.  If $T$ contains an eraseable edge $xy$, then we remove
	%   the edge $xy$ from $T$ to obtain smaller graph $T'$ on which we can
	%   apply induction. In the resulting drawing of $T'$, $x$ and $y$ lie on
	%   a common face (which may be the outer face of $T'$) and are visible.
	%   We can therefore add the edge $xy$ to obtain the desired drawing
	%   of $T$.
	
	\subsubsection{Flippable edges.} (See \figref{flippable}.) We say that an unmarked edge $xy$ of $T$ is \emph{flippable} if there
	exist distinct vertices $z$, $a$, $b$, and $c$ such that: (1) $xyb$, $zyc$, $xza$ are crossing faces of $T$; (2) $xyz$ is a non-crossing face of $T$; and either (3a) $C$ intersects $za$, $xa$, $xb$, $yb$, $yc$, and $zc$ in this order, or (3b) $C$ intersects $xa$, $xb$, $yb$, $yc$, $zc$, and $za$ in this order.  
	
	If $T$ contains the flippable edge $xy$ then we replace it with $zb$ to obtain a new triangulation $T'$. Note that, since $T$ has no separating triangles, the edge $zb$ is not in $T$. Further, $T'$ has the same number of vertices of $T$ and one less non-crossing edge. After choosing a crossing coordinate $y_{zb}$ for $zb$ between those $y_{xb}$ and $y_{yb}$ of $xb$ and $yb$, respectively, we can inductively embed $T'$ with value $\epsilon$ and sequences $r_1,\dots,xb,zb,yb,\dots,r_k$ and $y_1,\dots,y_{xb},y_{zb},y_{yb},\dots,y_k$.
	%
	We claim that in the resulting \Fary\ embedding of $T'$, the only open edge
	that intersects the open segment $xy$ is $zb$. It suffices to prove that $z$ is not a reflex vertex in the quadrilateral $xbyz$; note that $b$ is not a reflex vertex in $xbyz$, since $bx$ and $by$ are crossing edges. In Case~(3a), the existence of the
	edges $za$ and $zc$ ensures that, in the \Fary\ embedding of $T'$,
	$xbyz$ is convex. In Case~(3b), the triangle $zxa$ is convex and $xbyz$ is contained in this triangle, therefore $z$ is convex. In either case, removing $zb$ from the \Fary\ embedding of $T'$ and replacing it with $xy$ yields the desired \Fary\ embedding of $T$.
	
	\begin{figure}[tb]
		\centering{\includegraphics[scale=0.85]{figs/flippable}}
		\caption{Flipping edges in the proof of
			\thmref{main}}
		\figlabel{flippable}
	\end{figure}
	
	
	\subsubsection{Edges on $C$.} If $T$ contains an edge $xy$ that lies on $C$, then we treat it as we treated flippable edges. In this case, $xy$ is incident to two triangles $xyz$ and $yxb$ with $z\in C^+$ and $b\in C^-$. We replace $xy$ with an edge $zb$ to obtain a new triangulation $T'$ with the same number of vertices of $T$ and one less non-crossing edge. We apply induction and get a \Fary\ embedding of $T'$, in which $z$ and $b$ are
	on opposite sides of $Y$ and $x$ and $y$ are on $Y$, hence 	neither $z$ nor $b$ is a reflex vertex of the quadrilateral $xzyb$.
	Thus, removing $zb$ and adding $xy$ gives a \Fary\ embedding of $T$.
	
	\subsubsection{Base case.} We are left with the case in which $T$ is a triangulation
	with no separating triangles, no contractible edges, no flippable
	edges, and no edge contained in $C$.  If $T$ is the complete graph
	on three or four vertices, then the proof is trivial,
	so we may assume that $T$ has at least 5 vertices.
	
	\begin{claimx} \obslabel{unmarked}
	Any unmarked edge $xy \in C^+$ is on the boundary of two faces $xyz$ and $yxb$ where $z,b\in\{C \cup C^-\}$.
	\end{claimx}

	Symmetrically, every unmarked edge $xy\in C^-$ is incident to two faces
	$xyz$ and $yxb$ with $z,b\in \{C \cup C^+\}$.  This implies that no face of $T$ contains more than one unmarked edge.
	%
	Thus, every unmarked edge of $T$ is incident to two faces that intersect $C$. The union of these two faces is a quadrilateral whose boundary consists of four edges that intersect $C$.
	Let $\tilde{G}$ denote the embedded plane graph obtained by removing all unmarked edges
	from $T$.  By \thmref{dujmovic-frati}, we know that $\tilde G$ has
	a \Fary\ embedding $G$ whose edges and vertices intersect $Y$ in the
	same order as the same edges and vertices intersect $C$ in $\tilde{G}$. We have the following.
	
	\begin{claimx} \label{claim-a-graph}
		$G$ is an A-graph.
	\end{claimx}
	
	%
	%The graph $Q^*$ has two triangular faces $vab$
	%   and $vcd$ incident to $v$ such that $ab=r_{i-1}$ and $cd=r_{i+1}$.
	%   Split $v$ into two vertices $x\in L$ and $y\in R$ joined by the edge
	%   $xy$, make $x$ adjacent to all neighbours of $v$ in $R$, and make
	%   $y$ adjacent to all neighbours of $v$ in $L$. See \figref{split}.
	%   This splitting operation eliminates the triangular faces $vab$ and
	%   $vcd$ and introduces the quadrangular faces $xyab$ and $yxcd$.
	%
	%   \begin{figure}
	%      \centering{
	%        \begin{tabular}{c|c}
	%            \includegraphics{figs/split} & \includegraphics{figs/split-outer}
	%        \end{tabular}}
	%      \caption{Splitting vertices on $C$ in the proof of
	%      \thmref{main}.}
	%      \figlabel{split}
	%   \end{figure}

	We can hence apply \thmref{a-graph} to
	obtain a \Fary\ embedding of $G$ in which the intersection of $r_i$
	with $Y$ is at $(0,y_i)$.  Each internal edge $ac$ of $T$ not in $G$
	corresponds to a quadrangular face $q=abcd$ of $G$ in which $a$ or $c$ is a
	reflex vertex.  Therefore, the edge $ac$ can be added to the embedding
	without introducing crossings.  A single external edge $\alpha\beta$
	on the outer face of $T$ might not appear in $G$. In this case the outer
	face of $G$ is a quadrilateral $q'=\alpha a \beta b$ in which $a$ or $b$ is
	a reflex vertex, so the segment $\alpha\beta$ lies outside of $q'$, and the edge $\alpha\beta$ can therefore be added to the embedding of $G$
	without introducing crossings. Reinserting each edge of $T$ not in $G$ gives the desired
	\Fary\ embedding of $T$. This concludes the proof of \thmref{main}.
\end{proof}
	
We are finally ready to prove \thmref{our-bang}. Given an embedded plane graph $G$, a collinear set $S$ in $G$,
	and any $y_1'<\cdots<y_{|S|}'$, we need to prove that $G$ has
	a \Fary\ embedding in which the vertices in $S$ are embedded at
	$(0,y_1'),\ldots,(0,y_{|S|}')$. We can assume that $G$ is a triangulation. Indeed, if it is not,  edges can be added to it so that it becomes a triangulation, while preserving the property that $S$ is collinear set; after constructing a \Fary\ embedding in which the vertices of $S$ are embedded at $(0,y_1'),\ldots,(0,y_{|S|}')$ the inserted edges can be removed obtaining the desired \Fary\ embedding of the initial graph. Assume hence that $G$ is a triangulation. 	 \thmref{dujmovic-frati} implies
	that there exists a Jordan curve $C$ that is admissible for $G$
	and that contains all the vertices of $S$ in some order, say
	$v_1,\ldots,v_{|S|}$.  The curve $C$ intersects a subset of the edges
	and vertices of $G$ in some order $r_1,\ldots,r_k$.  We choose any
	sequence $y_1<\cdots<y_k$ so that, for all $i\in\{1,\ldots,k\}$ and
	$j\in\{1,\ldots,|S|\}$, $y_i = y_j'$ if $r_i=v_j$.  We select
	any triangle $\Delta$ that is compatible with $r_1,\ldots,r_k$ and
	$y_1,\ldots,y_k$ and choose $\epsilon = \min\{(1/3)(y_{i+1}-y_{i}):
	i\in\{1,\ldots,k-1\}\}$.  \thmref{main} then gives us a \Fary\
	embedding of $G$ in which the vertices in $S$ are embedded at $(0,y_1'),\ldots,(0,y_{|S|}')$, as required by \thmref{our-bang}.









