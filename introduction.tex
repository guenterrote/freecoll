%!TEX root = main.tex


\section{Introduction}

%A \emph{plane straight-line drawing} of a planar graph $G$ is a
%geometric representation of $G$ where vertices of $G$ are represented
%as a set of points in the plane and each pair of adjacent vertices
%$\{v,w\}$ is connected by a line segment $\overline{vw}$ that
%intersects only $v$ and $w$ and no other edge or vertex in $G$. 

A \emph{straight-line drawing} of a graph $G$ maps each vertex to a point in the plane and each edge to a line segment between its endpoints. A straight-line
drawing is \emph{plane} if no pair of edges cross other than at a
common endpoint. A set
of vertices  $S\subseteq V(G)$ in a planar graph $G$ is a \emph{free
  set} if for any set of points $X$ in the plane with $|X|=|S|$, $G$ has a plane
straight-line drawing in which the vertices of $S$ are mapped to the points in $X$.  Free sets are useful tools in graph drawing
and related areas and have been used to settle problems in untangling~\cite{bose.dujmovic.ea:polynomial,dalozzo.dujmovic.ea:drawing,dujmovic:utility,ravsky.verbitsky:on,ravsky.verbitsky:on-arxiv}, column planarity~\cite{dalozzo.dujmovic.ea:drawing,dujmovic:utility}, universal point subsets~\cite{dalozzo.dujmovic.ea:drawing,dujmovic:utility},
and partial simultaneous geometric drawings~\cite{dujmovic:utility}.

\begin{figure}[htb]
  \centering
  \includegraphics{figs/collinear}
  \caption{The 4 red vertices form a collinear set $S$. On the
    right, the graph is redrawn so that vertices of $S$ lie at some
    other collinear locations.}
  \label{fig:collinear}
\end{figure}

 A set of vertices  $S\subseteq V(G)$ in a planar graph $G$ is a
 \emph{collinear set} if $G$ has a plane straight-line drawing in
 which all vertices in $S$ are mapped to a single line,
see \figref{collinear}.
 A collinear set $S$
is a \emph{free collinear set} if, for any collinear set of points in
the plane $X$ with $|X|=|S|$, $G$ has a plane straight-line drawing in
which the vertices of $S$ are mapped the points in $X$.  
Ravsky and Verbistky \cite{ravsky.verbitsky:on,ravsky.verbitsky:on-arxiv}
define $\sv(G)$ and $\tv(G)$ as the respective sizes of the
largest collinear set and largest free collinear set in $G$, and ask
the following question:
\begin{quote}
	How far or close are parameters $\tv(G)$ and $\sv(G)$? It
	seems that \emph{a priori} we even cannot exclude equality. To clarify
	this question, it would be helpful to (dis)prove that every collinear
	set in any straight-line drawing is free.
\end{quote}
%
Here, we answer this question by proving that, for every planar graph $G$,
$\tv(G)=\sv(G)$, that is:

\begin{thm}\thmlabel{our-bang}
Every collinear set is a free collinear set. 
\end{thm}

Let $v(G)$ denote the largest free set for a planar graph $G$. Clearly, we have $v(G)\leq \tv(G) \leq \sv(G)$. Further, as discussed in detail below, it is well-known that $v(G)=\tv(G)$. However, prior to our work, the best known bound between $v(G)$,
$\tv(G)$, and $\sv(G)$ in the other direction was $v(G),\tv(G) \geq \sqrt{\sv(G)}$, proved by Ravsky and Verbitsky~\cite{ravsky.verbitsky:on}. 
Thanks to \thmref{our-bang}, we now know a stronger bound, in fact the ultimate $v(G)=
\tv(G) = \sv(G)$ relationship. This relationship was
previously only known for planar $3$-trees
\cite{dalozzo.dujmovic.ea:drawing}. \thmref{our-bang}, in fact, implies a stronger result than $v(G)= \tv(G) = \sv(G)$:


%Ravsky and Verbitsky
%\cite{ravsky.verbitsky:on} showed earlier that $2$-trees have large, $\Omega(n)$,
%free-colinear sets, however their result did not give
%\thmref{our-bang} for $2$-trees. 




%Before
%\thmref{our-bang}, the following relationships were known between
%$\sv(G)$, $\tv(G)$ and $v(G)$, starting with the obvious inequality:
%$v(G)\leq \tv(G) \leq \sv(G)$. It was
%also known that $v(G)=\tv(G)$, as discussed in detail below. However, in
%the other direction, the best known bound between $v(G)$,
%$\tv(G)$ and $\sv(G)$ was $\tv(G) \geq \sqrt{\sv(G)}$ and thus $v(G)\geq
%\sqrt{\sv(G)}$ as proved by Ravsky and Verbitsky~\cite{ravsky.verbitsky:on}.
%%as implied by Theorem 2 in \cite{dujmovic:utility}. 
%This bound, $v(G)\in \Omega(\sqrt{\sv(G)})$, was not strong enough for
%any novel results in the graph drawing applications of free sets. 
%Thanks to \thmref{our-bang}, we now know a much better and more useful bound, in fact the ultimate $v(G)=
%\tv(G) = \sv(G)$ relationship. This relationship was
%previously only known for planar $3$-trees
%\cite{dalozzo.dujmovic.ea:drawing}. Ravsky and Verbitsky
%\cite{ravsky.verbitsky:on} showed earlier that $2$-trees have large, $\Omega(n)$,
%free-colinear sets, however their result did not give
%\thmref{our-bang} for $2$-trees. 


\begin{cor}\corlabel{our-all}
For every planar graph $G$ and every $S\subseteq V(G)$, $S$ is a free set if
and only if it is a \mbox{collinear set}.
\end{cor}

%\corref{our-all} is a corollary of  \thmref{our-bang} for the
%following reasons. 
That every free set is a collinear set is immediate. \thmref{our-bang} then implies \corref{our-all} since every free collinear set is also a free set. 
This fact, which implies that $v(G)=\tv(G)$, has been observed by several
authors~\cite{bose.dujmovic.ea:polynomial,dalozzo.dujmovic.ea:drawing,dujmovic:utility,gkossw-upg-09}. To
see this,
let $X=\{(x_1,y_1),\ldots,(x_{|S|},y_{|S|})\}$ be a point set in which
no two points have the same y-coordinate, and let
$X_0=\{(0,y_1),\ldots,(0,y_{|S|})\}$.  By the definition of free
collinear set, $G$ has a plane straight-line drawing $\Gamma_0$ in
which $S$ maps to $X_0$.  Since the set of plane straight-line drawings of
$G$ is an open set, there exists some $\epsilon >0$ such that $G$ has a
plane straight-line drawing $\Gamma_{\epsilon}$ in which $S$ maps to
$X_\epsilon=\{(\epsilon x_1,y_1),\ldots,(\epsilon x_{|S|},y_{|S|})\}$.
Dividing all the $x$-coordinates of $\Gamma_\epsilon$ by $\epsilon$ then
yields a plane straight-line drawing $\Gamma$ in which $S$ maps to
$X$. 

Thus, \thmref{our-bang} is our main result and this paper is dedicated to
proving it. The following
characterization of collinear sets by Da Lozzo \etal\
\cite{dalozzo.dujmovic.ea:drawing}  is helpful in that goal.

\begin{thm}\cite{dalozzo.dujmovic.ea:drawing} \thmlabel{collinear-set}
	A set $S$ of the vertices of a graph $G$ is a collinear set if and
	only if there is a plane drawing of $G$ and a Jordan curve $C$
	that contains every vertex in $S$, that intersects the interior of
	at least one face of $G$, and whose intersection with
	each edge of $G$ is either empty, a single point, or the entire edge.
\end{thm}

 \thmref{collinear-set} is helpful because it reduces the problem of
finding large collinear sets in a graph $G$ to a topological game in
which one only needs to find a curve that contains many vertices
of $G$.  Indeed, Da Lozzo \etal\ used \thmref{collinear-set} to give
tight lower bounds on the sizes of collinear sets in planar graphs
of treewidth at most 3 and triconnected cubic planar graphs. Despite the conceptual simplification provided by \thmref{collinear-set},
the identification of collinear sets is highly non-trivial:
Mchedlidze, Radermacher, and Rutter~%\etal\
\cite{mchedlidze.radermacher.ea:aligned} showed that it is NP-hard to
determine if a given set of vertices in a planar graph is a collinear
set.
%
Nevertheless, \thmref{collinear-set} is a useful tool for finding large 
collinear sets. This in combination with \corref{our-all} gives a useful
tool for finding free sets, which have a wide variety of applications,
as outlined in the next section.


\subsection{Applications and Related Work}

The applicability of \corref{our-all} comes from the fact that a number of graph drawing applications require (large) free sets, whereas finding large collinear sets
is an easier task. Indeed there are planar graphs for which large collinear sets were known to exist, however large free sets were not. Those include 3-connected cubic planar graphs
and planar graphs of treewidth at least $k$.
%
%Free collinear sets have a number of applications in graph drawing and related areas. 
We now review applications of our result. 

%A \emph{geometric graph} is a graph $G$ whose vertices are distinct
%points in the plane (not necessarily in general position) and whose
%edges are straight-line segments between pairs of points.  If the
%underlying combinatorial graph of $G$ belongs to a class of graphs
%$\mathcal K$, then we say that $G$ is a \emph{geometric $\mathcal K$ graph}. 

%\paragraph{Untangling} \cite{bose.dujmovic.ea:polynomial,cano.toth.ea:upper,c-upg-10,dalozzo.dujmovic.ea:drawing,dujmovic:utility,gkossw-upg-09,kpr-upg-11,pt-up-02,ravsky.verbitsky:on,ravsky.verbitsky:on-arxiv}

\paragraph{Untangling.}  Given a straight-line drawing of a planar
graph $G$, possibly with crossings, to \emph{untangle} it means to assign
new locations to some of the vertices of $G$ so that the resulting
straight-line drawing of $G$ becomes noncrossing. The goal is to do so while
\emph{keeping fixed}
as many vertices as possible (that is, not changing their location). In 1998 Watanabe asked if every polygon can be untangled while keeping at least $\varepsilon n$ vertices
fixed, for some $\varepsilon >0$. Pach and Tardos\cite{pt-up-02} answered that question in
the negative by providing an $\mathcal{O}((n\log n)^{2/3})$ upper bound on the
number of fixed vertices. This has almost been  matched by
an 
$\Omega(n^{2/3})$ lower bound by Cibulka~\cite{c-upg-10}. Several papers have studied the untangling
problem~\cite{pt-up-02,cano.toth.ea:upper,c-upg-10,bose.dujmovic.ea:polynomial,gkossw-upg-09, kpr-upg-11,ravsky.verbitsky:on}. Asymptotically tight
bounds are known for paths \cite{c-upg-10}, trees \cite{gkossw-upg-09}, outerplanar graphs
\cite{gkossw-upg-09}, and planar graphs of treewidth two and three \cite{ravsky.verbitsky:on,
  dalozzo.dujmovic.ea:drawing}. For general
planar graphs there is still a large gap. Namely, it is known that every planar graph can be untangled while
keeping $\Omega(n^{0.25})$ vertices fixed
\cite{bose.dujmovic.ea:polynomial} (this answered a 
question by Pach and Tardos \cite{pt-up-02})  and that there are planar graphs
that cannot be untangled while keeping $\Omega(n^{0.4948})$ vertices
fixed \cite{cano.toth.ea:upper}. \thmref{our-bang} can help close this gap, whenever a good bound
on collinear sets is known.  %The connection between untangling and
                              %free sets comes from the following. 
Namely, Bose \etal\cite{bose.dujmovic.ea:polynomial} implicitly and  Ravsky and Verbitsky
\cite{ravsky.verbitsky:on} explicitly, proved that every straight-line
drawing of a planar graph $G$ can be untangled while keeping
$\Omega(\sqrt{|S|})$ vertices fixed, where $S$ is a free set of
$G$. Together with \corref{our-all} this implies that, for untangling, it is enough to
find large collinear sets.

\begin{thm}\thmlabel{our-untang}
Let $S$ be a collinear set of a planar graph $G$. Every straight-line drawing of $G$ can be untangled while keeping $\Omega(\sqrt{|S|})$ vertices fixed.
\end{thm}

Da Lozzo,
Dujmovi\'c, Frati, Mchedlidze, and Roselli%\etal
~\cite{dalozzo.dujmovic.ea:drawing}  proved that
every 3-connected cubic planar graph has a collinear set of size
$\Omega(n)$. Then \thmref{our-untang} implies the following new result,
for which $\Omega(n^{0.25})$ was a previously best known \mbox{untangling
bound.}
%  Note that triconnected cubic planar
%graphs form a rich subclass of planar graphs as they are duals of
%(non-trivial) triangulations.

\begin{cor}\corlabel{our-cubic-unt}
Every straight-line drawing of any $n$-vertex triconnected cubic planar
graph can be untangled while keeping $\Omega(\sqrt{n})$ vertices fixed.
\end{cor}

\corref{our-cubic-unt} is almost tight due to the $\mathcal{O}(\sqrt{n\log^3n })$ upper bound for 3-connected cubic planar graphs of diameter $\mathcal{O}(\log n)$ \cite{c-upg-10}. \corref{our-cubic-unt} cannot be extended to all bounded-degree planar graphs, see \cite{dujmovic:utility,DBLP:journals/dm/Owens81} for
reasons why.  Da Lozzo \etal~\cite{dalozzo.dujmovic.ea:drawing} also proved that planar graphs of treewidth at least
$k$ have $\Omega(k^2)$-size collinear sets. Together with
\thmref{our-untang}, this implies that 
%
%, the following:
%\begin{cor}\corlabel{our-tw}
every straight-line drawing of an $n$-vertex planar graph of treewidth
at least $k$ can be untangled while keeping $\Omega(k)$ vertices fixed. 
%\end{cor}
%
This gives, for example, a tight $\Theta(\sqrt{n})$
untangling bound for planar graphs of treewidth
$\Theta(\sqrt{n})$.  


%Many of these are outlined by Dujmovi\'c \cite{dujmovic:utility}, who will write the rest of this section\ldots

 \paragraph{Universal Point Subsets.}%~\cite{abehlmmo-ups-12,dalozzo.dujmovic.ea:drawing,dujmovic:utility},

% %A set of points $P$ is \emph{universal} for a set of planar graphs if
% %every graph from the set has a plane straight-line drawing where
% %each of its vertices maps to a distinct point in $P$.   It is known
% %that,  for all large enough $n$,  no universal pointset of size $n$
% %exists for all $n$-vertex planar graphs -- as first proved by de
% %Fraysseix~\etal~\cite{dFPP90}. Currently the best known lower bound
% %on the size of a smallest universal pointset for $n$-vertex planar
% %graphs is $1.235n-o(n)$ \cite{DBLP:journals/ipl/Kurowski04} and the
% %best known upper bound is $n^2/4 - O(n)$
% %\cite{DBLP:journals/jgaa/BannisterCDE14}. 

Closing the gap between $\Omega(n)$ and $\mathcal{O}(n^2)$ on the size of the
smallest \emph{universal point set} (a set of points on which
every $n$-vertex planar graph can be drawn with straight edges by picking $n$ of these
points as locations for the vertices.) is a major, extensively studied, and difficult graph
drawing problem, open since
$1988$~\cite{%deFraysseix:1988:SSS:62212.62254,
  dFPP90, DBLP:journals/ipl/Kurowski04, DBLP:journals/jgaa/BannisterCDE14}. The interest in this problem motivated the following notion introduced by Angelini~\etal~\cite{abehlmmo-ups-12}. 
A \emph{universal point subset} for  a set $\mathcal{G}$ of $n$-vertex planar graphs is a
set $P$ of $k\leq n$ points in the plane such that, for every
$G\in\mathcal{G}$, there is a plane straight-line
drawing of $G$ in which $k$ vertices of $G$ are mapped to the $k$
points in $P$. Every set of $n$ points in general position is a
universal point subset for $n$-vertex outerplanar graphs
\cite{GMPP,DBLP:journals/comgeo/Bose02,DBLP:conf/cccg/CastanedaU96};  every
set of $\lceil \frac{n-3}{8}\rceil$ points in the plane is a universal
point subset for the $n$-vertex planar graphs of treewidth at most
three \cite{dalozzo.dujmovic.ea:drawing}; and, every set of $\sqrt{\frac{n}{2}}$ points in the plane is a universal point
subset for the $n$-vertex planar graphs \cite{dujmovic:utility}. Dujmovi\'c~\cite{dujmovic:utility}
  proved that every set of $v(G)$ points in the plane is a universal point subset
  for a planar graph $G$. Together with \corref{our-all} this implies
  that, in order to find large universal point subsets, it is enough to look for large collinear sets.

\begin{thm}\thmlabel{our-subset}
Let $S$ be a collinear set for a graph $G$. Then every set of $|S|$ points in the
plane is a universal point subset for $G$.
\end{thm}

As was the case with untangling, \thmref{our-subset} implies new results
for universal point subsets of 3-connected cubic planar graphs and
treewidth-$k$ planar graphs. In particular, \thmref{our-subset} and the
fact that every triconnected cubic planar graph has a collinear set of
size $\ceil{\frac{n}{4}}$ \cite{dalozzo.dujmovic.ea:drawing} imply the
following asymptotically tight result. The previously best known bound
was $\Omega(\sqrt{n})$ \cite{dujmovic:utility}.

\begin{cor}\corlabel{our-cubic-sub}
Every set of $\ceil{\frac{n}{4}}$ points in the plane  is a universal
point subset for  every $n$-vertex 3-connected cubic
planar graph.
\end{cor}

Similarly, \thmref{our-subset} and the fact that planar graphs of
treewidth at least $k$ have collinear sets of size $ck^2$, for some
constant $c$ \cite{dalozzo.dujmovic.ea:drawing}, imply that every set
of  $c k^2$ points in the plane is a universal point subset for  such
graphs. This gives, for example, an asymptotically tight  $\Theta(n)$
results on the size of the largest universal point subset for planar
graphs of treewidth $\Theta(\sqrt{n})$. 

For similar applications of \thmref{our-bang}
and \corref{our-all}, such as \emph{column
planarity}~\cite{behks-cppsge-17,dalozzo.dujmovic.ea:drawing,dujmovic:utility}
and \emph{partial simultaneous geometric embeddings with and without
mappings}~\cite{behks-cppsge-17,ddlmw-pqp-15,dujmovic:utility} see a
survey by Dujmovi\'c~\cite{dujmovic:utility}.


% Cano \etal\ \cite[Theorem~2]{cano.toth.ea:upper} show that if a Jordan
% curve $C$ intersects each edge of a plane drawing of a graph $G$ in
% at most one point and does not contain any vertex of $G$, then $G$ has
% a straight-line plane drawing in which the edges of $G$ intersected
% by $C$ become line segments that cross the $y$-axis, and these crossings
% occur in the same order.  A restatement of \thmref{collinear-set} that
% we describe as \thmref{dujmovic-frati} in \secref{definitions} gives an
% extension of this result to curves that include vertices of $G$.

\subsection{Proof Outline for \thmref{our-bang}}

We assume w.l.o.g.\ that $G$ is a plane straight-line
drawing in which the collinear set $S\subseteq V(G)$ lies
on the $y$-axis $Y=\{(0,y):y\in\R\}$. Let
$L=\{(x,y)\in\R^2:x<0\}$ and $R=\{(x,y)\in\R^2: x >0\}$ denote the open
halfplanes to the left and right of $Y$. When talking about the order
of points on the $y$-axis $Y$, we are referring to the total order
$\prec_Y$ in which $(0,a) \prec_Y (0,b)$ if and only if $a<b$. We
assume, furthermore, that we are given $|S|$ distinct $y$-coordinates,
and the goal is to find another plane straight-line drawing of $G$ in
which the vertices in $S$ are positioned on $Y$ with the given $y$-coordinates.

If the vertices in $S$ induce a path with
both endvertices on the outer face of $G$, then no edge of $G$ crosses
$Y$. Then it is easy to prove that $S$ is a free
collinear set by separately drawing
the graphs induced
by $V(G)\cap(L\cup Y)$ and $V(G)\cap(Y\cup R)$ on the two sides of
$Y$.
For this task, we can apply
Tutte's Convex Drawing Theorem~\cite{tutte:how}
to suitable augmentations
of these graphs.
This theorem
gives a plane
straight-line drawing of an internally 3-connected graph, with the
outer face drawn as any prescribed convex polygon having the
correct number of vertices.
%using two applications of \cite{tutte:how}, on


Thus, the main difficulty comes from edges of $G$ that cross $Y$.
These edges must cross $Y$ in prescribed
intervals between the prescribed locations of vertices in $S$, and
these intervals may be arbitrarily small.  An extreme version of this
subproblem is the one in which $G$ is a drawing where every
edge intersects $Y$ in exactly one point (possibly an endpoint) and
the location of each crossing point is prescribed.  The most difficult
instances occur when $G$ is edge-maximal.

In \secref{quadrangulations} we describe these edge-maximal graphs, which
we call A-graphs.  A-graphs are a generalization of quadrangulations, in
which every face is either a quadrangle whose every edge intersects $Y$ or a triangle with one vertex in
each of $L$, $Y$, and $R$.  \thmref{a-graph} %in this section
shows that it
is possible to find a plane straight-line drawing of any A-graph where
the intersections of the drawing with $Y$ occur at prescribed locations.
This is done by showing that a certain system of linear equations has
a solution. This proof involves linear algebra and
 continuity arguments.

In \secref{triangulations} we prove that every collinear set is free.
The technical statement of this result, \thmref{main}, shows a somewhat
stronger result for triangulations that makes it possible not only to
prescribe the locations of vertices on $Y$ but also to nearly prescribe the
points at which edges of the triangulation cross $Y$.  This proof uses combinatorial
reductions that are applied to a triangulation $T$ that either reduce its
size or increase the number of edges that cross $Y$.  When none of these
reductions is applicable to $T$, removing the edges of $T$ that do not
cross $Y$ creates an A-graph, $G$, on which we can apply \thmref{a-graph}.

\secref{definitions}, next, begins our discussion with definitions and results that we use throughout.

%%% Local Variables:
%%% mode: latex
%%% TeX-master: "freecoll"
%%% End:
